\chapter{Shell Commands}

\section{基本知识}

\section{执行}

\begin{Command-Line}[开头告诉shell使用什么bash]
#!/bin/bash
\end{Command-Line}


执行脚本:

\begin{itemize}
\item 脚本作为参数的方式
\begin{Command-Line}
bash script.sh   #脚本位于工作目录(当前目录)
\end{Command-Line}

\item 修改脚本成可执行文件

\begin{Command-Line}[内核会读取首行来确定执行的bash]
chmod a+x script.sh
./script.sh
\end{Command-Line}

\end{itemize}

\section{变量}

变量名就是一个他引用的value的占位符。需要区分变量名,和变量名引用的值。如果variable1是变量名的话。\$variable1是变量名引用的值

变量不需要使用前置\$的场景:
\begin{itemize}
\item 声明和赋值的时候;
\item \lstinline$unset$和\lstinline$export$的时候;
\item 数学表达式((...))
\item loop的头中
\item ...
\end{itemize}


\subsubsection{对变量求值}



\subsubsection{参数}

\begin{itemize}
\item \$0, \$1, \$2, ... \${10}, \${11}, etc!,对应位置的参数,\$0表示的是命令名,参数是从\$1开始。
\item \$\#参数个数。
\item \$* 所有参数,但是是当做一个参数
\item \$@ 所有参数,但是是当做分开的参数
\item \lstinline$shift$ 是的所有参数都往前移动一位,\$1被丢弃。
\end{itemize}

\begin{Command-Line}[\$*和\$@的区别]
x
x
x
\end{Command-Line}

\section{String}

\section{exit和exit的状态}

可以使用\$?来取得最后执行的命令的exit status的值,0表示success。


在函数中,可以使用return来返回exit status。如果没有返回,则最后执行的一条语句的exit status就是函数的exit status。

\section{测试}

\subsubsection{if, else}

对于ifelse,其实就是判断条件的exit status来看执行哪个分支,0表示真,1表示false,因为在Unix中,0状态码表示success。

\subsubsection{test}

\lstinline$test$, \lstinline$[$

test的参数被当做comparison expression或者file test。根据比较结果来返回exit status,0为true,1为false。

\lstinline$[$是built-in命令;

\lstinline$test expr$

\lstinline$[expr]$

\begin{itemize}
\item !expr
\item (expr)
\item expr1 -a expr2
\item expr1 -o expr2 
\end{itemize}

expr计算结果按下面的规则根据参数的个数来返回结果
\begin{itemize}
\item 0 argument,结果是false
\item 1 argument,当argument不为null的时候,结果是true
\item 2 arguments

当第一个参数是!,之后当第二个参数是null的时候结果才为true

\begin{Command-Line}

上面的实例

\end{Command-Line}

当地一个参数是unary操作符,结果就是unary操作的结果;如果第一个参数是个非法的unary,则结果是false。


\item 3 arguments 

如果第二个参数是binary操作符,结果就是将第一个参数和第三个参数做为操作数,binary的结果。

如果第一个参数是!,那么结果就是将第二个和第三个参数按上面的描述执行

如果第一个参数是(,第三个参数是),则第二个参数按照一个参数执行方式执行,结果就是他。

否则,表达式结果是false。

\item 4 arguments

如果第一个是!,后面三个就按照三个参数的方式计算,否则按优先级使用上面的规则

\item 5 arguments

按优先级使用上面的规则

\end{itemize} 


\paragraph{File test Operators}

\begin{itemize}
\item -e, -a file exist, 判断文件是否存在,-a已经不推荐使用。
\item -a

\item -s file is not zero size.

\item -f
\item -d
\item -b
\item -c
\item -p
\item -h
\item -L
\item -S
\item -t

\item -r
\item -w
\item -x

\item -g
\item -u
\item -k

\item -O you are owner of file.
\item -G group-id of file same as yours.
\item -N file modified since it was last read.

\item f1 -nt f2 
\item f1 -ot f2
\item f1 -ef f2

\item !
\end{itemize}


\paragraph{Integer comparison}

\begin{itemize}
\item -eq
\item -ne
\item -gt
\item -ge
\item -lt
\item -le
\end{itemize}


\paragraph{String comparison}

\begin{itemize}
\item =
\item ==
\item !=

\item \lstinline$\<$
\item \lstinline$\>$

\item -z is null
\item -n is not null
\end{itemize}

\lstinline$[$和\lstinline$[[$中=和==不同的场景

test有\lstinline$[$和\lstinline$[[$

\subsubsection{let \& ((...))}

((...)) and let construct return an exit status, according to whether the arithmetic expressions they evaluate expand to a non-zero value.

\begin{Command-Line}
[root@192 tmp]# ((0 && 1))
[root@192 tmp]# echo $?
1
[root@192 tmp]# let "num = ((0 && 1))"
[root@192 tmp]# echo $num
0
[root@192 tmp]# let "num = ((0 && 1))"
[root@192 tmp]# echo $?
1
[root@192 tmp]# ((200 || 11))
[root@192 tmp]# echo $?
0
[root@192 tmp]# let "num = ((200 && 1))"
[root@192 tmp]# echo $num
1
[root@192 tmp]# let "num = ((200 && 1))"
[root@192 tmp]# echo $?
0
\end{Command-Line}

这里结果都很明确,搞清楚数学表达式的值和let及\lstinline$((...))$exit status的关系就可以了。

数学表达式的值非零的话,let和\lstinline$((...))$的exit status就是0,数学表达式的值为零的时候,exit status就为1.

然后就是在各种语言中\lstinline$num=2+1$的值就是num的值3,比如说c语言中经常出现的语句\lstinline$if((c=getchar()) != EOF){...}$


\lstinline$((...))$的Integer Comparison
\begin{itemize}
\item <
\item <=
\item >
\item >=
\end{itemize}


\section{运算符}


\section{sed \& awk}

\subsection{基本操作}

sed和awk的共同点:
\begin{itemize}
\item 相似的语法来调用。
\item 面向字符流,都是从文本文件中一次一行的读取输入,并将输出直接送到标准输出端。
\item 都使用正则表达式进行模式匹配。
\item 都允许用户在脚本中指定指令。
\end{itemize}

感觉重点就是第二个,从文本中一次一行的读取输入。


命令行的语法:
\begin{Command-Line}
command [option] script filename
\end{Command-Line}

sed和awk也可以从标准输入中读取输入,并将输出发送到标准输出;如果制定了文件名filename,则输入从这个文件中取得。在重定向输出的时候,是不能够指定为输入的同一个文件的。

script指定了要执行的指令,如果是在命令行,假如它包含有可以由shell解释的空格或者任意字符,那么他必须用单括号括起来。

sed和awk都有-f选项可以指定脚本文件的名字。

\begin{Command-Line}
sed -f scriptfile inputfile
\end{Command-Line}

sed和awk基本操作是,每次读取一行输入,生成该输入行的备份,并且对该备份执行脚本中指定的指令,因此,对输入行所做的改动不会影响正真的输入文件。


\subsubsection{sed处理过程}
sed和awk的每个指令都包含两部分:模式和过程。模式是由斜杠(/)分割的正则表达式。过程指定一个或者多个将被执行的动作。

当读取输入的每行时,程序赌气脚本中的第一个指令并检测当前行的模式,如果没有没有匹配,过程被忽略,并读取下一条指令;如果有一个匹配,那么执行过程中指定的一个或者多个动作。读取所有的指令,而不只是读取与输入行匹配的第一条指令。

过程sed类似于航编辑器中使用的那些编辑命令组成,而awk中则可以使用程序设计语句和函数组成(过程必须用大括号括起)。


\subsubsection{sed示例}
\begin{Command-Line}[用来被处理的文件]
[root@192 tmp]# cat list
John Daggett, 341 King Road, Plymouth MA
Alice Ford, 22 East Broadway, Richmond VA
Orville Thomas, 11345 Oak Bridge Road, Tulsa OK
Terry Kalkas, 402 Lans Road, Beaver Falls PA
Eric Adams, 20 Post Road Sudbury MA
Hubert Sims, 328A Brook Road, Roanoke VA
Amy Wilde, 334 Bayshore Pkwy, Mountain View CA
Sal Carpenter, 73 6th Street, Boston MA
\end{Command-Line}

\begin{Command-Line}[将MA替换成Massachusetts]
[root@192 tmp]# sed 's/MA/Massachusetts/' list
John Daggett, 341 King Road, Plymouth Massachusetts
Alice Ford, 22 East Broadway, Richmond VA
Orville Thomas, 11345 Oak Bridge Road, Tulsa OK
Terry Kalkas, 402 Lans Road, Beaver Falls PA
Eric Adams, 20 Post Road Sudbury Massachusetts
Hubert Sims, 328A Brook Road, Roanoke VA
Amy Wilde, 334 Bayshore Pkwy, Mountain View CA
Sal Carpenter, 73 6th Street, Boston Massachusetts
\end{Command-Line}

这里使用单引号,是因为这样使得对于shell来说是特殊字符的字符表达本身的意思。

\begin{Command-Line}[多条指令的时候的两种方式]
[root@192 tmp]# sed 's/MA/, Massachusetts/; s/PA/, Pennsylvania/' list
[root@192 tmp]# sed -e 's/MA/, Massachusetts/' -e 's/PA/, Pennsylvania/' list
\end{Command-Line}

一种是通过分号,另外一种是使用-e选项。

-f来执行脚本文件是创建长的指令的好的方式。

-n选项可以阻止自动输出,这个时候,如果要输出的指令都必须包含打印命令p(如果不带-n的时候,使用了p指令,那么一个被处理了的行会输出多次)

\begin{Command-Line}
sed -n 's/MA/Massachusetts/p' list
\end{Command-Line}

\subsubsection{awk}

基本上命令行和sed类似,读取标准输入或者文件,输出到标准输出;多条指令使用";"隔开;-f读取脚本文件等等。

但是,awk会将每个输入行解释成一条记录,而将一行上的每个单词(由空格或者制表符分隔)解释为每一个字段(这些默认设置可以修改的)。可以在模式或者过程中引用这些字段。\$0代表整个记录。

\begin{Command-Line}[没有指定模式的情况下就是所有行]
awk '{print $1}' list
\end{Command-Line}
%$
\begin{Command-Line}[指定模式]
awk '/MA/{print $1}' list
\end{Command-Line}
%$

可以使用-F选项来制定分隔符。

\begin{Command-Line}[-F紧跟着指定分隔符]
awk -F, '/MA/{print $1}' list
\end{Command-Line}
%$

\section{sed脚本}

\subsection{sed命令的语法}

\subsubsection{地址匹配}
\begin{itemize}
\item 没有的时候
\item 有一个的时候
\item 有两个,中间用逗号隔开
\item 匹配后面跟随"!"
\end{itemize}

\subsection{sed的基本原理}

理解sed,需要理解的三个基本原理
\begin{itemize}
\item 脚本中的所有编辑命令都将依次应用于每个输入行;
\item 命令应用于所有的行,除非寻址限制了受编辑命令影响的行;
\item 原始的输入未被改变,编辑命令修改了原始行的备份并且此备份被发送到标准输出。
\end{itemize}

\subsubsection{所有编辑命令依次应用于每一个输入行}

对已后面的命令,他处理的是前一个命令处理过的行的内容,而不是原始的行的内容。

% 举个例子

\subsubsection{寻址是全局的,对应所有的行}

\subsubsection{原始输入不会改变,结果发送到标准输出}


\subsection{基本的sed命令}

sed命令集合由25个命令组成.

命令的语法


\begin{itemize}
\item 行地址,可选的,如果不存在就是对所有行都做处理。被描述为斜杠,行号,或者航寻址符号阔住的正则表达式。大多数sed命令接收由逗号分隔的两个地址,这两个地址标识行的范围。有些命令只接受单个行地址。

\item 命令还可以使用大括号进行分组以使其作用于同一地址。左大括号放置在命令的同一行,但是右大括号必须单独处于一行。

\item 命令后面不能有空格。

\item 如果多个命令放在同一行,可以用分号隔开。



\end{itemize}

\subsubsection{s 替换}


\begin{Command-Line}
[address]s/patern/replacement/flags
\end{Command-Line}

address:
必须使用斜杠

flags,flag可以组合使用的,只要有意义就可以:
\begin{itemize}
\item n 1~512之间的一个数字,表示对文本中指定模式第n次出现的情况进行替换
\item g 多所有情况进行全局更改,没有g的情况下,只有第一个匹配被替换
\item p 打印模式空间的内容
\item W file,将模式空间的内容写道文件file中去。
\end{itemize}

pattern:
不需要使用斜杠,可以是除换行外的任意字符。这样档模式中有斜杠的时候,可以使用其他字符来分割。

replacement:
有一些有特殊意义的字符:
\begin{itemize}
\item \lstinline$&$ 用正则表达式匹配的内容进行替换.
\item \lstinline$\n$ 匹配的第n个字符串(\lstinline$"\(", "\)"$中括起来的指定)。
\item \lstinline$\$ 转义符。
\end{itemize}


\subsubsection{d 删除}

删除命令可以改变脚本中的控制流,一旦执行删除,那么在“空的”模式空间中就不会再有命令执行。删除命令会导致读取新输入行,而编辑脚本则从头开始新的一轮。

d是删除整行的命令,而不是删除匹配的部分。

可以删除一行,也可以删除多行

\subsubsection{a 追加,i 插入, c 更改}



\subsubsection{| 列表}

\subsubsection{y 转换}

按位置将字符串abc中的每个字符,都转换成字符串xyz中的等价字符串。




