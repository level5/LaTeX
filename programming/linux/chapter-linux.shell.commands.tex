\chapter{Shell Commands}

\section{基本知识}

\subsection{执行}

\begin{Bash}[开头告诉shell使用什么bash]
#!/bin/bash
\end{Bash}


执行脚本:

\begin{itemize}
\item 脚本作为参数的方式
\begin{Bash}
bash script.sh   #脚本位于工作目录(当前目录)
\end{Bash}

\item 修改脚本成可执行文件

\begin{Bash}[内核会读取首行来确定执行的bash]
chmod a+x script.sh
./script.sh
\end{Bash}

\end{itemize}

\subsection{变量}

变量名就是一个他引用的value的占位符。需要区分变量名,和变量名引用的值。如果variable1是变量名的话。\$variable1是变量名引用的值

变量不需要使用前置\$的场景:
\begin{itemize}
\item 声明和赋值的时候;
\item \lstinline$unset$和\lstinline$export$的时候;
\item 数学表达式((...))
\item loop的头中
\item ...
\end{itemize}

\subsection{String}

\subsection{exit和exit的状态}

\subsection{测试}

\subsubsection{if, else}

对于ifelse,其实就是判断条件的exit status来看执行哪个分支,0表示真,1表示false,因为在Unix中,0状态码表示success。

\subsubsection{test}

test的参数被当做comparison expression或者file test。根据比较结果来返回exit status,0为true,1为false。

\lstinline$[$是built-in命令;



test有\lstinline$[$和\lstinline$[[$

\subsubsection{let \& ((...))}

((...)) and let construct return an exit status, according to whether the arithmetic expressions they evaluate expand to a non-zero value.

\begin{Command-Line}
[root@192 tmp]# ((0 && 1))
[root@192 tmp]# echo $?
1
[root@192 tmp]# let "num = ((0 && 1))"
[root@192 tmp]# echo $num
0
[root@192 tmp]# let "num = ((0 && 1))"
[root@192 tmp]# echo $?
1
[root@192 tmp]# ((200 || 11))
[root@192 tmp]# echo $?
0
[root@192 tmp]# let "num = ((200 && 1))"
[root@192 tmp]# echo $num
1
[root@192 tmp]# let "num = ((200 && 1))"
[root@192 tmp]# echo $?
0
\end{Command-Line}

这里结果都很明确,搞清楚数学表达式的值和let及\lstinline$((...))$exit status的关系就可以了。

数学表达式的值非零的话,let和\lstinline$((...))$的exit status就是0,数学表达式的值为零的时候,exit status就为1.

然后就是在各种语言中\lstinline$num=2+1$的值就是num的值3,比如说c语言中经常出现的语句\lstinline$if((c=getchar()) != EOF){...}$

\subsection{运算符}


