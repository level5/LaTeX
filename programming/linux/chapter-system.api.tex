\chapter{System API}


\section{进程环境}

\subsection{进程终止}


\subsubsection{进程终止方式}
进程终止方式:
\begin{itemize}
\item 正常方式:
\begin{itemize}
\item 从\lstinline$main$返回
\item 调用\lstinline$exit$
\item 调用\lstinline$_exit$或者\lstinline$_Exit$
\item 最后一个线程从其启动例程返回
\item 最后一个线程调用\lstinline$pthread_exit$
\end{itemize}
\item 异常方式:
\begin{itemize}
\item 调用\lstinline$abort$
\item 接收到一个信号并终止
\item 最后一个线程对取消请求作出响应
\end{itemize}
\end{itemize}

\subsubsection{exit函数}

\begin{C}
#include <stdlib.h>

void exit(int status);
void _Exit(int status);
\end{C}

\begin{C}
#include <unistd.h>
void _exit(int status);
\end{C}


\lstinline$_exit$和\lstinline$_Exit$终止程序之后立即进入内核;\lstinline$exit$函数会先执行一些清理处理,执行个终止处理程序,关闭所有的标准IO流。


\subsubsection{atexit函数}

\begin{C}
#include <stdlib.h>

int atexit(void (*func)(void));
\end{C}

\begin{C}
#include "apue.h"

static void my_exit1(void);
static void my_exit2(void);

void test_atexit(void)
{
	if (atexit(my_exit2) != 0)
		err_sys("cannot register my_exit2");
	if(atexit(my_exit1) != 0)
		err_sys("cannot register my_exit1");
	if(atexit(my_exit1) != 0)
		err_sys("cannot register my_exit1");

	printf("test_atexit is done\n");
}

static void my_exit1()
{
	printf("first exit handler.\n");
}

static void my_exit2()
{
	printf("second exit handler.\n");
}
\end{C}

\begin{Command-Line}[执行结果]
test_atexit is done
first exit handler.
first exit handler.
second exit handler.
\end{Command-Line}


\subsection{C程序内存布局}

\subsubsection{C程序的存储空间布局}


\begin{itemize}
\item 正文段,CPU执行的机器指令部分;
\item 初始化数据段,程序中明显的赋初值的变量。
\begin{C}[出现在任何函数之外的声明]
int max_count = 99;
\end{C}
\item 非初始化数据段,在程序开始之前,内核将此段中的数据初始化为0或者空指针。
\begin{C}[出现在任何函数外的C声明]
long sum[1000];
\end{C}
\item 栈,自动变量以及每次函数调用时所需要保存的信息放在此段中;
\item 堆,动态存储分配
\end{itemize}

\subsubsection{共享库}

\subsubsection{动态存储分配}

\begin{C}
#include <stdlib.h>

void *malloc(size_t size);
void *calloc(size_t nobj, size_t size);
void *realloc(void *ptr, size_t newsize);

void free(void *ptr);
\end{C}

三个函数返回的都是\lstinline$void *$通用指针。内存要满足最苛刻的对齐要求。
\begin{itemize}
\item \lstinline$malloc$,分配指定字节数的存储区,初始值不确定。
\item \lstinline$calloc$,分配给指定长度的具有指定长度的对象存储空间,每位都初始化为0;
\item \lstinline$realloc$,增加或者减少一起分配区的长度,当增长长度时,可能需要将以前分配区的内容移动到另外一个足够大的区域,新增区域内的初始值不确定。最后一个参数是存储区的新长度,而不是增加或者减少的长度。
\end{itemize}

\subsubsection{环境变量}

\begin{C}
#include <stdlib.h>

char *getenv(const char *name);
\end{C}

\begin{C}[成功返回0]
#include <stdlib.h>

int putenv(char *str);

int setenv(const char *name, const char *vlaue, int rewrite);
int unsetenv(const char *name);
\end{C}

\begin{itemize}
\item \lstinline$putenv$参数形式为name = value,如果已存在先删除其原来的定义。
\item \lstinline$setenv$,若rewrite非0,先铲除现有定义,若rewrite为0,则不修改现有定义,也不报错
\item \lstinline$unsetenv$,删除定义,没有也不报错。
\end{itemize}

由于环境变量表和环境变量字符串通常存放在进程存储空间的顶部,所以修改他的过程比较复杂。
%  以后补充
\begin{itemize}
\item 删除,...
\item 修改,...
\item 新增,...
\end{itemize}

\subsection{跨函数跳转 setjmp和longjmp}