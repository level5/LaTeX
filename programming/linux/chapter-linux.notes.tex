\chapter{Notes}

\section{每天一条命令}
命令列表:


find, wc, dd, head, curl,



================================



grep, iptables, \$((...)) ,  gzip, cut

sed和awk

yum, rpm


tree, 



\subsection{find}

find列出指定目录及其子目录的文件和文件夹

\begin{Bash}[查找当前目录及子目录的文件和文件夹]
find .
\end{Bash}


\subsubsection{-print, -print0}
\lstinline$-print$, 打印出查找到的文件和文件夹,'\\n'作为文件名的间隔。默认情况下不输入-print也会打印出文件夹和文件名。

但是'\\n'是文件名的合法字符,所以这样可能导致错误的文件名。这个时候可以使用\lstinline$-print0$,这个时候改用'\\0'作为间隔。

\subsubsection{-name, -iname}

\lstinline$-name$指定了文件名必须匹配的字符串,可以含有通配符。\lstinline$-iname$作用和\lstinline$-name$一样,但是不区分大小写。

\begin{Bash}[名字匹配指定的字符串]

find . -name 'Lib'

find . -iname 'LIB'

\end{Bash}


\subsubsection{ -path, -regex, -iregex}


而\lstinline$-path$则是匹配路径(包含文件名)是否匹配给定字符串, \lstinline$-regex$和\lstinline$-path$类似,不过是基于正则表达式来匹配路径的。

这部分需要加强,已将《学习正则表达式》加入日程之中。这本书真不怎么样,还是翻出以前那本来比较好。


\subsubsection{否定参数 !}

\begin{Bash}[名字不以.txt结尾的文件]

find . ! -name "*.txt"

\end{Bash}


\subsubsection{基于深度查询 -maxdepth, -mindepth}

\begin{Bash}[查找第二层目录的所有文件]

find . -maxdepth 2 -mindepth 2

\end{Bash}

\begin{Bash}[深度为0的文件就是查找的目录]

find . -maxdepth 0

find /root -maxdepth 0

\end{Bash}

\subsubsection{-type 文件类型}

\begin{Bash}

find . -type f 

find . -type d

\end{Bash}


% 这里使用表格, 今天回家搞定

文件		f

目录		d

符号链接	l

...


\subsubsection{按文件时间}


\begin{itemize}
\item 访问时间 -atime(access time) 最近一次访问时间
\item 修改时间 -mtime(modify time) 文件内容最后一次修改时间
\item 变化时间 -ctime(change time) 文件元数据(权限或者所有权)最后一次改变时间
\end{itemize}


\begin{Bash}[单位是天, 当前为0]

find . -atime -7  # 最近7天访问过的

find . -atime 7	  # 刚好7天前访问过

find . -atime +7  # 访问时间超过7天

\end{Bash}


\subsubsection{-size 基于文件大小}


\begin{Bash}

find . -type f -size +2k 		# 大于2KB的文件

find . -type f -size 2k			# 等于2KB的文件

find . -type f -size -2k		# 小于2KB的文件
\end{Bash}


\subsubsection{-perm 权限}

\subsubsection{-delete}

\begin{Bash}[删除匹配的文件]

find . -type f -perm 644 -delete


\end{Bash}

\subsubsection{-exec 将查找的文件当做输入执行命令}

\subsubsection{多个条件}

\subsubsection{跳过特定目录}

\begin{Bash}[]

find devel/source_path \( -name ".git" -prune \) -o \( -type -f -print \)

\end{Bash}

\subsubsection{结合xargs}

\paragraph{xgrgs的使用}

什么命令都不带时,默认使用的是\lstinline$\bin\echo$.

-d, 输入的截断符号

\begin{Bash}[第一次后面有空行是因为截断符是:,所以最后的换行得到了保留]

[root@192 Development]# echo "args1:args2" | xargs -d :
args1 args2

[root@192 Development]# echo -n "args1:args2" | xargs -d :
args1 args2
[root@192 Development]# 

\end{Bash}

而 -0 则表示使用\lstinline$\0$来截断输入。这样可以配合find命令的\lstinline$-print0$来使用,可以在文件名中有换行的情况下安全使用。


-n, 如果参数超出指定的数量,则每次最多使用多少个参数来多次执行命令。如果-x也只指定了,那么如果参数超出就会退出了。



\subsection{wc 统计文件}

\begin{itemize}
\item -b --bytes
\item -m --chars
\item -l --lines
\item -L --max-line-length
\item -w --wrods
\end{itemize}

\subsection{解压命令 tar}

\begin{Bash}
tar -xzvf xx.tar.gz # -z解压tar.gz

tar -xjvf xx.tar.bz2 # -j解压tar.bz2

# -v 打印处理过程
# -x 解压
# -f 不知道
\end{Bash}


%gzip这条命令也看看吧。


\subsection{创建链接命令 ln}

\begin{Bash}
ln -s item link # 创建symbol link, 如果item是相对位置,则是相对于link的位置。
ln file link # 默认创建硬链接
\end{Bash}


符号链接类似于windows的快捷方式;


硬链接应该就是一个真正指向文件的链接。只有所有的硬链接都删除,文件才会被删除。等书拿出来之后补充。

\begin{Bash}

\end{Bash}

\subsection{判断当前脚本是否是以root用户来执行}


\$(..)看做`..`的另外一种形式。将一条或多条命令的output重新分配。



\begin{Bash}[old way]
#!/bin/bash
# Init
FILE="/tmp/out.$$"
GREP="/bin/grep"
#....
# Make sure only root can run our script
if [ "$(id -u)" != "0" ]; then
   echo "This script must be run as root" 1>&2
   exit 1
fi
# ...
\end{Bash}


\begin{Bash}[new way]
#!/bin/bash
# Init
FILE="/tmp/out.$$"
GREP="/bin/grep"
#....
# Make sure only root can run our script
if [[ $EUID -ne 0 ]]; then
   echo "This script must be run as root" 1>&2
   exit 1
fi
# ...
\end{Bash}

\subsection{收集进程资料}



常用的几条命令,ps,top, pgrep.

\subsubsection{ps}

ps不带任何参数,收集当前终端相关的进程。

\begin{itemize}

\item -f, full,包含更多列信息。

\item -e, every,系统上运行的所有信息,-ax(all)也可以达到同样效果。

%查一下a和x的含义

\item -o,指定想显示的列,用逗号作为限定符,分隔的参数之间没有空格。

\begin{itemize}
\item pcpu
\item pid
\item ppid
\item pmem
\item com, 可执行文件名
\item cmd
\item user
\item nice, 优先级
\item time, 累计CPU时间
\item etime, 启动后流逝的时间
\item tty
\item euid,有效用户ID
\item stat, 进程状态
\end{itemize}

参数后面加上=号,表示移除列名,不是说移除这一列。

% -o使用的过滤器是什么?查一下。

% u呢,是什么啊?

% -f呢?

% ps aux 是怎么说??

\item --sort,对命令进行排序,格式是

\begin{Bash}[参数前的+-表示升序或者降序]
ps [OPTIONS] --sort -parameter1,+parameter2,parameter3
\end{Bash}


\item -C,这里是大写C,指定命令名称.这里是要全称,如果使用pgrep就会好一些,只需要一部分名称就可以了。

%是要全称吗??如果是要全称,那这个不太好用啊。 pgrep是否更好用

\item -t,指定终端

\item -L,线程相关,这里先不关心了。

\item 原来f,u,w是一些制定好输出格式的选项。





\end{itemize}

\subsubsection{pgrep}

-d,指定定界符

-u,指定用户(拥有者)列表

-u和-U,小写是有效用户,大写是真实用户

%最后还是提醒我需要查看用户这部分的知识啊。



\subsubsection{top}

top

\subsubsection{which, whereis, file, whatis}

\paragraph{which} 找出某个命令的位置

\paragraph{whereis} 和which类似,但不仅返回命令的路径,还能够打印出其对应的命令手册的位置以及命令源代码的路径。

\paragraph{file} 可以用来确定文件的类型,这条命令会打印出该文件类型相关的细节信息。



\subsection{杀死进程及发送或者相应信号}

kill可以用来发送信号,trap用来处理所接受的信号

%这部分作为明天的内容算了,睡觉。

\subsection{curl命令}


\begin{itemize}
\item -\# --progress-bar
\item 1-6
	\begin{itemize}
	\item -4
	\item -6
	\end{itemize}
\item -A --user-agent
\item --anyauth
	\begin{itemize}
	\item --basic
	\item --digest
	\item --ntlm ??
	\item --negotiate
	\end{itemize}	
\item -b --cookie
\begin{Bash}
curl -b "NAME1=VALUE1;NAME2=VALUE2" "http://www.google.com"
\end{Bash}	
\item -j --junk-session-cookies

\item -e --referer
\item -E --cert
\item -F --form

\item -G --get
\item -H --header

\item -I --head

\item --interface

\item -k --insecure

\item --socks4a
\item --socks5

\item -T --upload-file

\item --trace

\item -u --user
\item -U --proxy-user

\item --url

\item -x --proxy

\item -X --request


\item -d --data
If  you  start  the data with the letter @, the rest should be a file name to read the data from, or - if you want curl  to  read the  data  from stdin.  The contents of the file must already be URL-encoded. 
\item -c --cookie-jar
\item -D --dump-header
\item -o --output

\end{itemize}

\subsection{dd,生成任意大小的文件}

\begin{Bash}[生成1MB大小的文件,内容为全0]

dd if=/dev/zero of=junk.data bs=1m count=1

\end{Bash}

\subsubsection{/dev/null和/dev/zero}

/dev/zero is an endless source of zeros. It's there to provide a simple raw, data stream of unvarying content. /dev/null is a black hole. Anything sent into it just disappears. It's the programmers trash can.

So /dev/zero is meant as a source of (empty) data, and /dev/null is meant as a destination for (unwanted) data.

/dev/zero产生二进制0的数据;/dev/null是一个黑洞,可以往其中写入数据,都会被丢掉。


\subsection{head \& tail}

打印头部或者尾部;

-c bytes,如果带减号,就是和功能相反的部分,比如 head -c -10,打印除了所有数据,最后10bytes。
-n 行数,也可以带减号
-q --quiet --slient 不显示文件名
-v --verbose 显示文件名


对于tail, -f 持续打印文件增长的内容


\section{shell中我感到疑惑或者不知道的东西}


\subsection{subprocess Vs. subshell}


\section{我记不太清楚的操作符}

\section{Linux中文件和文件权限问题}

\subsection{用户和用户组}

这个没什么好多说的,文件对应的三组权限就是User, Group和Others.

\subsection{文件属性}

\begin{Bash}
[root@192 Development]# ls -l
total 4
-rw-r--r--. 1 root root 54 May 28 01:32 foo.txt
\end{Bash}

文件类型:

% use table to hold other environment, and allows to add a caption to our table. data is contained in tabular environment
\begin{table}
% use center to put table in the center of page
%\begin{center}
\caption{文件类型}    % Caption
%\label{tab:文件属性}  % ??

\begin{tabular}{|l|l|} 
% l,c,r: content align
% |: vertical lines

% ampersands & as column seperators and newline symbols \\ as row seperators.

\hline % row line
- & 文件\\
\hline
d & 目录\\
\hline
l & 连接文件\\
\hline
b & block,可供存储的接口设备\\
\hline
c & char, 串行端口设备\\
\hline

\end{tabular}

%\end{center}

\end{table}

权限:顺序是rwx。
 
r, read读权限,对应的数字是4;w,write写权限,对应的数字是2;x, execute执行权限,对应的数字是1。(这里数字二进制表示的话,分别是第三位,第二位,第一位,这样就可以将所有权限组合起来用1个Byte表示)
 
三组分别代表User,Group,Other的这三者的权限。

第二栏是i-node的数量,3,4栏是User和Group。

\paragraph{i-node}

% 补充...

\paragraph{chgrp, chown}

\begin{Bash}

chgrp -R usersgroup install.log 	# -R进行递归的持续更改

chown users install.log

chown users:usersgroup install.log  # 修改User和Group

chown :usersgroup install.log 		# 只修改Group

\end{Bash}


\paragraph{chmod}

\begin{Bash}

chmod 777 filename # 三个分组的权限都定义为rwx

chmod a+x filename
chmod a-x filename
chmod a=x filename

chmod u=x filename
chmod g=x filename
chomd o=x filename

\end{Bash}


\paragraph{权限对于文件}这个很好理解。

\paragraph{权限对于目录}

\begin{itemize}
\item r(read contents in directory),可以读取目录的结构,表示可以查看该目录下面的文件名数据,所以可以使用命令显示目录的内容列表。
\item w(modify contents of directory),表示可以修改该目录结构列表的权限。
	\begin{itemize}
	\item 建立新的文件和目录;
	\item 删除已存在的文件和目录,不论改文件的权限是什么;
	\item 重命名已存在的文件和目录;
	\item 转移该目录内的文件,目录位置。
	\end{itemize}
\item x(access directory), x表示用户能否进入该目录成为工作目录。
\end{itemize}


\subsubsection{测试一下}

用户相关文件,/etc/passwd,/etc/shadow,/etc/group
\begin{Bash}
[root@192 Development]# cat /etc/passwd | head -n 4
root:x:0:0:root:/root:/bin/bash
bin:x:1:1:bin:/bin:/sbin/nologin
daemon:x:2:2:daemon:/sbin:/sbin/nologin
adm:x:3:4:adm:/var/adm:/sbin/nologin
[root@192 Development]# 
[root@192 Development]# cat /etc/shadow | head -n 4
root:$6$/5.c2ZG9GmbRe4ot$KbfkB.hduc6He0qNkWqpniGWoBCPMrymx9/qkkGcN0zy4oQgbvyFIO.mzbt1H4zR9Blo01ROWixMjJrfLL5RS1:16541:0:99999:7:::
bin:*:16231:0:99999:7:::
daemon:*:16231:0:99999:7:::
adm:*:16231:0:99999:7:::
[root@192 Development]# 
[root@192 Development]# cat /etc/group | head -n 4
root:x:0:
bin:x:1:
daemon:x:2:
sys:x:3:
\end{Bash}

%$

显示所有用户,操作/etc/passwd文件
\begin{Bash}[显示所有用户]

[root@192 Development]# cut -d: -f1 /etc/passwd

\end{Bash}

% 学一下这个命令啊

\begin{Bash}[用户组也是同样吧]

[root@192 Development]# cut -d: -f1 /etc/group

\end{Bash}

\begin{Bash}[查看用户组]
[root@192 Development]# groups
root
[root@192 Development]# groups root
root : root
[root@192 Development]# groups shifeng
shifeng : shifeng wheel
\end{Bash}

\begin{Bash}[创建用户,使用命令useradd]
[root@192 Development]# useradd huang
[root@192 Development]# ll -d /home/huang/
drwx------. 3 huang huang 87 May 31 10:27 /home/huang/
[root@192 Development]# grep huang /etc/passwd /etc/shadow /etc/group
/etc/passwd:huang:x:1001:1001::/home/huang:/bin/bash
/etc/shadow:huang:!!:16586:0:99999:7:::
/etc/group:huang:x:1001:
[root@192 Development]# 
\end{Bash}
创建一个用户,系统会给我们生成很多默认是这
\begin{itemize}
\item /etc/passwd里面创建一行和账号相关的数据,包括创建UID/GID/主文件夹
\item /etc/shadow里面加入账号密码相关数据,现在还没有密码
\item /etc/group里面加入一个和账号相同的用户组
\item /home下生成一个账号同名的目录,权限是700
\end{itemize}

\begin{Bash}[当前用户还是不能使用,需要先设置密码,使用passwd]
[root@192 Development]# passwd huang
Changing password for user huang.
New password: 
Retype new password: 
passwd: all authentication tokens updated successfully.
\end{Bash}

\begin{Bash}[一般用户只能修改自己的密码,而且还需要确认旧密码]
[root@192 Development]# su huang
[huang@192 Development]$ passwd
Changing password for user huang.
Changing password for huang.
(current) UNIX password: 
New password: 
\end{Bash}
%$

% usermod, change命令就先不管了

\begin{Bash}[创建测试文件]
[root@192 Development]# mkdir test-directory
[root@192 Development]# ll
total 4
-rwxrwxrwx. 1 root root 54 May 28 01:32 foo.txt
drwxr-xr-x. 2 root root  6 May 31 10:42 test-directory
[root@192 Development]# 
[root@192 Development]# cd test-directory/
[root@192 test-directory]# 
[root@192 test-directory]# 
[root@192 test-directory]# touch empty-file
[root@192 test-directory]# ll
total 0
-rw-r--r--. 1 root root 0 May 31 10:53 empty-file
[root@192 test-directory]# ll >directory-content
[root@192 test-directory]# ll
total 4
-rw-r--r--. 1 root root 115 May 31 10:54 directory-content
-rw-r--r--. 1 root root   0 May 31 10:53 empty-file
[root@192 test-directory]# vi execute.sh
[root@192 test-directory]# ll
total 8
-rw-r--r--. 1 root root 115 May 31 10:54 directory-content
-rw-r--r--. 1 root root   0 May 31 10:53 empty-file
-rw-r--r--. 1 root root  43 May 31 10:55 execute.sh
[root@192 test-directory]# 

\end{Bash}

有三种方式从bash来创建文件:
\begin{itemize}
\item touch命令创建空文件
\item 重定向命令行的标准输出或者出错输出
\item 通过vi等打开一个不存在的文件
\end{itemize}

这个时候创建的文件默认都是644的权限。这是因为通过umask和默认权限来控制。

\begin{Bash}
[root@192 test-directory]# umask
0022
[root@192 test-directory]# umask -S
u=rwx,g=rx,o=rx
\end{Bash}

umask表示在创建文件的时候,需要削减的权限,而默认的权限,对文件来说是666,没有执行权限;对于目录来说则是777.

这样先不管第一组(和下一节相关),则unmask是022,文件666-022是644,和上面文件符合。

\begin{Bash}[测试一下umask]
[root@192 test-directory]# umask 000
[root@192 test-directory]# umask
0000
[root@192 test-directory]# touch test
[root@192 test-directory]# ll test
-rw-rw-rw-. 1 root root 0 May 31 11:12 test
\end{Bash}

接着创建其他用户的文件
\begin{Bash}
[root@192 test-directory]# vi huang-execute.sh
[root@192 test-directory]# 
[root@192 test-directory]# chown huang:huang huang-execute.sh 
[root@192 test-directory]# ll
total 12
-rw-r--r--. 1 root  root  115 May 31 10:54 directory-content
-rw-r--r--. 1 root  root    0 May 31 10:53 empty-file
-rwxr-xr-x. 1 root  root   43 May 31 10:55 execute.sh
-rw-r--r--. 1 huang huang   0 May 31 11:35 huang-epmty-file
-rw-r--r--. 1 huang huang  37 May 31 11:36 huang-execute.sh
[root@192 test-directory]# chmod a+x huang-execute.sh 
[root@192 test-directory]# ll
total 12
-rw-r--r--. 1 root  root  115 May 31 10:54 directory-content
-rw-r--r--. 1 root  root    0 May 31 10:53 empty-file
-rwxr-xr-x. 1 root  root   43 May 31 10:55 execute.sh
-rw-r--r--. 1 huang huang   0 May 31 11:35 huang-epmty-file
-rwxr-xr-x. 1 huang huang  37 May 31 11:36 huang-execute.sh
[root@192 test-directory]# 

\end{Bash}

先测试只有r的权限的情况
\begin{Bash}[修改目录权限到r]
[root@192 test-directory]# cd ..
[root@192 Development]# chmod o=r test-directory
[root@192 Development]# ll
total 8
-rwxrwxrwx. 1 root root   54 May 28 01:32 foo.txt
drwxr-xr--. 2 root root 4096 May 31 11:36 test-directory
[root@192 Development]# 

\end{Bash}


\begin{Bash}[切换用户,执行相关操作]
[root@192 Development]# su huang
[huang@192 Development]$ 
[huang@192 Development]$ ll test-directory/
ls: cannot access test-directory/empty-file: Permission denied
ls: cannot access test-directory/directory-content: Permission denied
ls: cannot access test-directory/execute.sh: Permission denied
ls: cannot access test-directory/huang-epmty-file: Permission denied
ls: cannot access test-directory/huang-execute.sh: Permission denied
total 0
?????????? ? ? ? ?            ? directory-content
?????????? ? ? ? ?            ? empty-file
?????????? ? ? ? ?            ? execute.sh
?????????? ? ? ? ?            ? huang-epmty-file
?????????? ? ? ? ?            ? huang-execute.sh
[huang@192 Development]$ cd test-directory
bash: cd: test-directory: Permission denied
[huang@192 Development]$ 
[huang@192 Development]$ touch test-directory/file-name
touch: cannot touch ‘test-directory/file-name’: Permission denied

\end{Bash}

%$

可以读取目录结构,但是具体文件的信息没办法读取,没办法进入目录,以及创建文件。

然后测试w权限

\begin{Bash}[虽然有w权限,但是没有进入的权限,所以什么都没办法做]
[root@192 Development]# chmod o=w test-directory
[root@192 Development]# 
[root@192 Development]# 
[root@192 Development]# 
[root@192 Development]# ll
total 8
-rwxrwxrwx. 1 root root   54 May 28 01:32 foo.txt
drwxr-x-w-. 2 root root 4096 May 31 11:36 test-directory
[root@192 Development]# 
[root@192 Development]# su huang
[huang@192 Development]$ ll test-directory
ls: cannot open directory test-directory: Permission denied
[huang@192 Development]$ cd test-directory
bash: cd: test-directory: Permission denied
[huang@192 Development]$ 
[huang@192 Development]$ touch test-directory/file-name
touch: cannot touch ‘test-directory/file-name’: Permission denied

\end{Bash}

所以和我开始假想的还不太一样,我以为在外面操作,不切换城工作目录就可以使用,看来是错误的。所以感觉说没有x就只是不能切换成工作目录应该是不对的吧。应该是就没办法进入操作了。


然后测试x权限
\begin{Bash}[虽然可以进入目录,其他操作做不了,但是如果知道文件名,而且自己有权限的话,是可以访问的]
-rwxrwxrwx. 1 root root   54 May 28 01:32 foo.txt
drwxr-x--x. 2 root root 4096 May 31 11:36 test-directory
[root@192 Development]# su huang
[huang@192 Development]$ 
[huang@192 Development]$ ll test-directory
ls: cannot open directory test-directory: Permission denied
[huang@192 Development]$ cd test-directory/
[huang@192 test-directory]$ ll
ls: cannot open directory .: Permission denied
[huang@192 test-directory]$ touch test-file
touch: cannot touch ‘test-file’: Permission denied
[huang@192 test-directory]$
[huang@192 Development]$ cat ./test-directory/huang-empty-file
cat: ./test-directory/huang-empty-file: No such file or directory
[huang@192 Development]$ cat ./test-directory/huang-epmty-file 
[huang@192 Development]$ 
\end{Bash}

\begin{Bash}[w和x,可以创建文件,但是没办法看到自己创建的文件]
[root@192 Development]# su huang
[huang@192 Development]$ ll
total 8
-rwxrwxrwx. 1 root root   54 May 28 01:32 foo.txt
drwxr-x-wx. 2 root root 4096 May 31 11:55 test-directory
[huang@192 Development]$ 
[huang@192 Development]$ ls test-directory/
ls: cannot open directory test-directory/: Permission denied
[huang@192 Development]$ 
[huang@192 Development]$ touch test-directory/test-file
[huang@192 Development]$ 

\end{Bash}

%$

\subsection{特殊文件属性}

%\subsubsection{文件隐藏属性 chattr, lsattr}
%通过命令添加一些特殊属性,先不看了。

\subsubsection{文件特殊权限,SUID,SGID,SBIT}

\paragraph{setUID}

对二进制执文件有效,需要x权限,仅在执行改程序时有效,执行者具有所有者的权限。

\paragraph{setGID}

对二进制执文件有效,需要x权限,仅在执行改程序时有效,执行者具有改程序用户组的支持。

\paragraph{Sticky Bit}

对目录有效,需要w,x权限(即具有写入的权限),在该目录下创建的文件或者目录,只有root和用户自己才能删除(这里对应上面说的有w,x权限是,可以删除owner不是自己的文件)。


\subsubsection{进一步解释这个执行时的权限}

\subsection{对文件操作的说明}

和进程相关的ID有

\begin{table}
\begin{tabular}{|l|l|}
\hline
实际用户ID & 我们实际上是谁\\
实际用户组 & \\
\hline
有效用户ID & 用于文件访问权限检查\\
有效组ID & \\
附加组ID & \\
\hline
保存的设置用户ID & 由exec函数保存\\
保存的设置组ID & \\
\hline
\end{tabular}
\end{table}


\begin{itemize}
\item 实际用户ID和实际组ID就是是以谁来登陆的。
\item 有效用户ID和有效组ID及附加组ID决定我们的文件访问权限。
\item 保存的设置用户ID和保存的设置组ID在执行一个程序时包含了有效用户ID和有效组ID的副本。
\end{itemize}

当对一个文件进行打开,创建或者删除操作时:
\begin{enumerate}
\item 若进程的有效用户ID是0(超级用户),则允许访问;
\item 若进程的有效用户ID等于文件的所有者ID,所有者恰当的权限被设置,则允许访问,否则拒绝访问;
\item 若进程的有效组ID或者附加组ID等于文件的有效组ID,若组恰当的权限被设置,则允许访问,否则拒绝访问;
\item 若其他用户恰当的权限被设置,则允许访问,否则拒绝访问。
\end{enumerate}

创建文件,新文件的组ID将是:
\begin{itemize}
\item ...
\item ...
\end{itemize}

\subsection{(...)}

\subsubsection{命令组}

命令组,会启动一个子shell(subsehll).

\subsubsection{数组初始化}





