\chapter{基础内容}

\begin{CSharp}
    public class Person
    {
        public string Name { get; set; }
        public int Age { get; set; }

        List<Person> friends = new List<Person>();
        public List<Person> Friends { get; }

        Location home = new Location();
        public Location Home { get; }

        public Person(string name)
        {
            Name = name;
        }

        public Person() { }
    }

    public class Location
    {
        public string Country { get; set; }
        public string Town { get; set; }
    }
\end{CSharp}

\begin{CSharp}[使用属性]
    class UserAutoProperties
    {
        public static void CreatePerson()
        {
            Person tom = new Person("Tom");

            Person jack = new Person { Name = "Jack", Age = 9 };

            Person david = new Person("David") { Age = 10 };

            Person jin = new Person
            {
                Name = "jim",
                Age = 3,
                Home = {
                    Country = "China",
                    Town = "Zhuzhou"
                }
            };

            Person snow = new Person
            {
                Name = "Snow",
                Age = 1,
                Home = { Country = "China", Town = "Xiangtan" },
                Friends = {
                    new Person("Titi"){Age = 12},
                    new Person{Name = "Jim", Age=11},
                    new Person{Name = "Kate", Age=5,
                        Home = {Country = "China", Town = "Xiangtan"}
                    }
                }
            };
        }
\end{CSharp}

想自定义可以使用集合初始化的类的话,实现IEnumerable接口和提供公共的Add方法。
\begin{CSharp}[自定义集合初始化类]
class SampleCollection:IEnumerable
    {
        List<Tuple<string, int, string>> person = new List<Tuple<string, int, string>>();

        public void Add(string name, int age, string country)
        {
            person.Add(Tuple.Create(name, age, country));
        }


        public IEnumerator GetEnumerator()
        {
            return person.GetEnumerator();
        }

        public static void CeateSameple()
        {
            var sample = new SampleCollection{
                {"Snow", 1, "China"},
                {"Davide", 5, "China"},
                {"Jim", 2, "China"}
            };
        }
    }
\end{CSharp}

\begin{CSharp}[匿名类]
            var family = new[]{
                new {Name = "Feng", Age=35},
                new {Name="Hui",Age=36},
                new {Name="Nuo", Age=4}
            };

            int totalAge = 0;
            foreach (var p in family)
            {
                totalAge += p.Age;
            }
\end{CSharp}

\chapter{C\# 委托 delegate}



\section{委托,匿名函数和Lambda的语法}


\begin{CSharp}[委托,匿名函数和Lambda]
        [TestMethod]
        public void TestMethod4()
        {
            List<int> list = new List<int>(new int[] { 1, 2, 3 });
            List<int> expectResult = new List<int>(new int[] { 1, 4, 9 });

            List<int> ret1 = list.ConvertAll(new Converter<int, int>(Sqrt));
            List<int> ret2 = list.ConvertAll(delegate(int x) { return x * x; });
            List<int> ret3 = list.ConvertAll((int x) => { return x * x; });
            List<int> ret4 = list.ConvertAll((int x) => x * x);
            List<int> ret5 = list.ConvertAll<int>((x) => x * x);
            List<int> ret6 = list.ConvertAll<int>(x => x * x);

            CollectionAssert.AreEqual(expectResult, ret1);
            CollectionAssert.AreEqual(expectResult, ret2);
            CollectionAssert.AreEqual(expectResult, ret3);
            CollectionAssert.AreEqual(expectResult, ret4);
            CollectionAssert.AreEqual(expectResult, ret5);
            CollectionAssert.AreEqual(expectResult, ret6);
        }

        public static int Sqrt(int x)
        {
            return x * x;
        }
\end{CSharp}

从7~12行是从委托到Lambda表达式的所有语法形式,从上往下,越来越简单。11,12行的形式需要显式写明泛型类型来帮助Lambda表达式推断变量的类型。


第7行,C\#1中的delegate。

第8行,匿名函数。

第9行,Lambda表达式,花括号中可以有多个语句。

第10行,Lambda表达式,省略花括号,直接返回expression的值。

第11行,参数类型可以省略,如果编译器能够推断出此处的类型的话。

第12行,参数的括号可以省略(不知道两个参数的时候是否可以省略,等补充)。


\section{协变和逆变}

协变就是如果类型是另一个类型的子类,就能隐式的将数组,泛型,委托转换为另一个类型。

而逆变就是相反的一个过程。

当类型描述的是返回类型时,协变是安全的;当类型描述的是接受类型,逆变是安全的。

数组的协变:

\begin{CSharp}[数组的协变]
        [TestMethod]
        [ExpectedException(typeof(ArrayTypeMismatchException))]
        public void TestVariant()
        {
            string[] strs = new string[] { "a", "b", "c" };
            object[] objs = strs;
            objs[0] = new object();
        }
\end{CSharp}

问题出现了,这个问题不能被编译器检测出来。会在运行时抛出一个ArrayTypeMismatchException。那为什么数组要支持协变呢?因为Java支持,最开始为了支持从Java源码编译过来,所以.Net也支持协变数组。

泛型不支持协变。

委托是支持协变和逆变的。

\begin{CSharp}[协变和逆变]
        
        public delegate Object Dele(String str);

        [TestMethod]
        public void TestMethod1()
        {
            Dele b = new Dele(this.Delegated);

            Assert.AreEqual(typeof(String), o.GetType());
            Assert.AreEqual("This is a test string.", o.ToString());            
        }

        public String Delegated(Object obj) { return "This is a test string.";}
\end{CSharp}

上面是委托最初的语法,使用关键词delegate定义一个委托类型。在方法中生成一个委托对象,然后就可以像调用方法一直接调用委托对象。

同样,返回值是协变,输入参数是逆变。


\section{预定义委托类型}

C\#2.0引入了一个泛型委托Action<T>

\begin{CSharp}[Actio<T>的签名]
		public delegate void Action<T>(T obj);
\end{CSharp}


\begin{CSharp}[Action和匿名方法]
        [TestMethod]
        public void TestMethod2()
        {
            Action<char[]> reverseString = delegate(char[] chars){
                System.Array.Reverse(chars);
            };

            char[] test = "chars".ToCharArray();
            reverseString(test);

            Assert.AreEqual("srahc", new string(test));

        }
\end{CSharp}

需要注意的是逆变性不适应匿名方法。而且值类型不支持使用this,而引用类型支持使用this。


Action<T>没有返回值,还有另外一个有返回类型的Predicate<T>,返回bool型,常用来过滤或者匹配。签名是
\begin{CSharp}
		public delegate bool Predicate<T>(T obj);
\end{CSharp}

\section{闭包}
\begin{itemize}
\item 外部变量( outer variable)是指作用域( scope)内包括匿名方法的局部变量或参数(不包括ref和out参数)。在类的实例成员内部的匿名方法中, this引用也被认为是一个外部变量。
\item 捕获的外部变量( captured outer variable)通常简称为捕获变量( captured variable),它是在匿名方法内部使用的外部变量。
变量。
\end{itemize}

感觉和JavaScript的暂时还没有什么不同。


\#更新\#
由于JS没有块作用域,所以和C\#还是有点点不同的地方。

\begin{CSharp}[此处创建的匿名函数更新的都是一个counter,因为counter只申明了一次]
        public IList<Func<int>> createDelegate()
        {
            int counter = 5;
            IList<Func<int>> ret = new List<Func<int>>();
            for (int i = 0; i < 5; i++)
            {
                ret.Add( () => ++counter );
            }
            return ret; 
        }
\end{CSharp}

\begin{CSharp}[而此处创建的匿名函数更新的是不同的counter,因为在迭代过程中,每次创建匿名函数之前,都在各自的作用域(scope)中申明了一个变量counter]
        public IList<Func<int>> createDelegate2()
        {
            IList<Func<int>> ret = new List<Func<int>>();
            for (int i = 0; i < 5; i++)
            {
                int counter = 5;    
                ret.Add(() => ++counter );
            }
            return ret;
        }
\end{CSharp}

\begin{CSharp}[运行结果]
        [TestMethod]
        public void TestMethod3()
        {
            int counter = 5;
            IList<Func<int>> de1 = createDelegate();
            foreach (var func in de1)
            {
                int ret = func();
                Assert.AreEqual(++counter, ret);
            }

            IList<Func<int>> de2 = createDelegate2();
            foreach (var func in de2)
            {
                int ret = func();
                Assert.AreEqual(6, ret);
            }
        }
\end{CSharp}


\chapter{C\#的迭代器}

\section{实现和注意}
\begin{CSharp}[C\#1中的实现]
using System;
using System.Collections;
using System.Linq;
using System.Text;
using System.Threading.Tasks;

namespace MyFirstConsoleApp.Iterator
{
    class IterationSample: IEnumerable
    {
        object[] values;
        int startingPoint;

        public IterationSample(object[] values, int position)
        {
            this.values = values;
            this.startingPoint = position;
        }

        public IEnumerator GetEnumerator()
        {
            throw new NotImplementedException();
        }
        /* 此处生成一个新的Enumerator类,而不是用自身的原因
         * 是可能同时调用GetEnumerator方法多次,这样状态就会乱掉
         */
        class IterationSampleIterator : IEnumerator
        {
            IterationSample parent;
            int position;

            internal IterationSampleIterator(IterationSample parent)
            {
                this.parent = parent;
                this.position = -1;
            }

            public object Current
            {
                get { 
                    if(position == -1 || position == parent.values.Length)
                    {
                        throw new InvalidOperationException();
                    }
                    int index = position + parent.startingPoint;
                    index = index % parent.values.Length;
                    return parent.values[index];
                }
            }

            public bool MoveNext()
            {
                if (position != parent.values.Length)
                {
                    position++;
                }
                return position < parent.values.Length;
            }

            public void Reset()
            {
                position = -1;
            }
        }
    }


}
\end{CSharp}
\begin{CSharp}[C\#2中的实现]
        public IEnumerator GetEnumerator()
        {
           for(int index = 0; index < values.Length; index++)
           {
               yield return values[(index + startingPoint) % values.Length];
           }
        }
\end{CSharp}

下面代码会打印出各个操作之后所执行的代码。

\begin{CSharp}
        static readonly string padding = new string(' ', 30);

        public static IEnumerable<int> CreateEnumerable()
        {
            Console.WriteLine("{0}Starting of CreateEnumerable()", padding);

            for(int i = 0; i < 3; i++)
            {
                Console.WriteLine("{0}About to yield {1}", padding, i);
                yield return i;
                Console.WriteLine("{0}After yield {1}", padding, i);
            }

            Console.WriteLine("{0}Yiedling final value", padding);
            yield return -1;

            Console.WriteLine("{0}End of createEnumerable()", padding);
        }

        public static void PrintIteratorSteps()
        {
            IEnumerable<int> iterable = CreateEnumerable();
            IEnumerator<int> iterator = iterable.GetEnumerator();


            Console.WriteLine("Pre starting to iterate, Current = {0}", iterator.Current);

            Console.WriteLine("Starting to iterate");

            while(true)
            {
                Console.WriteLine("Calling MoveNext()...");
                bool result = iterator.MoveNext();
                if(!result)
                {
                    break;
                }
                Console.WriteLine("Fetching Current...");
                Console.WriteLine("...Current result = {0}", iterator.Current);
            }

            Console.WriteLine("After iterate, Current = {0}", iterator.Current);

            try
            {
                iterator.Reset();
            }
            catch (NotSupportedException e)
            {
                Console.WriteLine("Reset Method has not been implemented.");
            }
        }
\end{CSharp}

执行结果是
\begin{CSharp}
Pre starting to iterate, Current = 0
Starting to iterate
Calling MoveNext()...
                              Starting of CreateEnumerable()
                              About to yield 0
Fetching Current...
...Current result = 0
Calling MoveNext()...
                              After yield 0
                              About to yield 1
Fetching Current...
...Current result = 1
Calling MoveNext()...
                              After yield 1
                              About to yield 2
Fetching Current...
...Current result = 2
Calling MoveNext()...
                              After yield 2
                              Yiedling final value
Fetching Current...
...Current result = -1
Calling MoveNext()...
                              End of createEnumerable()
After iterate, Current = -1
Reset Method has not been implemented.
\end{CSharp}
\section{三个实例}
\subsection{1}
\subsection{2}
\subsection{3}

