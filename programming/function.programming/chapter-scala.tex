\chapter{Scala}

\section{对我来说快速入门的部分}

\subsection{变量和函数}

val定义的是常量,不能重新赋值类似于Java中的final;var定义的是可变的,可以重新赋值.

我个人倾向是在熟练使用scala之前尽量使用val.

\subsection{命名空间}

\begin{itemize}
\item Java相同的方式
\item C++,C\#的方式

这种方式我也不是太清楚,C\#部分补全
\end{itemize}

\subsection{匿名函数}

\begin{Scala}
	def sum(f:Int=>Int, a:Int, b:Int):Int = {
		if (a > b)
			0
		else
			f(a) + sum(f, a+1, b)
	}                                         //> sum: (f: Int => Int, a: Int, b: Int)Int
	
	// 使用匿名函数
	sum((x:Int)=>x, 0, 10)                    //> res0: Int = 55
	
	// 编译器自动推到类型
	sum(x=>x, 0, 10)                          //> res1: Int = 55
	
	//其实匿名函数就是
	sum({def f(x:Int) = x;f}, 0, 10)          //> res2: Int = 55

	// 使用参数的位置记法
	sum({_*1}, 0, 10)                         //> res3: Int = 55
	
	//偏应用函数
	(sum(_:Int=>Int, 0, 10)){x=>x}            //> res4: Int = 55
	
	
\end{Scala}

\subsubsection{参数的位置记法}

\subsubsection{偏应用函数}

\subsection{操作符}

===的规则需要好好查查资料,书上说的不完全.



\section{概念}

\subsection{函数式编程}

\subsubsection{函数是一等公民}

函数和其他值一样,没有什么区别,可以作为函数参数;也可以作为函数返回值;可以在函数中定义;

\subsubsection{函数是map输入到输出}

而不是在函数中改变数据.其实本质就是不可变数据结构;要求函数没有副作用


