\chapter{Function Programming}

\subsection{基本概念}

函数式编程严格意义上的定义是没有可变量,没有赋值操作,没有命令控制结构的编程。

在实际操作中,函数式编程语言就是functions are first-class-citizens.
意思:
\begin{itemize}
\item function能在任何地方定义,包括在其他function中
\item function可以作为函数的参数和返回值
\item 存在一系列操作符来构建function
\end{itemize}

\begin{Scala}
  def square(x: Double) = x * x                   //> square: (x: Double)Double
  
  def sumOfSquare(x: Double, y: Double): Double = square(x) + square(y)
                                                  //> sumOfSquare: (x: Double, y: Double)Double
  
  sumOfSquare(3, 2 + 2)                           //> res0: Double = 25.0
\end{Scala}

\subsubsection{CBV \& CBN}

\begin{Scala}[运算过程:call by value]
sumOfSquare(3, 2+2)
sumOfSquare(3, 4)
3*3 + square(4)
9 + square(4)
9 + 4*4
9 + 16
25
\end{Scala}

\begin{Scala}[运算过程:call by name]
sumOfSquare(3, 2+2)
square(3) + square(2+2)
3*3 + square(2+2)
9 + square(2+2)
9 + (2+2)*(2+2)
9 + 4*4
25
\end{Scala}

最终计算结果相同,如果满足下面的条件:
\begin{itemize}
\item 被计算的表达式是纯函数,
\item 并且,表达式可以完成计算。
\end{itemize}

优势:
\begin{itemize}
\item call by value(CBV)的优势是所有的参数都只计算一次;
\item call by name(CBN)的有优势是对没有用的参数不进行计算。
\end{itemize}

表达式可以被终结
\begin{itemize}
\item CBV如果可以被终结,CBN也可以被终结
\item CBN可以被终结,CBV不一定可以被终结
\end{itemize}

\begin{Scala}
  def loop:Int = loop                             //> loop: => Int
  def first(x:Int, y:Int) = x                     //> first: (x: Int, y: Int)Int
  // Scala默认使用CBV,如果要强制使用CBN,使用=>来修饰参数
  def second(x: => Int, y:Int) = y                //> second: (x: => Int, y: Int)Int
 
  second(loop, 2)                                 //> res1: Int = 2
\end{Scala}

\subsubsection{if - else Expression}

\begin{Scala}[和Java相似,但是这里是表达式,不是Java中的statement]
  def abs(x: Int) = if (x >= 0) x else -x         //> abs: (x: Int)Int
\end{Scala}

\begin{Scala}[使用if实现and和or]
  def and(x: Boolean, y: Boolean) = if (x) y else false
                                                  //> and: (x: Boolean, y: Boolean)Boolean
  def or(x: Boolean, y: Boolean) = if (x) true else y
                                                  //> or: (x: Boolean, y: Boolean)Boolean

  and(true, false)                                //> res2: Boolean = false
  and(true, true)                                 //> res3: Boolean = true
  and(false, true)                                //> res4: Boolean = false
  and(false, false)                               //> res5: Boolean = false

  or(true, false)                                 //> res6: Boolean = true
  or(false, true)                                 //> res7: Boolean = true
  or(true, true)                                  //> res8: Boolean = true
  or(false, false)                                //> res9: Boolean = false
\end{Scala}

\subsubsection{Name Definition \& Value Definition}
\begin{Scala}[名字和值的不同]
  //name definition & value definition
  def x = square(2)                               //> x: => Double

  val y = square(2)                               //> y  : Double = 4.0
\end{Scala}

\subsubsection{牛顿法求根}
对于x求x的根y,首先预估一个y,这个数只需要是正数,然后通过不断改进这个值来找到合适的y(误差在要求范围之内的y),改进的公式是$y_{n+1} = (y_n + x/y)/2$
\begin{Scala}[没太弄懂为什么需要除以x来解决问题]
  def abs(x: Double): Double = if (x < 0) -x else x
                                                  //> abs: (x: Double)Double

  def sqrtIter(guess: Double, x: Double): Double =
    if (isGoodEnough(guess, x)) guess
    else sqrtIter(improve(guess, x), x)           //> sqrtIter: (guess: Double, x: Double)Double

  def isGoodEnough(guess: Double, x: Double): Boolean =
    abs(guess * guess - x) / x < 0.01             //> isGoodEnough: (guess: Double, x: Double)Boolean

  def improve(guess: Double, x: Double): Double =
    (guess + x / guess) / 2                       //> improve: (guess: Double, x: Double)Double

  def sqrt(x: Double): Double = sqrtIter(1.0, x)  //> sqrt: (x: Double)Double
\end{Scala}
注:对于递归调用的function,Scala必须指定返回的类型,而其他function则不是必须的,因为编译器无法判断递归调用的返回类型是什么。

\subsubsection{Block}
Block: {...}
\begin{itemize}
\item 包含一系列的definition和expression
\item 最后的一个元素是一个expression,定义了block的值
\item block可以出现在任何expression出现的地方
\end{itemize}

\begin{Scala}[嵌套]
  def sqrt(x: Double): Double = {

    def abs(x: Double): Double = if (x < 0) -x else x

    def sqrtIter(guess: Double, x: Double): Double =
      if (isGoodEnough(guess, x)) guess
      else sqrtIter(improve(guess, x), x)

    def isGoodEnough(guess: Double, x: Double): Boolean =
      abs(guess * guess - x) / x < 0.01

    def improve(guess: Double, x: Double): Double =
      (guess + x / guess) / 2

    sqrtIter(1.0, x)
  }      
\end{Scala}


\begin{Scala}[由于嵌套作用域,可以删除定义在里面的x]
  def sqrt(x: Double): Double = {

    def abs(x: Double): Double = if (x < 0) -x else x

    def sqrtIter(guess: Double): Double =
      if (isGoodEnough(guess)) guess
      else sqrtIter(improve(guess))

    def isGoodEnough(guess: Double): Boolean =
      abs(guess * guess - x) / x < 0.01

    def improve(guess: Double): Double =
      (guess + x / guess) / 2

    sqrtIter(1.0)
  }  
\end{Scala}


\subsubsection{代码分行}
Scala的分号可以不写,也可以写,但是如果将一个表达式分行写

\begin{Scala}
someExpression
 + someExpression
\end{Scala}
会被编译器解析为:
\begin{Scala}
someExpression;
+ someExpression;
\end{Scala}

可以这么写
\begin{Scala}
(someExpression
 + someExpression)
\end{Scala}
或者这么写
\begin{Scala}
someExpression + 
	someExpression
\end{Scala}


\subsubsection{尾递归}

递归的例子:
\begin{Scala}[GCD]
def gcd(a: Int, b: Int): Int = if (b == 0) a else gcd(b, a % b)
\end{Scala}
今天才了解到,不需要a > b,a < b的时候,第一次迭代自然的交换了两者的位置。

factorial:
\begin{Scala}
  def factorial(n: Int): Int =
    if (n == 0) 1 else n * factorial(n - 1)       //> factorial: (n: Int)Int
\end{Scala}
GCD和factorial一处很大的不同,factorial最后的结果需要外面一层的context的值,这样就导致计算机的stack必须保存外一层的execution context,而gcd的话,只是对自身本身的调用,所以stack可以重用,这样就可以使用常量的memroy space。


\subsubsection{high order function}
high order function就是函数能够被当做函数参数,能够当做函数返回值。

\begin{Scala}
  def id(x: Int): Int = x                         //> id: (x: Int)Int
  def sumInts(a: Int, b: Int): Int =
    if (a > b) 0 else id(a) + sumInts(a + 1, b)   //> sumInts: (a: Int, b: Int)Int

  def cube(x: Int): Int = x * x * x               //> cube: (x: Int)Int
  def sumCubes(a: Int, b: Int): Int =
    if (a > b) 0 else cube(a) + sumCubes(a + 1, b)//> sumCubes: (a: Int, b: Int)Int

  def factorial(n: Int): Int =
    if (n == 0) 1 else n * factorial(n - 1)       //> factorial: (n: Int)Int
  def sumFactorial(a: Int, b: Int): Int =
    if (a > b) 0 else factorial(a) + sumFactorial(a + 1, b)
                                                  //> sumFactorial: (a: Int, b: Int)Int
\end{Scala}

上面的函数表示的是$\sum_{n=a}^{b}f(x)$

\begin{Scala}
  def sum(f: Int => Int, a: Int, b: Int): Int =
    if (a > b) 0 else f(a) + sum(f, a + 1, b)     //> sum: (f: Int => Int, a: Int, b: Int)Int

  def sumInts2(a: Int, b: Int): Int = sum(x => x, a, b)
                                                  //> sumInts2: (a: Int, b: Int)Int

  def sumCubes2(a: Int, b: Int): Int = sum(x => x * x * x, a, b)
                                                  //> sumCubes2: (a: Int, b: Int)Int

  def sumFactorials2(a: Int, b: Int): Int = sum(factorial, a, b)
                                                  //> sumFactorials2: (a: Int, b: Int)Int
\end{Scala}

然后是
\begin{Scala}
  def sum2(f: Int => Int): (Int, Int) => Int = {
    def sumOfF(a: Int, b: Int): Int =
      if (a > b) 0 else f(a) + sumOfF(a + 1, b)
    sumOfF
  }                                               //> sum2: (f: Int => Int)(Int, Int) => Int

  def sumInt3 = sum2(x => x)                      //> sumInt3: => (Int, Int) => Int

  def sumCubes3 = sum2(x => x * x * x)            //> sumCubes3: => (Int, Int) => Int

  def sumFactorials3 = sum2(factorial)            //> sumFactorials3: => (Int, Int) => Int	
\end{Scala}

上面sum2的定义可以简写为:
\begin{Scala}
  def sum3(f: Int => Int)(a: Int, b: Int): Int =
    if (a > b) 0 else f(a) + sum3(f)(a + 1, b)    //> sum3: (f: Int => Int)(a: Int, b: Int)Int
\end{Scala}

\subsubsection{Map Reduce}
找到prodcut和sum的共通
\begin{Scala}
  def product(f: Int => Int)(a: Int, b: Int):Int =
    if (a > b) 1 else f(a) * product(f)(a + 1, b) //> product: (f: Int => Int)(a: Int, b: Int)Int
    
  def sum(f:Int=>Int)(a:Int, b:Int):Int =
  	if (a> b) 0 else f(a) + sum(f)(a+1, b)    //> sum: (f: Int => Int)(a: Int, b: Int)Int
 
 def reduce(f:Int=>Int, combine:(Int, Int)=>Int, zero:Int)(a:Int, b:Int):Int =
 	if (a>b) zero
 	else combine(f(a),reduce(f, combine, zero)(a+1, b))
                                                  //> reduce: (f: Int => Int, combine: (Int, Int) => Int, zero: Int)(a: Int, b: In
                                                  //| t)Int
  product(x=>x*x)(4,6)                            //> res0: Int = 14400
  
  reduce(x=>x*x, (x,y)=>x*y, 1)(4,6)              //> res1: Int = 14400
\end{Scala}


\section{Data Struct}

\subsubsection{class}
