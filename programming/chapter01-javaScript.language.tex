\renewcommand\lstlistingname{Mocha}  
\chapter{JavaScript 语言}

JavaScript语言有一些重要的概念,原型,函数,作用域和闭包.掌握了这些概念,基本就掌握了JavaScript这门语言,其他的就是一些语言细节特性.

我们所说的面向对象.继承,封装和多态.从语言上来说,有两种实现面向对象的方式:基于类和基于原型的继承.

基于类的继承. 类是模板,继承的是行为.这类语言有Java.
	
基于原型的继承, 所有的都是对象(当然还有基本类型),继承的是状态和行为.这类语言有JavaScript

如果是从Java这种基于类的语言转过来学些JavaScript的,要有这样一个转变,在JavaScript的世界中,只有对象这个概念.当然还有基本类型,但是在需要对象的时候,会自动将基本类型转换为对象的. 在这里,没有类。new Date(),这里的Date不是类,也是一个对象。

先来看看简单的代码。
	
\begin{JavaScript}
var a = {};	
a.b = 10;
console.log(a.b); // 10
console.log(a["b"]); // 10
a.foo = function() {
	console.log("invoke foo");
};
a.foo();
\end{JavaScript}

JavasScript中对象的定义是: Object是Property的集合,property是一个值或者对象的引用. JavaScript对象是一组属性的集合,这些属性引用的是一个对象或者基本类型.
	
首先,第一行生成一个对象,并将它赋值给a。然后第二行,a.b = 10,这个时候,因为a没有b这个property,所以给他生成了一个b property,并将10复制给他,第三行,打印a.b,结果是10.第四行,对对象属性的访问有两种方式,.和[]。不同之处是.后面是直接跟标识符,而[]中是字符串,注意第四行的b是有引号的。
	
在在后面,将我们定义的函数赋值给foo property,这里又不同于Java,我们前面说了,JavaScript中一切皆对象,所以函数也是对象,他是一类比较特殊的对象,可以通过被调用。因为函数是对象,所以他也可以赋值给对象的属性。赋值之后,我们可以调用他。
	
\section{JavaScript的类型}

\subsection{基本类型}

JavaScript只有对象这种说法也不是太准确,JavaScript有基本类型和对象两类。只是在使用基本类型的时候,如果必要,基本类型会被自动转换为对象类型。

基本类型有Undefined, Null, Boolean, String和Number类型

\begin{tabular}{|r|l|}
\hline
类型 & 说明 \\
\hline
Undefined & 有且只有一个值,undefined。\\
\hline
Null & 有唯一的值null \\
\hline
Boolean & 两个值,true和false \\
\hline
String & 如 "abc" \\
\hline
Number & 只有浮点型表示。 NaN, Infinity, -Infinity \\
\hline
\end{tabular}

JavaScript中的undefined就是这个值,只是undefined不是关键字,而是一个全局变量。可能被赋其他值。所以可以使用void表达式来表示,如void 0 
\begin{JavaScript}
var foo = undefined;
if (foo === void 0) 
{
	console.log("foo is undefined");
}
\end{JavaScript}

\subsection{对象类型}

\begin{itemize}
\item 对象就是property的集合。
\item 给对象property赋值,如果存在,就修改值,如果不存在,就创建一个新的property,并赋值。
\item 函数也是对象,可以赋值给property。
\end{itemize}
	
对于property,我们可以先将他看成有两类property,一类就是我们上面看到的,实际上,他还要细分为data property 和 access property。上面和下面所说的也就是data property。我们平时写代码生成的也是data property类型。
	
再来有一类我们叫做internal property,他是在我们程序级别是看不到的,而是提供给语言内部实现级别所使用的。例如我们将要讨论的原型,每个对象都有原型属性,但是我们在程序中无法通过.的方式来访问他,我们在谈论到internal property的时候,会使用[[]]来表示,例如,原型属性我们会表示成[[prototype]]。
	
\subsection{Prototype}
	
我们说到,js是通过原型的方式来实现继承的。原型实际上就是对象的一个internal property [[prototype]],每个对象都有[[prototype]]属性,他指向一个对象,而原型本身也有[[prototype]]属性,这样一直链接到最顶端的一个特殊对象,他的原型是空的,所有的对象的原型链的顶端都是这个对象。这就是我们说的原型链。

在浏览器中,我们可以通过属性名\lstinline!__proto__!来引用内部属性[[prototype]]。
	
然后我们说说原型的作用,他就是帮我们来解析.和[]取得什么样的值的。当我们通过.的方式来读取一个对象的property的时候,会在当前对象中查找看看是否有这个property,如果有,就返回,如果没有,就会尝试着在他的[[prototype]]上来查找,如果找到了就返回,如果没有找到,就继续在这个链往上找,直到找到最顶,如果还没有找到,那就会返回undefined。
	
	
\begin{JavaScript}
var proto = {bar : 10};
var foo = Object.create(proto);

console.log(foo.bar); // 10
console.log(foo.bar2); // undefined

proto.bar2 = 20; 
console.log(foo.bar2); // 20
\end{JavaScript}
	
这里的Object.create函数是用来创建一个对象,他的原型是传入的参数。在第四行,因为在原型对象上定义了bar,所以可以取到值10,而bar2没有定义,所以得不到值,返回undefined。然后在第七行,我们给proto的bar2赋值20,这样当我们再次调用foo.bar2的时候,在原型链上查找,可以找到值20了。
	
	
这里说到的是取值,实现了继承。接着是赋值,通过下面赋值的方式,实现了类似于Java中的多态,这里说类似于,是因为我认为js中没有类,依旧不存在相同类型不同行为,而是动态语言所说的鸭子类型。赋值很简单,如果存在就修改值,不存在就新增这个属性,这里所说的存在与不存在是指当前对象,而不是只原型链上存不存在。
	
\begin{JavaScript}
var proto = {x: 10};

var foo = Object.create(proto);
var bar = Object.create(proto);

console.log(foo.x); // 10
console.log(bar.x); // 10

foo.x = 20;

console.log(foo.x); // 20
console.log(bar.x); // 10
console.log(proto.x); // 10
\end{JavaScript}
		
这里可以看到foo和bar都是继承proto,所以x都等于10,而在foo.x = 20之后,foo上创建了x,并且赋值20。所以当取foo.x是等于20。而bar.x和proto.x还是10。

\subsection{Property}

\subsubsection{Data Property 和 Access Property}

从逻辑上来讲,Object是一系列property的集合,其中property可能是data property或者access property。
\begin{itemize}
\item data property就是一个key和一个ECMAScript语言类型的值和一系列boolean型的Attribute组成。
\item access property就是一个key和一个或者两个accessor function和一系列boolean型的Attribute组成。
\end{itemize}	
这里的key,要么是一个string,要么是一个symbol(ECMAScript6)。

ECMAScript的使用key来标示Property,有两中方式来访问Property。get和set。

\subsubsection{Attribute}
Attribute则是规范用来定义和解释Property的。

Data Property的Attribute

\begin{tabular}{|l|l|l|}
\hline
属性名 & 属性值 & 解释 \\
\hline
\lstinline![[Value]]! & 任意ECMAScript语言类型 & 属性的值 \\
\hline
\lstinline![[Writable]]! & Boolean & 如果是false,则通过[[Set]]来设定[[Value]]不会成功。 \\
\hline
\lstinline![[Enumerable]]! & Boolean & 如果是true,则可以在for-in中被迭代。 \\
\hline
\lstinline![[Configurable]]! & Boolean & \parbox[t]{8cm}{如果是false,则删除,从Data转换成accessor Property,修改除\lstinline![[Value]]! 之外的值都会失败}\\
\hline
\end{tabular}

\begin{JavaScript}[Data Property中Attribute的默认值]
		it(' should return correct default descriptor', function(){
			var foo  = Object.create({}, {
				bar: {}
			});
			var descriptor = Object.getOwnPropertyDescriptor(foo, 'bar');
			descriptor.should.eql(
				{
					value: undefined, 
					writable: false, 
					enumerable: false, 
					configurable: false
				});
		});
\end{JavaScript}

\begin{JavaScript}[writable为false时,不能修改Value]
		it(' should not able to change property value when writable is false', function(){
			var foo = Object.create({}, {
				bar: {value: 100, writable: false}
			})

			foo.bar = 200;
			foo.bar.should.be.exactly(100).and.be.Number;
		});
\end{JavaScript}

\begin{JavaScript}[enumerable为false时,key不会再for in语句中出现]
		it(' should not be able to enumerate in for in statement when enumerable is false', function(){
			var foo = Object.create({}, {
				bar: {value: 100, enumerable: false}
			});

			for (var key in foo)
			{
				key.should.not.be.exactly('bar');
			}
		});
\end{JavaScript}

\begin{JavaScript}[configurable为false时,不能修改除Value之外的Attributes]
		it(' should not be able to modify attributes of property excluded [[Value]] when configurable is false', function(){
			var foo = Object.create({}, {
				bar: {value: 100, writable: true, enumerable: false, configurable: false}
			});

			var descriptor = {value: 200, writable: false, enumerable: true, configurable: true};
			(function(){
				Object.defineProperty(foo, 'bar', descriptor);
			}).should.throw();

			foo.bar = 200;
			foo.bar.should.be.exactly(200).and.be.Number;			
		});
\end{JavaScript}

\begin{tabular}{|l|l|l|}
\hline
属性名 & 属性值 & 解释 \\
\hline
\lstinline![[Get]]! & & \\
\hline
\lstinline![[Set]]! & & \\
\hline
\lstinline![[Enumerable]]! & & \\
\hline
\lstinline![[Configurable]]! & Boolean & \\
\hline
\end{tabular}

\begin{JavaScript}[Accessor Property的Attributs的默认值]
		it('should get correct default descriptor', function(){
			var foo = Object.create({}, {
				bar : {get: undefined}
			});
			var descriptor = Object.getOwnPropertyDescriptor(foo, 'bar');
			descriptor.should.eql({
				get: undefined,
				set: undefined,
				enumerable: false,
				configurable: false
			});
		});
\end{JavaScript}

\begin{JavaScript}[对于Accessor Property,接受请求的对象会被当做this]
		it('base should be treat as this when assign or get property value', function(){
			var foo = Object.create({}, {
				bar: {get: function(){
					return this._bar;
				}, 
				set: function(value){
					this._bar = value;
				}}
			});
			foo.bar = 100;
			foo.should.have.ownProperty('_bar').be.exactly(100);
			foo.should.have.ownProperty('bar').be.exactly(100);
		});
\end{JavaScript}

\subsubsection{对Property进行赋值操作}
这里稍微比前面说如果对象上不存在某个被赋值的属性,就创建会要复杂点。对于对象上已经存在这个Property的情况,赋值的赋值,调用set方法的调用set方法。

而对于Property在对象上不存在的,可以分三种情况处理:
\begin{itemize}
\item 如果这个Property在原型链上不存在的话,则会创建一个Data Property
\begin{JavaScript}[Property在原型链上也不存在的话]
		it('should create a own data property when assignning a not exist property', function() {
			foo = Object.create({}, {});
			foo.bar = 100;
			foo.should.have.ownProperty('bar').be.exactly(100);
			var descriptor = Object.getOwnPropertyDescriptor(foo, 'bar');
			descriptor.should.eql({
				value: 100,
				writable: true,
				enumerable: true,
				configurable: true
			})
		});
\end{JavaScript}
\item 如果这个Property在原型链上存在一个Accessor Property的话,直接调用这个Accessor Property的\lstinline![[Set]]!方法,reciver会被当做this来传给这个方法。
\begin{JavaScript}[原型链上存在Accessor Property的情况]
		it('should not create own propert when assignning a inherited accessor property', function(){
			var x = 100
			var parent= {};
			Object.defineProperty(parent, 'bar', {set: function(value){x = value}});

			foo = Object.create(parent, {});

			foo.should.not.have.ownProperty('bar');
			foo.bar = 200;
			foo.should.not.have.ownProperty('bar');
			x.should.be.exactly(200);
		});
\end{JavaScript}

\item 如果原型链上存在一个Data Property的话,还是直接在当前对象上创建一个Data Property.
\begin{JavaScript}[原型链上存在Data Property的情况]
		it('should create a own data property when assignning a inherited data property', function(){
			foo = Object.create({bar: 200}, {});

			foo.should.not.have.ownProperty('bar');
			foo.bar = 100;
			foo.should.have.ownProperty('bar').be.exactly(100);
		});
\end{JavaScript}
\end{itemize}
	
\subsection{== 和 ===}

首先要了解ToNumber

\begin{tabular}{|l|l|}
\hline
参数 & 值 \\
\hline
Undefined & NaN \\
Null & 0 \\
Boolean & true转换为1,false转换为0 \\
String & 将string转换为数字 \\
Object & 先转换为基本类型,然后再按上面的转换
\end{tabular}

对象转换为基本类型一般是看valueOf是不是返回基本类型,不是再看看啊toString是不是返回基本类型,如果还不是则抛错。

\begin{itemize}
\item == 

\begin{enumerate}
\item 对左右引用类型求值,设左边是lval,右边是rval;
\item 如果lval和rval是相同类型的话。
	\begin{itemize}
	\itme 基本类型的话,按正常的相等性进行比较.
	\item NaN不等于其他值.
	\item 不是基本类型时,如果不是引用同一个对象,则两者不相等。
	\end{itemize}
\item 当lval和rval是不同类型的时候
	\begin{itemize}
	\item undefined等于null。
	\item undefined和null不等于其他值
	\item 如果有boolean型,则将其转换为number,再比较
	\item 如果有一方是对象,一方是数字或者字符串,将对象转换为基本类型再比较
	\item 如果是数字和字符串比较,转换为数字在比较
	\end{itemize}
\end{enumerate}

\item
\end{itemize}

\subsection{typeof 和 instanceof}
typeof就是返回后面所跟引用的类型,这个值是固定的。

\begin{tabular}{|r|l|}
		\hline
		类型	& 	结果\\
		\hline
		Undefined 	& 	"undefined"\\
		\hline
		Null		&	"object"\\
		\hline
		Boolean		&	"boolean"\\
		\hline
		Number		&	"number"\\
		\hline
		String		&	"string"\\
		\hline
		Host object & Implementation-dependent\\
		\hline
		Function object & "function"\\
		\hline
		Any other object &	"object"\\
		\hline
\end{tabular}

	instanceof 则是检查第一个对象的原型链上是否包含第二个构造函数的prototype属性。

	
\subsection{ECMA spec中和本节相关概念}

\section{函数}
函数在JavaScript中是第一等公民。

	
\subsection{arguments}
\subsection{函数作为构造函数}	
	
\section{作用域}
	函数中的变量的定义和取值和函数的作用域有关。
\section{闭包}
	函数和作用域,就构成了闭包。
\section{语言细节}
	这部分可以讨论一下很多关于JavaScript的细节.