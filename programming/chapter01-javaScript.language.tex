\chapter{JavaScript 语言}
	JavaScript语言有一些重要的概念,原型,函数,作用域和闭包.掌握了这些概念,基本就掌握了JavaScript这门语言,其他的就是一些语言细节特性.
	\section{原型}
	我们所说的面向对象.继承,封装和多态.从语言上来说,有两种实现面向对象的方式:基于类和基于原型的继承.

	基于类的继承. 类是模板,继承的是行为.这类语言有Java.
	
	基于原型的继承, 所有的都是对象(当然还有基本类型),继承的是状态和行为.这类语言有JavaScript

	如果是从Java这种基于类的语言转过来学些JavaScript的,要有这样一个转变,在JavaScript的世界中,只有对象这个概念.当然还有基本类型,但是在需要对象的时候,会自动将基本类型转换为对象的. 在这里,没有类。new Date(),这里的Date不是类,也是一个对象。
	
	
	\subsection{Property}

	先来看看简单的代码。
	
	\lstinputlisting[
		style=js,
		linerange={1-8},
		firstnumber=1
	]{code/javascript/prototype.js}

	JavasScript中对象的定义是: Object是Property的集合,property是一个值或者对象的引用. JavaScript对象是一组属性的集合,这些属性引用的是一个对象或者基本类型.
	
	首先,第一行生成一个对象,并将它赋值给a。然后第二行,a.b = 10,这个时候,因为a没有b这个property,所以给他生成了一个b property,并将10复制给他,第三行,打印a.b,结果是10.第四行,对对象属性的访问有两种方式,.和[]。不同之处是.后面是直接跟标识符,而[]中是字符串,注意第四行的b是有引号的。
	
	在在后面,将我们定义的函数赋值给foo property,这里又不同于Java,我们前面说了,JavaScript中一切皆对象,所以函数也是对象,他是一类比较特殊的对象,可以通过被调用。因为函数是对象,所以他也可以赋值给对象的属性。赋值之后,我们可以调用他。
	\begin{itemize}
	\item 对象就是property的集合,在本例中,这个集合中有b这个property。
	\item 给对象property赋值,如果存在,就修改值,如果不存在,就创建一个新的property,并赋值。
	\item 函数也是对象,可以赋值给property。
	\end{itemize}
	
	对于property,我们可以先将他看成有两类property,一类就是我们上面看到的,实际上,他还要细分为data property 和 access property,这个后面再讨论,再来有一类我们叫做internal property,他是在我们程序级别是看不到的,而是提供给语言内部实现级别所使用的。例如我们将要讨论的原型,每个对象都有原型属性,但是我们在程序中无法通过.的方式来访问他,我们在谈论到internal property的时候,会使用[[]]来表示,例如,原型属性我们会表示成[[prototype]]。
	
	\subsection{Prototype}
	
	我们说到,js是通过原型的方式来实现继承的。原型实际上就是对象的一个internal property [[prototype]],每个对象都有[[prototype]]属性,他指向一个对象,而原型本身也有[[prototype]]属性,这样一直链接到最顶端的一个特殊对象,他的原型是空的,所有的对象的原型链的顶端都是这个对象。这就是我们说的原型链
	
	然后我们说说原型的作用,他就是帮我们来解析.和[]取得什么样的值的。当我们通过.的方式来读取一个对象的property的时候,会在当前对象中查找看看是否有这个property,如果有,就返回,如果没有,就会尝试着在他的[[prototype]]上来查找,如果找到了就返回,如果没有找到,就继续在这个链往上找,直到找到最顶,如果还没有找到,那就会返回undefined。
	
	
	\lstinputlisting[
		style=js,
		linerange={11-18},
		firstnumber=1
	]{code/javascript/prototype.js}
	
	这里的Object.create函数是用来创建一个对象,他的原型是传入的参数。在第四行,因为在原型对象上定义了bar,所以可以取到值10,而bar2没有定义,所以得不到值,返回undefined。然后在第七行,我们给proto的bar2赋值20,这样当我们再次调用foo.bar2的时候,在原型链上查找,可以找到值20了。
	
	
	这里说到的是取值,实现了继承。接着是赋值,通过下面赋值的方式,实现了类似于Java中的多态,这里说类似于,是因为我认为js中没有类,依旧不存在相同类型不同行为,而是动态语言所说的鸭子类型。赋值很简单,如果存在就修改值,不错在就新增这个属性,这里所说的存在与不存在是指当前对象,而不是只原型链上存不存在。
	
	
	\lstinputlisting[
		style=js,
		linerange={22-34},
		firstnumber=1
	]{code/javascript/prototype.js}	
		
	这里可以看到foo和bar都是继承proto,所以x都等于10,而在foo.x = 20之后,foo上创建了x,并且赋值20。所以当取foo.x是等于20。而bar.x和proto.x还是10。
	
	
	\subsection{typeof 和 instanceof}
	typeof就是返回后面所跟引用的类型,这个值是固定的。


	\begin{tabular}{rl}
		类型	& 	结果\\[5pt]
		Undefined 	& 	"undefined"\\
		Null		&	"object"\\
		Boolean		&	"boolean"\\
		Number		&	"number"\\
		String		&	"string"\\
		Host object & Implementation-dependent\\
		Function object & "function"\\
		Any other object &	"object"\\
	\end{tabular}

	instanceof 则是检查第一个对象的原型链上是否包含第二个构造函数的prototype属性。

	
	\subsection{ECMA spec中和本节相关概念}

	\section{函数}
	函数在JavaScript中是第一等公民。

	
	\subsection{arguments}
	\subsection{函数作为构造函数}	
	
	\section{作用域}
	函数中的变量的定义和取值和函数的作用域有关。
	\section{闭包}
	函数和作用域,就构成了闭包。
	\section{语言细节}
	这部分可以讨论一下很多关于JavaScript的细节.