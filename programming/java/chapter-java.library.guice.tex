\chapter{Guice}

\section{动机}

\subsection{直接构建}

先通过一个例子来说明一下为啥要用DI。这里是一个Pizza订购的网站服务。

\begin{Java}[服务接口定义]
public interface BillingService {

  /**
   * Attempts to charge the order to the credit card. Both successful and
   * failed transactions will be recorded.
   *
   * @return a receipt of the transaction. If the charge was successful, the
   *      receipt will be successful. Otherwise, the receipt will contain a
   *      decline note describing why the charge failed.
   */
  Receipt chargeOrder(PizzaOrder order, CreditCard creditCard);
}
\end{Java}

\begin{Java}[服务接口的具体实现]
public class RealBillingService implements BillingService {
  public Receipt chargeOrder(PizzaOrder order, CreditCard creditCard) {
    CreditCardProcessor processor = new PaypalCreditCardProcessor();
    TransactionLog transactionLog = new DatabaseTransactionLog();

    try {
      ChargeResult result = processor.charge(creditCard, order.getAmount());
      transactionLog.logChargeResult(result);

      return result.wasSuccessful()
          ? Receipt.forSuccessfulCharge(order.getAmount())
          : Receipt.forDeclinedCharge(result.getDeclineMessage());
     } catch (UnreachableException e) {
      transactionLog.logConnectException(e);
      return Receipt.forSystemFailure(e.getMessage());
    }
  }
}
\end{Java}

这个时候整个代码是无法测试的。

\subsection{工厂}

如果我们该用Factory模式来重构上面的代码。加入一个Factory类。

\begin{Java}[CreditCardProcessor的工厂]
public class CreditCardProcessorFactory {

  private static CreditCardProcessor instance;

  public static void setInstance(CreditCardProcessor processor) {
    instance = processor;
  }

  public static CreditCardProcessor getInstance() {
    if (instance == null) {
      return new SquareCreditCardProcessor();
    }

    return instance;
  }
}
\end{Java}

\begin{Java}[使用Factory的服务实现]
public class RealBillingService implements BillingService {
  public Receipt chargeOrder(PizzaOrder order, CreditCard creditCard) {
    CreditCardProcessor processor = CreditCardProcessorFactory.getInstance();
    TransactionLog transactionLog = TransactionLogFactory.getInstance();

    try {
      ChargeResult result = processor.charge(creditCard, order.getAmount());
      transactionLog.logChargeResult(result);

      return result.wasSuccessful()
          ? Receipt.forSuccessfulCharge(order.getAmount())
          : Receipt.forDeclinedCharge(result.getDeclineMessage());
     } catch (UnreachableException e) {
      transactionLog.logConnectException(e);
      return Receipt.forSystemFailure(e.getMessage());
    }
  }
}
\end{Java}

此时,我们能够添加相应的测试代码了。

\begin{Java}[单元测试]
public class RealBillingServiceTest extends TestCase {

  private final PizzaOrder order = new PizzaOrder(100);
  private final CreditCard creditCard = new CreditCard("1234", 11, 2010);

  private final InMemoryTransactionLog transactionLog = new InMemoryTransactionLog();
  private final FakeCreditCardProcessor processor = new FakeCreditCardProcessor();

  @Override public void setUp() {
    TransactionLogFactory.setInstance(transactionLog);
    CreditCardProcessorFactory.setInstance(processor);
  }

  @Override public void tearDown() {
    TransactionLogFactory.setInstance(null);
    CreditCardProcessorFactory.setInstance(null);
  }

  public void testSuccessfulCharge() {
    RealBillingService billingService = new RealBillingService();
    Receipt receipt = billingService.chargeOrder(order, creditCard);

    assertTrue(receipt.hasSuccessfulCharge());
    assertEquals(100, receipt.getAmountOfCharge());
    assertEquals(creditCard, processor.getCardOfOnlyCharge());
    assertEquals(100, processor.getAmountOfOnlyCharge());
    assertTrue(transactionLog.wasSuccessLogged());
  }
}
\end{Java}

同样,这个代码其实是很有问题的,我们必须特别小心的setting和tearing down。如果teardown失败了。全局变量会保留在这个地方设定的值。而且使用全局变量的没有办法多测试并行运行。

But the biggest problem is that the dependencies are hidden in the code. If we add a dependency on a CreditCardFraudTracker, we have to re-run the tests to find out which ones will break. Should we forget to initialize a factory for a production service, we don't find out until a charge is attempted. As the application grows, babysitting factories becomes a growing drain on productivity.


\subsection{依赖注入}

\begin{Java}[服务的具体实现不再负责对象的依赖]
public class RealBillingService implements BillingService {
  private final CreditCardProcessor processor;
  private final TransactionLog transactionLog;

  public RealBillingService(CreditCardProcessor processor, 
      TransactionLog transactionLog) {
    this.processor = processor;
    this.transactionLog = transactionLog;
  }

  public Receipt chargeOrder(PizzaOrder order, CreditCard creditCard) {
    try {
      ChargeResult result = processor.charge(creditCard, order.getAmount());
      transactionLog.logChargeResult(result);

      return result.wasSuccessful()
          ? Receipt.forSuccessfulCharge(order.getAmount())
          : Receipt.forDeclinedCharge(result.getDeclineMessage());
     } catch (UnreachableException e) {
      transactionLog.logConnectException(e);
      return Receipt.forSystemFailure(e.getMessage());
    }
  }
}
\end{Java}

\begin{Java}[我们不再需要工厂了]
public class RealBillingServiceTest extends TestCase {

  private final PizzaOrder order = new PizzaOrder(100);
  private final CreditCard creditCard = new CreditCard("1234", 11, 2010);

  private final InMemoryTransactionLog transactionLog = new InMemoryTransactionLog();
  private final FakeCreditCardProcessor processor = new FakeCreditCardProcessor();

  public void testSuccessfulCharge() {
    RealBillingService billingService
        = new RealBillingService(processor, transactionLog);
    Receipt receipt = billingService.chargeOrder(order, creditCard);

    assertTrue(receipt.hasSuccessfulCharge());
    assertEquals(100, receipt.getAmountOfCharge());
    assertEquals(creditCard, processor.getCardOfOnlyCharge());
    assertEquals(100, processor.getAmountOfOnlyCharge());
    assertTrue(transactionLog.wasSuccessLogged());
  }
}
\end{Java}

\begin{Java}[问题出现了,我们需要在顶级代码中处理依赖]
  public static void main(String[] args) {
    CreditCardProcessor processor = new PaypalCreditCardProcessor();
    TransactionLog transactionLog = new DatabaseTransactionLog();
    BillingService billingService
        = new RealBillingService(processor, transactionLog);
    ...
  }
\end{Java}

\subsection{使用Guice进行依赖注入}

这个时候可以使用Guice来完成这部分的工作。

\begin{Java}[通过configure来指定如果生成对象]
public class BillingModule extends AbstractModule {
  @Override 
  protected void configure() {
    bind(TransactionLog.class).to(DatabaseTransactionLog.class);
    bind(CreditCardProcessor.class).to(PaypalCreditCardProcessor.class);
    bind(BillingService.class).to(RealBillingService.class);
  }
}
\end{Java}


\begin{Java}[然后可以这样来实现依赖注入,使用@Inject注解]
public class RealBillingService implements BillingService {
  private final CreditCardProcessor processor;
  private final TransactionLog transactionLog;

  @Inject
  public RealBillingService(CreditCardProcessor processor,
      TransactionLog transactionLog) {
    this.processor = processor;
    this.transactionLog = transactionLog;
  }

  public Receipt chargeOrder(PizzaOrder order, CreditCard creditCard) {
    try {
      ChargeResult result = processor.charge(creditCard, order.getAmount());
      transactionLog.logChargeResult(result);

      return result.wasSuccessful()
          ? Receipt.forSuccessfulCharge(order.getAmount())
          : Receipt.forDeclinedCharge(result.getDeclineMessage());
     } catch (UnreachableException e) {
      transactionLog.logConnectException(e);
      return Receipt.forSystemFailure(e.getMessage());
    }
  }
}
\end{Java}

\begin{Java}[这样在顶层只需要这样]
  public static void main(String[] args) {
    Injector injector = Guice.createInjector(new BillingModule());
    BillingService billingService = injector.getInstance(BillingService.class);
    ...
  }
\end{Java}


\section{Getting Started}

\subsubsection{object graph}
通过构造函数来接受依赖,那么在构造对象之前,需要先构造他所依赖的对象,而在构造他所依赖的对象之前,又需要将依赖对象所依赖的对象先构建。这样就需要构建一个对象图(Object Graph).

\subsubsection{使用}

\begin{Java}[使用@Inject来告诉Guice通过构造函数构造]
class BillingService {
  private final CreditCardProcessor processor;
  private final TransactionLog transactionLog;

  @Inject
  BillingService(CreditCardProcessor processor, 
      TransactionLog transactionLog) {
    this.processor = processor;
    this.transactionLog = transactionLog;
  }

  public Receipt chargeOrder(PizzaOrder order, CreditCard creditCard) {
    ...
  }
}
\end{Java}

\begin{Java}[通过AbstractModule的子类来配置binding]
public class BillingModule extends AbstractModule {
  @Override 
  protected void configure() {

     /*
      * This tells Guice that whenever it sees a dependency on a TransactionLog,
      * it should satisfy the dependency using a DatabaseTransactionLog.
      */
    bind(TransactionLog.class).to(DatabaseTransactionLog.class);

     /*
      * Similarly, this binding tells Guice that when CreditCardProcessor is used in
      * a dependency, that should be satisfied with a PaypalCreditCardProcessor.
      */
    bind(CreditCardProcessor.class).to(PaypalCreditCardProcessor.class);
  }
}
\end{Java}

\begin{Java}[获取Guice的Object-Graph Builder]
 public static void main(String[] args) {
    /*
     * Guice.createInjector() takes your Modules, and returns a new Injector
     * instance. Most applications will call this method exactly once, in their
     * main() method.
     */
    Injector injector = Guice.createInjector(new BillingModule());

    /*
     * Now that we've got the injector, we can build objects.
     */
    BillingService billingService = injector.getInstance(BillingService.class);
    ...
  }
\end{Java}

