\chapter{Spring}

\section{声明bean}


ref属性来引用其他bean

factory-method属性来通过静态工厂创建bean

spring bean默认都是单例,可以通过scope来声明一个作用域。 spring的单例是保证在spring应用上下文中只有一个bean的实例,并不能阻止你使用传统方法来创建bean实例。

使用init-method和destroy-method属性来初始化和销毁bean。(也可以实现特定的两个接口,InitializingBean和DisposableBean,这样就无需额外配置了)
如果很多bean有相同的方法,则可以在beans上配置default-init-method和default-destroy-method属性。

内部bean的注入,setter或者constructor args。内部bean仅被使用一次,而且不能被其他bean使用。


\subsection{bean注入}

\subsubsection{采用setter注入属性}
\begin{itemize}
\item property标签,使用value属性注入基本类型和string类型
\item property标签,采用ref属性来引用其他在spring中配置的bean
\item property标签,嵌入内部的bean标签。
\end{itemize}

\subsubsection{采用构造函数参数来注入属性}
采用构造函数参数来注入属性,采用constructor-arg标签。


\subsubsection{p命名空间}

采用p命名空间,这样就在bean标签中加属性来注入值。


\subsubsection{装配集合}

\begin{itemize}
\item <list>,成员可以是<ref>,<value>,<bean>,<null/>,<list> 
\item <set>

如果属性是java.util.Collection类型是,这两个配置元素在使用时几乎可以完全互换。

\item <map> 对应jva.util.Map

成员entry的属性可以是key,key-ref, value, value-ref

\item <props> 对应java.util.Properties,键值都必须是String。

成员prop的属性是key,值是标签中的文本。

\end{itemize}

\subsubsection{装配空值}

使用<null/>

\subsubsection{使用SpEL装配}

使用\#\{\}界定符

\begin{itemize}
\item 字面量
\item 引用其他Bean,一个用途就是如果一个属性配置为另一个bean的属性的时候。

\item \lstinline$#{songSelector.selectSon()?.toUpperCase()}$ 类似于swift的方式。

\item \lstinline$T(java.lang.Math).PI$取得静态变量或者调用静态方法。\lstinline$T(java.lang.Math)$取得class类

\item 运算符 \lstinline$#{counter.total + 42}$


\end{itemize}

\subsection{自动装配bean}

对应明显的装配场景,可以自动装配这种类型(autowiring)。比如整个Context只有一个某种类型的时候。在bean标签中加入autowire属性,取下面的四个值。

\begin{itemize}
\item no 不自动装配。
\item byName 与bean属性具有相同名字(或者ID)的bean自动装配到bean的对应属性。
\item byType 与bean的属性具有相同类型的其他bean自动装配到bean的对应属性
\item constructor 与bean的构造器入参具有相同类型的其他bean自动装配到bean构造器的对应入参
\item autodetect 现场时constructor自动转配,如果失败,在尝试byType的方式。
\end{itemize}

可以在beans标签上设定一个default-autowire属性,为所有的bean给定默认的配置。默认情况下,default-autowire被设定为none。


\subsection{annotation装配}

在xml中加入\lstinline$<context:annotation-config />$来开启注解配置。

\begin{itemize}
\item @Autowired,使用byType方式。

只能有一个构造函数的注解required能够被设置为true,@Autowired(required=true)

使用@Autowired标注多个构造函数式,如果都满足,Spring会挑一个入参最多的构造函数。

当有多个bean合适是,可以使用@Qualifier("name")注解来指定ID为name的bean。或者直接在xml配置中加入qualifier标签。


\item @Inject
\item @Resource
\end{itemize}



