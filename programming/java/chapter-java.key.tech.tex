\chapter{关键技术}

\section{IoC 和 DI}

IoC is a generic term meaning rather than having the application call the methods in a framework, the framework calls implementations provided by the application.

DI is a form of IoC, where implementations are passed into an object through constructors/setters/service look-ups, which the object will 'depend' on in order to behave correctly.

IoC without using DI, for example would be the Template pattern because the implementation can only be changed through sub-classing.

DI Frameworks are designed to make use of DI and can define interfaces (or Annotations in Java) to make it easy to pass in implementations.

IoC Containers are DI frameworks that can work outside of the programming language. In some you can configure which implementations to use in metadata files (e.g. XML) which are less invasive. With some you can do IoC that would normally be impossible like inject implementation at pointcuts.

\section{JSR-330}



\section{•}

\section{性能优化}

\subsection{性能术语}

\begin{itemize}
\item 等待时间 Latency
\item 吞吐量 Throughput
\item 利用率
\item 效率
\item 容量

\end{itemize}

\subsection{务实的性能分析法}

\begin{itemize}
\item 你正在测量的代码有哪些可观测的环节?
\item 如何测量那些可观测环节?
\item 这些可观测环节的目标是什么?
\item 你怎么判断性能调优是否做好了?
\end{itemize}