\chapter{Maven}

\section{Install}

\subsection{安装java和配置JAVA\_HOME}

Linux中安装java的时候,将java加入PATH中,以及配置JAVA\_HOME的话,可以配置脚本为用户修改这两个环境变量。

我是在\lstinline$~/.bashrc$文件中加入的。这部分再看看Linux就是这个范来补充一下。

\begin{Bash}[假设java安装在/opt/java中]
export JAVA_HOME=/opt/java
export PATH=$JAVA_HOME/bin:$PATH
\end{Bash}

\subsection{安装Maven}

安装Maven,配置环境变量M2\_HOME为Maven的安装目录,PATH中加入M2\_HOME/bin.由于Maven是Java程序,所以可以配置JVM参数,这部分不建议修改mvn脚本,而是添加MAVEN\_OPTS环境变量。这样的话,就方便升级。

\begin{Bash}
export MAVEN_OPTS="Xms128m -Xmx512m"
\end{Bash}

\subsection{settings.xml}

将settings.xml配置在\lstinline$~/.m2$目录中,这样为用户级别进行配置,而且方便升级。


\subsection{配置代理}


在settings.xml中配置代理。

\begin{XML}

<settings>

	<proxies>
		<proxy>
			<id>my-proxy</id>
			<active>true</active>
			<protocol>http</protocol>
			<host>218.14.227.197</host>
			<port>3128</port>
			<!--
			<username></username>
			<password></password>
			<nonProxyHosts>www.a.com|*.google.com</nonProxyHosts>
			-->
		</proxy>
	</proxies>

</settings>

\end{XML}

可以配置多个proxy元素,默认情况下第一个被激活的proxy会生效(active为true表示被激活)。

nonProxyHosts可以使用“|”来间隔区分多个主机名,也可以使用通配符"*"。


\subsection{IDE内嵌Maven}

IDE一般内嵌了一个Maven,这往往导致命令行和IDE中使用的不是同一个Maven。容易导致IDE和命令行行为不一致。

Preferences -> Maven -> Installation

这里会有一个 Embeded Maven,选择Add来添加安装的Maven,然后选择这个外部Maven。

\section{快速入门}


\subsection{编写POM}

创建hello-world目录,在其中添加pom.xml文件。

\begin{XML}[pom.xml]
<?xml version="1.0" encoding="utf-8" ?>
<project xmlns="http://maven.apache.org/POM/4.0.0" xmlns:xsi="http://www.w3.org/2001/XMLSchema-instance"
    xsi:schemaLocation="http://maven.apache.org/POM/4.0.0 http://maven.apache.org/xsd/maven-4.0.0.xsd">
	<modelVersion>4.0.0</modelVersion>
	<groupId>com.level.maven</groupId>
	<artifactId>hello-world</artifactId>
	<version>1.0-SNAPSHOT</version>
	<name>Maven Hello World Project</name>
</project>
\end{XML}

Maven2, 3都必须使用POM4.0

然后最重要的三个元素groupId, artifactId和version,这三个元素定义了一个项目在Maven中基本坐标。

\section{坐标和依赖}

Maven的坐标:
\begin{itemize}
\item groupId Maven项目隶属的实际项目,groupId不应该对应项目隶属的组织或者公司。因为一个组织下会有很多项目。groupId和报名表示方法类似,与域名反向意义对应。
\item artifactId 定义实际项目中的一个Maven项目(模块)。推荐做法是使用实际项目名作为前缀,例如nexus-indexer。 
\item version 定义Maven项目当前所处版本。SNAPSHOT表示当前版本还不是稳定的,每天会做更新。
\item packaging Maven项目打包方式,默认是jar,对于jar和war,打包会使用不同的命令。但是并不一定就是对应扩展名。
\item classifier 定义构建输出的一些附属构件。。。。这个没看得太懂。
\end{itemize}


项目构件的文件名与坐标相对应,一般是artifactId-version[-classifier].packaging,

