\chapter{排序}

\section{快速排序}
最坏为 $\theta(n^2)$,期望的运行时间为$\theta(nlg n)$

\subsection{快速排序的描述}
分治过程:

分解:数组A[p..r]被划分为两部分(可能空)字数组A[p..q-1]和A[q+1..r],是的A[p..q-1]的所有元素都小于等于A(q),而A[q+1..r]中的所有元素都大于等于A(q)。

解决:通过递归调用快速排序,对子数组A[p..q-1]和A[q+1..r]排序。

合并:因为排序是就地进行,所以不需要合并,数组已经收有序。

\begin{CSharp}
        public void QuickSortArray(int[] a, int p, int r)
        {
            if (p < r){
                int q = Partition(a, p, r);
                QuickSortArray(a, p, q-1);
                QuickSortArray(a, q+1, r);
            }
        }
\end{CSharp}

\subsubsection{数组的划分}

\paragraph{Lomuto's} Partition
算法的关键就是Partition的过程。这个算法中,x用来作为将被放在A[q]位置的值,i用来记录小于等于x的元素的最大位置,j用来记录已经做过比较的值得位置。等到一轮比较完结之后,p到i位置所有的元素都小于x,交换i+1和r位置,则,i+1之前都小于等于x,i+1止呕都大于x。
\begin{CSharp}
        private int Partition(int[] a, int p, int r)
        {
            int x = a[r];
            int i = p - 1;
            for(int j = p; j < r; j++)
            {
                if (a[j] <= x)
                {
                    i = i + 1;
                    Utils.Swap(ref a[i], ref a[j]);
                }
            }
            Utils.Swap(ref a[i + 1], ref a[r]);
            return i + 1;
        }
\end{CSharp}

对上面的Partition做一个随机化处理,可以得到更好的平均性能。
\begin{CSharp}
        public void RandomQuickSortArray(int[] a, int p, int r)
        {
            if (p < r)
            {
                int q = Partition(a, p, r);
                RandomQuickSortArray(a, p, q - 1);
                RandomQuickSortArray(a, q + 1, r);
            }
        }

        private int RandomPartition(int[] a, int p, int r)
        {
            Random rnd = new Random();
            int i = rnd.Next(p, r + 1);
            Utils.Swap(ref a[i], ref a[r]);
            return Partition(a, p, r);
        }
\end{CSharp}

\paragraph{Hoare's} Partition
\begin{CSharp}
        public void HoareQuickSortArray(int[] a, int p, int r)
        {
            if (p < r)
            {
                int q = HoarePartition(a, p, r); // p <= q < r
                HoareQuickSortArray(a, p, q);
                HoareQuickSortArray(a, q + 1, r);
            }
        }

        private int HoarePartition(int[] a, int p, int r)
        {
            int x = a[p];
            int i = p - 1;
            int j = r + 1;
            while (true)
            {
                while (a[--j] > x) { }
                while (a[++i] < x) { }
                if (i < j)
                {
                    Utils.Swap(ref a[i], ref a[j]);
                }
                else
                {
                    return j;
                }
            }
        }
\end{CSharp}

\section{希尔排序}


\section{堆排序}
对一般通过一维数组来实现,堆的子节点小于父节点
\begin{itemize}
\item 父节点i的左孩子节点$2*i+1$
\item 父节点i的右孩子节点$2*i+2$
\item 子节点i的父节点$floor((i-1)/2)$
\end{itemize}

\begin{enumerate}
\item 建堆;
\item 移除堆的root元素,然后调整堆;
\item 当堆中所有元素都移除完毕,数组也就排好序了。
\end{enumerate}

\section{计数排序}
算法步骤:
\begin{enumerate}
\item 找出待排序的数组中最大和最小的元素
\item 统计数组中每个值为i的元素出现的次数,存入数组C的第i项
\item 对所有的计数累加(从C中的第一个元素开始,每一项和前一项相加)
\end{enumerate}
\section{归并排序 Merge}

\section{选择排序}
迭代一遍,每次选择剩下的元素的最小的,交换最小元素和第个i元素的位置。复杂度$n^2$
\section{插入排序}
迭代一遍,每次将第i个元素插入其前面的已排序的元素中。复杂度为$n^2$

\section{冒泡排序}





