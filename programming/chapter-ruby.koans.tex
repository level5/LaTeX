\chapter{Ruby Koans}

开始使用Koans来学习Ruby,第一步,当然是安装Ruby,然后设置环境变量。

\section{Koans流程}

\subsection{assert}

我们可以通过assert来测试确定我们的代码是否按照我们期望的方式来运行。可以使用assert和assert\_equal。

\lstinputlisting[
	style=ruby,
	linerange={10-10},
	firstnumber=1
]{code/ruby/koans/about_asserts.rb}

\lstinputlisting[
	style=ruby,
	linerange={33-33},
	firstnumber=1
]{code/ruby/koans/about_asserts.rb}

\subsection{nil}

和其他语言不太一样,Ruby中,表示Null的nil是一个对象。

\lstinputlisting[
	style=ruby,
	linerange={5-5},
	firstnumber=1
]{code/ruby/koans/about_nil.rb}

\lstinputlisting[
	style=ruby,
	linerange={25-27},
	firstnumber=1
]{code/ruby/koans/about_nil.rb}


\lstinputlisting[
	style=ruby,
	linerange={38-38},
	firstnumber=1
]{code/ruby/koans/about_nil.rb}


\subsection{Object}
万物皆为Object。Object是默认的所哟Ruby对象的祖先。

\lstinputlisting[
	style=ruby,
	linerange={5-9},
	firstnumber=1
]{code/ruby/koans/about_objects.rb}

每个对象有一个不同的object\_id


\subsection{Array}
数组可以直接通过Array.new或者[]来创建,数组长度可以动态增长。可以通过length来取得数组的长度。注意<<,他是将333添加到数组的最后,类似于push方法,不过这个方法返回数组本身,所以可以连续添加。

\lstinputlisting[
	style=ruby,
	linerange={11-24},
	firstnumber=1
]{code/ruby/koans/about_arrays.rb}


数组的index从0开始,可以是正数,也可以是负数,负数的话,就是从最后往前计算。

\lstinputlisting[
	style=ruby,
	linerange={28-35},
	firstnumber=1
]{code/ruby/koans/about_arrays.rb}

通过array[start, length]可以取得数组中的一个子数组。注意最后两行个结果的区别,一个是[],一个是nil。如果start是数组的结尾,返回[],如果start超出数组长度,则返回nil。

\lstinputlisting[
	style=ruby,
	linerange={39-47},
	firstnumber=1
]{code/ruby/koans/about_arrays.rb}

通过range也可以取得array[start..end],或者array[start...end]。区别是...不包括最后一个元素。同样start超出或者在结尾和上面产生相同效果。

\lstinputlisting[
	style=ruby,
	linerange={58-64},
	firstnumber=1
]{code/ruby/koans/about_arrays.rb}

同时,其他语言一般有的push, pop(最后添加和弹出), unshift, shift(最前面插入和弹出)方法也有。

\lstinputlisting[
	style=ruby,
	linerange={68-75},
	firstnumber=1
]{code/ruby/koans/about_arrays.rb}

\lstinputlisting[
	style=ruby,
	linerange={79-86},
	firstnumber=1
]{code/ruby/koans/about_arrays.rb}

对于数组变量的赋值,Ruby提供了一些很方便的操作。

\lstinputlisting[
	style=ruby,
	linerange={10-12},
	firstnumber=1
]{code/ruby/koans/about_array_assignment.rb}

可以看到变量和数组中的元素是按从开始到结尾的顺序一一对应的,如果变量不够就抛弃后面的元素,如果元素不够的话,就分配nil。
\lstinputlisting[
	style=ruby,
	linerange={16-18},
	firstnumber=1
]{code/ruby/koans/about_array_assignment.rb}

\lstinputlisting[
	style=ruby,
	linerange={28-30},
	firstnumber=1
]{code/ruby/koans/about_array_assignment.rb}

\lstinputlisting[
	style=ruby,
	linerange={34-36},
	firstnumber=1
]{code/ruby/koans/about_array_assignment.rb}

如果想要最后一个元素保存剩余元素的一个数组,使用*放在变量之前。

\lstinputlisting[
	style=ruby,
	linerange={22-24},
	firstnumber=1
]{code/ruby/koans/about_array_assignment.rb}

如果只想取得第一个元素,可以像下面这样,变量后面加一个,

\lstinputlisting[
	style=ruby,
	linerange={40-41},
	firstnumber=1
]{code/ruby/koans/about_array_assignment.rb}

下面是一个方便交换变量的值的方法。

\begin{Ruby}
	first_name = "Roy"
	last_name = "Rob"
	first_name, last_name = last_name, first_name
	assert_equal "Rob", first_name
	assert_equal "Roy", last_name
\end{Ruby}


\subsection{Hash}
Hash的创建可以通过字面常量,new方法。

\begin{Ruby}
	hash = { :one => "uno", :two => "dos" }
	assert_equal "uno", hash[:one]
	assert_equal "dos", hash[:two]
	assert_equal nil, hash[:doesnt_exist]
\end{Ruby}

当new不带参数时,创建一个空的Hash。当new带一个参数是,设定了default值,当Hash中不存在某个key时,返回default值。
default值也可以通过hash对象的default来修改。

\begin{Ruby}
    hash1 = Hash.new
    hash1[:one] = 1

    assert_equal 1, hash1[:one]
    assert_equal nil, hash1[:two]

    hash2 = Hash.new("dos")
    hash2[:one] = 1

    assert_equal 1, hash2[:one]
    assert_equal "dos", hash2.default
    assert_equal "dos", hash2[:two]
    assert_equal "dos", hash2[:three]

    hash2.default = "uno"

    assert_equal "uno", hash2[:four]
\end{Ruby}

当new带有代码块是,取某个不存在的值的时候,会调用代码块。

\begin{Ruby}
    hash = Hash.new {|hash, key| hash[key] = [] }

    hash[:one] << "uno"
    hash[:two] << "dos"

    assert_equal ["uno"], hash[:one]
    assert_equal ["dos"], hash[:two]
    assert_equal [], hash[:three]
\end{Ruby}


\subsection{String}
我们可以使用'或者"来生成一个String的字面常量。
\begin{Ruby}
    string = "Hello, World"
    assert_equal true, string.is_a?(String)
    
    string = 'He said, "Go Away."'
    assert_equal "He said, \"Go Away.\"", string
    
\end{Ruby}

他们的不同之处是"中可以插入变量,同时,有些转义符在'中不生效。
\begin{Ruby}
	value = 123
    string = "The value is #{value}"
    assert_equal "The value is 123", string
    
    value = 123
    string = 'The value is #{value}'
    assert_equal "The value is \#{value}", string
    
    string = "\n"
    assert_equal 1, string.size
    
    string = '\n'
    assert_equal 2, string.size
    
    string = '\\\''
    assert_equal 2, string.size
    assert_equal "\\'", string
\end{Ruby}

除了"和',我们还可以自定义包裹字符串的符号。(TODO:具体了解一下这个地方左右是怎么对应的。)如:
\begin{Ruby}
    a = %(flexible quotes can handle both ' and " characters)
    b = %!flexible quotes can handle both ' and " characters!
    c = %{flexible quotes can handle both ' and " characters}
    assert_equal true, a == b
    assert_equal true, a == c
\end{Ruby}

还有一种叫做Here Document的方式,采用<<加一个包裹符号。
\begin{Ruby}
    long_string = <<EOS
It was the best of times,
It was the worst of times.
EOS
    assert_equal 53, long_string.length
    assert_equal 2, long_string.lines.count
    assert_equal 'I', long_string[0,1]
\end{Ruby}
他的内容是EOS和EOS之间的内容,第一个EOS之后的回车不会计算到字符串中,这不同于前面的方式,下面代码的第一个回车是会计入到字符串中间的。
\begin{Ruby}
    long_string = %{
It was the best of times,
It was the worst of times.
}
    assert_equal 54, long_string.length
    assert_equal 3, long_string.lines.count
    assert_equal "\n", long_string[0,1]
\end{Ruby}

我们还可以很方便的使用[]取得子字符串,其中的参数是start, length。如果只有一个参数,就表示取得这个索引的字符。和数组一样。
\begin{Ruby}
    string = "Bacon, lettuce and tomato"
    assert_equal "let", string[7,3]
    assert_equal "let", string[7..9]
    
    string = "Bacon, lettuce and tomato"
    assert_equal "a", string[1]
\end{Ruby}

数组的拼接有两种方式,+和<<,两者稍有不同,可以从下面的代码看出来。
\begin{Ruby}
    string = "Hello, " + "World"
    assert_equal "Hello, World", string
    
    hi = "Hello, "
    there = "World"
    string = hi + there
    assert_equal "Hello, ", hi
    assert_equal "World", there
    
    hi = "Hello, "
    there = "World"
    hi += there
    assert_equal "Hello, World", hi
    
    original_string = "Hello, "
    hi = original_string
    there = "World"
    hi += there
    assert_equal "Hello, ", original_string   
    
    hi = "Hello, "
    there = "World"
    hi << there
    assert_equal "Hello, World", hi
    assert_equal "World", there

    original_string = "Hello, "
    hi = original_string
    there = "World"
    hi << there
    assert_equal "Hello, World", original_string             
\end{Ruby}
+是不会改变原来的字符串的,+=也只是将变量引用新的字符串。而<<则修改了字符串的内容。

然后就还有一些基本的方法,如拼接,分隔。这些方法其他语言都有。
\begin{Ruby}
    string = "Sausage Egg Cheese"
    words = string.split
    assert_equal ["Sausage", "Egg", "Cheese"], words
    
    string = "the:rain:in:spain"
    words = string.split(/:/)
    assert_equal ["the", "rain", "in", "spain"], words
    
    words = ["Now", "is", "the", "time"]
    assert_equal "Now is the time", words.join(" ")        
\end{Ruby}

最后,有一点需要注意,相同的字面常量引用的实际上不是一个对象。
\begin{Ruby}
    a = "a string"
    b = "a string"

    assert_equal true, a           == b
    assert_equal false, a.object_id == b.object_id
\end{Ruby}


\subsection{Symbol}
我的感觉Symbol就是一个特殊类型的对象,使用:name或者:"name"无论在什么地方都是引用的相同的对象,具有相同的object\_id。要注意的是,它本身不是string,而是一种特殊类型的对象。
\begin{Ruby}
    symbol1 = :a_symbol
    symbol2 = :a_symbol
    symbol3 = :something_else

    assert_equal true, symbol1 == symbol2
    assert_equal false, symbol1 == symbol3
    
    string = "catsAndDogs"
    assert_equal :catsAndDogs, string.to_sym
    assert_equal :catsAndDogs, :"catsAndDogs"	    
    
    symbol = :"cats and dogs"
    assert_equal "cats and dogs".to_sym, symbol
\end{Ruby}

(TODO:目前对symbol的印象和作用了解的还不是很深,等之后再补充)



\subsection{Regular Expression}
目前来看基本和其他语言的正则表达式没啥区别,需要注意的是是贪婪匹配。string[]没有匹配时返回nil。string[]最左边的匹配会被返回。
\begin{Ruby}
	assert_equal "abb", "abbcccddddeeeee"[/ab*/]

    assert_equal nil, "some matching content"[/missing/]
    
    assert_equal "a", "abbccc az"[/az*/]
\end{Ruby}

下面是我以前没有注意的
\begin{itemize}
\item .表示除了\textbackslash\ n所有的字符。
\item \textbackslash\ A表示字符串的头,\textbackslash\ Z表示字符串的结尾。
\end{itemize}

然后就是一些貌似Ruby有的\$1, \$2在匹配之后可以使用
\begin{Ruby}
    assert_equal "Gray, James", "Name:  Gray, James"[/(\w+), (\w+)/]
    assert_equal "Gray", $1
    assert_equal "James", $2
\end{Ruby}

另外3个String中和正则相关的方法。scan找出所有匹配,sub替换第一个找到的匹配,gsub替换所有找到的匹配
\begin{Ruby}
    assert_equal ["one", "two", "three"], "one two-three".scan(/\w+/)
    
    assert_equal "one t-three", "one two-three".sub(/(t\w*)/) { $1[0, 1] }
    
    assert_equal "one t-t", "one two-three".gsub(/(t\w*)/) { $1[0, 1] }     
\end{Ruby}


\subsection{Methods}
依稀记得,Ruby中所有的操作符什么的都是方法(TODO:这里以后再补充,具体现在也不清楚)。

使用def来定义方法,对方法的调用可以带括号,或者不带括号。
\begin{Ruby}
	def my_global_method(a,b)
  		a + b
	end

	assert_equal 5, my_global_method(2,3)	
	
	result = my_global_method 2, 3
    assert_equal 5, result
\end{Ruby}

方法定义的时候可以给参数带默认值
\begin{Ruby}
  	def method_with_defaults(a, b=:default_value)
    	[a, b]
  	end
  
    assert_equal [1, :default_value], method_with_defaults(1)
    assert_equal [1, 2], method_with_defaults(1, 2)
\end{Ruby}

可变参数
\begin{Ruby}
  	def method_with_var_args(*args)
   		args
  	end
  
    assert_equal Array, method_with_var_args.class # call the method without arguments
    assert_equal [], method_with_var_args
    assert_equal [:one], method_with_var_args(:one)
    assert_equal [:one, :two], method_with_var_args(:one, :two)
\end{Ruby}

如果方法没有显示的返回,就返回方法最后一行代码的返回值。
\begin{Ruby}
  	def method_without_explicit_return
    	:a_non_return_value
    	:return_value
  	end
  	
  	assert_equal :return_value, method_without_explicit_return
\end{Ruby}


对于同一个class的方法,可以直接调用,或者self.来调用。

定义私有方法。(TODO:补充)
\begin{Ruby}
  class Dog
    def name
      "Fido"
    end

    private

    def tail
      "tail"
    end
  end
\end{Ruby}

\subsection{Keyword Arguments}

\subsection{Control Statements}

\subsection{True \& False}

\subsection{Exception}


































