\chapter{ECMA 6}

\section{Basic}

\subsection{Better Unicode Support}

\subsubsection{code point}

\subsection{new methods for string}

\subsubsection{Object.is}

\subsection{Regex}

\subsubsection{u}

\subsubsection{y}

\subsection{let}

\subsubsection{var}

\subsubsection{let in block scope}

\subsubsection{let in loop}

\subsubsection{let Vs. var}

\subsubsection{Global let declaration}

\subsubsection{constant declarations}
\begin{itemize}
\item Constants are also block-level declarations, similar to 'let'.
\item Constants are destroyed once execution flow out of the block
\item declarations are hoisted to the top of block.
\end{itemize}

\subsubsection{caution about const}

\subsubsection{Number}


\section{Function}


\subsection{Default Parameters}

\begin{JavaScript}[ecma 5实现默认参数的方式]
function makeRequest(url, timeout, callback) {

    timeout = timeout || 2000;
    callback = callback || function() {};

    // the rest of the function

}
\end{JavaScript}

此处一个小小的问题是如果timeout是0也会导致timeout被替换成2000.另外一个方式是通过arguments.length来判断参数的个数。

\begin{JavaScript}[ecma 6的方式]
function makeRequest(url, timeout = 2000, callback = function() {}) {

    // the rest of the function

}
\end{JavaScript}

设置默认参数不需要一定是在方法的最后
\begin{JavaScript}[这样也是可以的]
function makeRequest(url, timeout = 2000, callback) {

    // the rest of the function

}
\end{JavaScript}

这个时候,要使用默认参数,可以只带一个参数,或者第二个参数显示的指定为undefined
\begin{JavaScript}[第一和第二个调用使用默认参数,第三个不使用默认参数]
// uses default timeout
makeRequest("/foo", undefined, function(body) {
    doSomething(body);
});

// uses default timeout
makeRequest("/foo");

// doesn't use default timeout
makeRequest("/foo", null, function(body) {
    doSomething(body);
});
\end{JavaScript}

哈哈,默认参数不是必须是基本类型,甚至可以是返回函数的一个函数调用。
\begin{JavaScript}[函数调用返回函数的默认值]
function getCallback() {
    return function() {
        // some code
    };
}

function makeRequest(url, timeout = 2000, callback = getCallback()) {

    // the rest of the function

}
\end{JavaScript}

每次当最后一个参数不传入的时候,getCallback都将会被调用一次。

\begin{JavaScript}
// 此处应该加代码来证明这一点。
\end{JavaScript}

\subsection{Rest parameters}

Can't have a named parameter after rest parameters.

\subsection{Destructured Parameters}
\begin{JavaScript}
function setCookie(name, value, options) {

    options = options || {};

    var secure = options.secure,
        path = options.path,
        domain = options.domain,
        expires = options.expires;

    // ...
}

setCookie("type", "js", {
    secure: true,
    expires: 60000
});
\end{JavaScript}
上面的代码采用Destructured Parameters的话,是下面这样的:
\begin{JavaScript}
function setCookie(name, value, { secure, path, domain, expires }) {

    // ...
}

setCookie("type", "js", {
    secure: true,
    expires: 60000
});
\end{JavaScript}

One quirk of this pattern is that the destructured parameters throw an error when the argument isn't provided.
因为编译器实际上是产生下面这样的代码
\begin{JavaScript}
function setCookie(name, value, options) {

    var { secure, path, domain, expires } = options;

    // ...
}
\end{JavaScript} 

Since destructuring assignment throws an error when the right side expression evaluates to `null` or `undefined`, the same is true when the third argument isn't passed.

可以通过提供一个默认参数来避免上面的情况
\begin{JavaScript}
function setCookie(name, value, { secure, path, domain, expires } = {}) {

    // ...
}
\end{JavaScript}

It's recommended to always provide the default value for destructured parameters to avoid all errors that are unique to their usage.


\subparagraph{The Spread Operator}

\begin{JavaScript}
let values = [25, 50, 75, 100]

// equivalent to
// console.log(Math.max(25, 50, 75, 100));
console.log(Math.max(...values));           // 100
\end{JavaScript}

可以和其他参数混合使用
\begin{JavaScript}
let values = [-25, -50, -75, -100]

console.log(Math.max(...values, 0));        // 0
\end{JavaScript}

\subsection{name property}
每个function都有一个name property。会在函数创建时生成。他的writable属性是false.

\subsection{Block-Level Functions}

\begin{JavaScript}[ecma 5 stric mode下,在block中定义函数会抛错]
"use strict";

if (true) {

    // Throws a syntax error in ES5, not so in ES6
    function doSomething() {
        // ...
    }
}
\end{JavaScript}

因为各个浏览器在非strict模式下,定义函数的行为不统一,应该尽量避免在block中申明函数,而是采用function expression.

而在ecma 6中,Block-Level function是合法的。
\begin{JavaScript}[ecma6中]
"use strict";

if (true) {

    console.log(typeof doSomething);        // "function"

    function doSomething() {
        // ...
    }

    doSomething();
}

console.log(typeof doSomething);            // "undefined"
\end{JavaScript}

这个时候,块级别的函数申明会被提升到块的开头。而在块结束之后,对应的变量就不存在了。 和let有一点类似,唯一的不同是let不会将定义提升到代码块的最开始。

\begin{JavaScript}
"use strict";

if (true) {

    console.log(typeof doSomething);        // "function"

    function doSomething() {
        // ...
    }

    doSomething();
}

console.log(typeof doSomething);            // "undefined"
\end{JavaScript}

\begin{JavaScript}[let不会提升定义到代码块的最前面,所以下面是会报错的]
"use strict";

if (true) {

    console.log(typeof doSomething);        // throws error

    let doSomething = function () {
        // ...
    }

    doSomething();
}

console.log(typeof doSomething);
\end{JavaScript}

在非严格模式下,函数声明会被提升到全局或者包含函数中。

\begin{JavaScript}
// ECMAScript 6 behavior
if (true) {

    console.log(typeof doSomething);        // "function"

    function doSomething() {
        // ...
    }

    doSomething();
}

console.log(typeof doSomething);            // "function"
\end{JavaScript}


\section{Object}

\subsection{Object literal}

\subsubsection{Property初始化的简写}

\begin{JavaScript}[ecma5的写法,重复输入]
function createPerson(name, age) {
    return {
        name: name,
        age: age
    };
}
\end{JavaScript}

\begin{JavaScript}[ecma6的写法,省略重复输入]
function createPerson(name, age) {
    return {
        name,
        age
    };
}
\end{JavaScript}
如果property字面量只有一个名字,JavaScript引擎会查找包含他的scope来查找名字相同的变量,如果找到了,就将变量的值赋予这个对象对应的属性名。

\subsubsection{Method初始化的简写}
\begin{JavaScript}[ecma5]
var person = {
    name: "Nicholas",
    sayName: function() {
        console.log(this.name);
    }
};
\end{JavaScript} 

\begin{JavaScript}[ecma6]
var person = {
    name: "Nicholas",
    sayName() {
        console.log(this.name);
    }
};
\end{JavaScript}

\subsubsection{Computed Property Names}
在ECMA5中,可以使用引号将带空格的字面常量定义为Property名,但是如果是变量引用的字面常量的话,就没有办法在literal Object中定义了。
\begin{JavaScript}
var person = {
    "first name": "Nicholas"
};

console.log(person["first name"]);   
\end{JavaScript}

\begin{JavaScript}[ecma6中,可以引用变量中的带空白字符串]
var lastName = "last name";

var person = {
    "first name": "Nicholas",
    [lastName]: "Zakas"
};

console.log(person["first name"]);      // "Nicholas"
console.log(person[lastName]);          // "Zakas"
\end{JavaScript}
 
\begin{JavaScript}[ecma6]
var suffix = " name";

var person = {
    ["first" + suffix]: "Nicholas",
    ["last" + suffix]: "Zakas"
};

console.log(person["first name"]);      // "Nicholas"
console.log(person["last name"]);       // "Zakas"
\end{JavaScript}
 
所有可以放入方括号的符号都可以放在Computed Property Names中。

\subsubsection{get set}
这个ECMA5中就可以使用,只是我不知道而已
\begin{JavaScript}
var foo = {
	set bar(value){
		this.x = value;
	},
	get bar() {
		return this.x;
	}
};
foo.bar = 100;

console.log(foo.bar); // 100
console.log(foo.x);   // 100

\end{JavaScript}
 

\subsection{Object.assign()}

\subsubsection{mixins} 

\begin{JavaScript}
unction mixin(receiver, supplier) {
    Object.keys(supplier).forEach(function(key) {
        receiver[key] = supplier[key];
    });

    return receiver;
}
\end{JavaScript}

是的receiver不通过继承来获取新的行为。


ECMA6添加的Object.assign()方法就是实现mixin的。不同之处是通过名字来反映了真正发生了什么操作,因为是通过=号来拷贝属性的,所以accessor properties没办法被当做accessor属性拷贝过去。

\begin{JavaScript}
//此处应该有代码
\end{JavaScript}

\subsubsection{Duplicate Object Literal Properties}
在ECMA5的strict模式下,如果有重复的property名字的时候,会抛出异常。而这一行为在ECMA6的strict模式下被移除掉了。

\subsubsection{改变Prototype}
ECMA5中添加了方法来获取Prototype,Object.getPrototypeOf(),而在ECMA6中,与之对应的Object.setPrototypeOf()被添加了进来。
\begin{JavaScript}
//此处应该有代码
\end{JavaScript}

%In ECMAScript 6 engines, `Object.prototype.__proto__` is defined as an accessor property whose `get` method calls `Object.getPrototypeOf()` and whose `set` method calls `Object.setPrototypeOf()`. This means that there is no real difference between using `__proto__` and the other methods except that `__proto__` allows you to set the prototype of an object literal directly.

\begin{JavaScript}
//此处应该有代码
\end{JavaScript}

对于\lstinline$__proto__$来说,有一些特别之处
\begin{itemize}
\item 只能在字面对象定义中出现一次\lstinline$__proto__$属性,出现两次将报错。
\item 使用Computed form \lstinline$['__proto__']$的话,定义的是一般的property。不会影响对象的\lstinline$[[prototype]]$。(这一点是为什么需要查文档来看看原因是什么)
\end{itemize}

\subsection{super 关键字}
通过super来引用对象的\lstinline$[[prototype]]$。
\begin{JavaScript}
let person = {
    getGreeting() {
        return "Hello";
    }
};

let dog = {
    getGreeting() {
        return "Woof";
    }
};

// prototype is person
let friend = {
    __proto__: person,
    getGreeting() {
        // same as this.__proto__.getGreeting.call(this)
        return Object.getPrototypeOf(this).getGreeting.call(this) + ", hi!";
    }
};

console.log(friend.getGreeting());                      // "Hello, hi!"
console.log(Object.getPrototypeOf(friend) === person);  // true
console.log(friend.__proto__ === person);               // true

// set prototype to dog
friend.__proto__ = dog;
console.log(friend.getGreeting());                      // "Woof, hi!"
console.log(friend.__proto__ === dog);                  // true
console.log(Object.getPrototypeOf(friend) === dog);     // true
\end{JavaScript}

在这个代码中,需要注意的是通过\lstinline$Object.getPrototypeOf()$来获取\lstinline$[[prototype]]$,通过\lstinline$call(this)$来保证方法执行时this能够被正确的设定,这个时候的this是当前对象而不是他的原型。

\begin{JavaScript}[ecma6中简单的使用super就可以达到上面的效果了]
let friend = {
    __proto__: person,
    getGreeting() {
        // same as Object.getPrototypeOf(this).getGreeting.call(this)
        // or this.__proto__.getGreeting.call(this)
        return super.getGreeting() + ", hi!";
    }
};
\end{JavaScript}

调用\lstinline$super.getGreeting.call(this)$的效果和调用\lstinline$Object.getPrototypeOf(this).getGreeting.call(this)$与\lstinline$this.__proto__.getGreeting.call(this)$的效果是一样的. 

如果调用的方法的名字是一样的,那么可以直接使用\lstinline$super()$来调用。

\begin{JavaScript}
let friend = {
    __proto__: person,
    getGreeting() {
        // same as Object.getPrototypeOf(this).getGreeting.call(this)
        // or this.__proto__.getGreeting.call(this)
        // or super.getGreeting()
        return super() + ", hi!";
    }
};
\end{JavaScript}

实际效果是\lstinline$super()$使用containing function的name属性作为名字来查找正确的方法。

\begin{JavaScript}
//此处应该有代码来说明上面的观点.
\end{JavaScript}

super只能在function内使用,而不能在全局范围使用.


\subsection{Method}

ECMA6之前是没有method的正式概念的,method就是对象的属性的值是一个function。而在ECMA6中正式定义了method,method就是有\lstinline$[[HomeObject]]$的function,这个属性的值是method的拥有者。

\begin{JavaScript}
let person = {

    // method
    getGreeting() {
        return "Hello";
    }
};

// not a method
function shareGreeting() {
    return "Hi!";
}
\end{JavaScript}

\lstinline$[[HomeObject]]$和super有着密切的关系,super使用\lstinline$[[HomeObject]]$来决定怎么做,第一步,对\lstinline$[[HomeObject]]$调用\lstinline$Object.getPrototypeOf()$来获取原型的引用,接着,使用executing function的名字来找到对应的要被执行的function.接着设定this并执行method。

\begin{JavaScript}
let person = {
    getGreeting() {
        return "Hello";
    }
};

// prototype is person
let friend = {
    __proto__: person,
    getGreeting() {
        return super() + ", hi!";
    }
};

function getGlobalGreeting() {
    return super.getGreeting() + ", yo!";
}

console.log(friend.getGreeting());  // "Hello, hi!"

getGlobalGreeting();                      // throws error
\end{JavaScript}

\begin{JavaScript}
//这里需要大量代码来说明上面的问题。
\end{JavaScript}

每个function都有一个\lstinline$toMethod()$的方法来生成一个相同版本的function,他的\lstinline$[[HomeObject]]$被指定为新的对象。
(此处需要读文档,看看是是不是新创建了一个function,他的\lstinline$[[Code]]$和原来function一样,\lstinline$[[Scope]]$和原来的function一样,这样就保证了代码和Scope Chain是一样的。)
\begin{JavaScript}
// 也许需要代码来说明
\end{JavaScript}

这里可以看到this和super有一个很大的区别,this是在运行时决定他的取值,而super是在function被创建的时候就决定好了。\lstinline$[[HomeObject]]$在创建之后就不能被改变了。所以对于super的使用,如果function被传递来传递去的话,很难搞清楚这个是super到底是什么值了。因此最好不要再对象被创建之后在上面删除或者添加方法。如果一定要这么做,可以使用\lstinline$toMethod()$方法来指定新的\lstinline$[[HomeObject]]$.

\begin{JavaScript}
//需不需要代码带说明一下呢?
\end{JavaScript}

\section{Symbol}




