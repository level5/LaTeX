\chapter{Grommet}

\section{node}

\subsection{npm}

\subsubsection{代理,仓库的配置}

\subsubsection{package.json}

\subsubsection{看到的有用的module}

\begin{Bash}

# serve, 在存放静态文件的目录运行serve,自动启动一个端口为3000的server来访问这些静态文件
npm i serve -g


\end{Bash}

\subsection{express}

这里先不用管....

\section{Gulp}

\subsection{steam的概念是个基础}

看上去挺简单的,就是几个接口

\begin{JavaScript}[Gulp]

gulp.task()

gulp.src()

gulp.dest()

gulp.watch()

\end{JavaScript}

\subsection{dev task}

\subsection{test task}

\subsection{release task}


\section{webpack}

gulp和webpack的区别.

\subsection{js模块化}
目前主流三种模块方式
\begin{itemize}
\item CommonJS
\begin{JavaScript}
var module = require('path/module')
\end{JavaScript}
\item AMD
\begin{JavaScript}
require(['module1', 'module2'], function(module1, module2) {
	return newModule
})
\end{JavaScript}
\item es2015 Module
\begin{JavaScript}

\end{JavaScript}
\end{itemize}

\section{Babel}


babel的配置\lstinline$.bablerc$文件

babel6的两种插件:
\begin{itemize}
\item sytnax
\item tranform???
\end{itemize}

\section{React}

\subsection{Virtual DOM}


\subsection{知识点}

\subsubsection{state}

\subsubsection{life cycle}

\subsubsection{Refs to Components}


\section{React-Router}

\begin{JavaScript}[目前使用的配置方案]

const routeConfig = [
  { path: '/',
    component: App,
    indexRoute: { component: Dashboard },
    childRoutes: [
      { path: 'about', component: About },
      { path: 'inbox',
        component: Inbox,
        childRoutes: [
          { path: '/messages/:id', component: Message },
          { path: 'messages/:id',
            onEnter: function (nextState, replaceState) {
              replaceState(null, '/messages/' + nextState.params.id)
            }
          }
        ]
      }
    ]
  }
]

// TODO: 查看一下API,看看history这个怎么定义
render(<Router routes={routeConfig} />, document.body)

\end{JavaScript}

\section{Flux}
...

\section{Redux}

\subsection{redux}

\subsubsection{三个优先原则}

由Flux而来,发展了Flux。

\begin{itemize}

\item The state of your whole application is stored in an object tree within a single store.

\item The only way to mutate the state is to emit an action, an object describing what happened.

保证view或者network的callback都不会去修改状态。

\item To specify how the state tree is transformed by actions, you write pure reducers.

什么时候reducer?

Reducers are just pure functions that take the previous state and an action, and return the next state. 

\end{itemize}

\subsubsection{和相似的模式的对比}

\paragraph{Flux}

\begin{itemize}

\item concentrate your model update logic in a certain layer of your application (“stores” in Flux, “reducers” in Redux). Instead of letting the application code directly mutate the data, both tell you to describe every mutation as a plain object called an “action”.

\item Redux does not have the concept of a Dispatcher; Redux assumes you never mutate your data(You should always return a new object)



\end{itemize}

\subsubsection{state的范式化, 参看normalizr }

\subsubsection{Action 和 Action创建函数}

Action 是把数据从应用.传到 store 的有效载荷。它是 store 数据的唯一来源。一般来说你会通过 store.dispatch() 将 action 传到 store.

Action创建函数应该是纯函数,这个点貌似我们的项目中做得有问题呀....

那么,改在哪里放松rest请求呢,如果不能再Action创建函数中的话!!!



\subsubsection{Reducer}

永远不要在 reducer 里做这些操作:

\begin{itemize}
\item 修改传入的参数
\item 执行有副作用的操作,如API请求和路由跳转
\end{itemize}


Reducer
\begin{itemize}
\item 不要修改 state
\item 在 default 情况下返回旧的 state。遇到未知的 action 时,一定要返回旧的 state。
\end{itemize}

\paragraph{组合Reducer}

注意每个 reducer 只负责管理全局 state 中它负责的一部分。每个 reducer 的 state 参数都不同,分别对应它管理的那部分 state 数据。

\begin{JavaScript}

function todos(state = [], action) {
  switch (action.type) {
    case ADD_TODO:
      return [...state, {
        text: action.text,
        completed: false
      }];
    case COMPLETE_TODO:
      return [
        ...state.slice(0, action.index),
        Object.assign({}, state[action.index], {
          completed: true
        }),
        ...state.slice(action.index + 1)
      ];
    default:
      return state;
  }
}

function visibilityFilter(state = SHOW_ALL, action) {
  switch (action.type) {
    case SET_VISIBILITY_FILTER:
      return action.filter;
    default:
      return state;
  }
}

function todoApp(state = {}, action) {
  return {
    visibilityFilter: visibilityFilter(state.visibilityFilter, action),
    todos: todos(state.todos, action)
  };
}


\end{JavaScript}

\begin{JavaScript}[使用Redux提供的工具]

import { combineReducers } from 'redux';

const todoApp = combineReducers({
  visibilityFilter,
  todos
});

export default todoApp;

\end{JavaScript}

\begin{JavaScript}[上面的代码完全等价于]

export default function todoApp(state, action) {
  return {
    visibilityFilter: visibilityFilter(state.visibilityFilter, action),
    todos: todos(state.todos, action)
  };
}

\end{JavaScript}

\begin{JavaScript}[再次说明]

const reducer = combineReducers({
  a: doSomethingWithA,
  b: processB,
  c: c
});


// 上面等价于下面这段:

function reducer(state, action) {
  return {
    a: doSomethingWithA(state.a, action),
    b: processB(state.b, action),
    c: c(state.c, action)
  };
}

\end{JavaScript}

\subsubsection{Store}

Redux 应用只有一个单一的 store, store的职责:
\begin{itemize}

\item 维持state 

\item 提供\lstinline$getState()$方法获取 state

\item 提供\lstinline$dispatch(action)$方法更新state

\item 提供\lstinline$subscribe(listner)$注册监听

\end{itemize}

\begin{JavaScript}[使用已有的reduce来创建]

import { createStore } from 'redux'
import todoApp from './reducers'
let store = createStore(todoApp)

\end{JavaScript}

\subsubsection{数据流动}

% 这里修改成ol
严格的单向数据流是 Redux 架构的设计核心(和Flux有共同的设计思想)。
\begin{itemize}

\item 调用 \lstinline$store.dispatch(action)$

\item Redux store 调用传入的 reducer 函数

\item 根 reducer 应该把多个子 reducer 输出合并成一个单一的 state 树

\item Redux store 保存了根 reducer 返回的完整 state 树

\end{itemize}

\subsection{react-redux redux的react绑定}


\begin{Bash}[需要单独安装]
npm install --save react-redux
\end{Bash}

只在最顶层组件(如路由操作)里使用 Redux。其余内部组件仅仅是展示性的,所有数据都通过 props 传入。


\subsection{redux and react}

怎么组织componet和redux???


My dumb components:
\begin{itemize}
\item Have no dependencies on the rest of the app, e.g. Flux actions or stores.
\item Often allow containment via this.props.children.
\item Receive data and callbacks exclusively via props.
\item Have a CSS file associated with them. ??? 这里怎么做???
\item Rarely have their own state.
\item Might use other dumb components.
\end{itemize}


\section{local storage}

\section{history}

\subsection{HTML5 History API}

history对象。

修改URL,但是页面并不会真正的跳转。不过,这些URL都会被保存在浏览器的历史记录中。如果用户返回上一张幻灯片,就会触发\lstinline$onPopState$事件。


\begin{itemize}

\item \lstinline$pushState()$ 只允许修改网页的URL部分。

\item \lstinline$onPopState$事件, 在用户点击回退的时候触发

\item \lstinline$replaceState()$ 修改与当前页面相关的状态。

\end{itemize}

\subsection{history库}



\section{Web Socket}

\subsection{简介}

\subsubsection{URL}

\begin{JavaScript}[ws和wss]


var socket = new WebSocket("ws://localhost/socketServer.php");

var socket = new WebSocket("wss://localhost/socketServer.php");

\end{JavaScript}


\subsection{Browser Webscoket API}

\begin{JavaScript}

// 补充代码

\end{JavaScript}


\subsection{websocket/ws}

\begin{JavaScript}[client]

var WebSocket = require('ws');
var ws = new WebSocket('ws://www.host.com/path');

ws.on('open', function open() {
  ws.send('something');
});

ws.on('message', function(data, flags) {
  // flags.binary will be set if a binary data is received.
  // flags.masked will be set if the data was masked.
});

\end{JavaScript}


\begin{JavaScript}[server]
var WebSocketServer = require('ws').Server
  , wss = new WebSocketServer({ port: 8080 });

wss.on('connection', function connection(ws) {
  ws.on('message', function incoming(message) {
    console.log('received: %s', message);
  });

  ws.send('something');
});
\end{JavaScript}

\begin{JavaScript}[express]
var server = require('http').createServer()
  , url = require('url')
  , WebSocketServer = require('ws').Server
  , wss = new WebSocketServer({ server: server })
  , express = require('express')
  , app = express()
  , port = 4080;

app.use(function (req, res) {
  res.send({ msg: "hello" });
});

wss.on('connection', function connection(ws) {
  var location = url.parse(ws.upgradeReq.url, true);
  // you might use location.query.access_token to authenticate or share sessions
  // or ws.upgradeReq.headers.cookie (see http://stackoverflow.com/a/16395220/151312)

  ws.on('message', function incoming(message) {
    console.log('received: %s', message);
  });

  ws.send('something');
});

server.on('request', app);
server.listen(port, function () { console.log('Listening on ' + server.address().port) });
\end{JavaScript}


\section{本地化}

\subsection{yahoo react-intl}