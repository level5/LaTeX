\chapter{Node}

事件驱动的非阻塞I/O.

\section{模块加载}

\section{异步I/O}


\subsubsection{轮询}

轮询技术:
\begin{itemize}
\item read 通过重复调用来检查I/O状态。

\item select ??? 改进在哪里??

\item 
\end{itemize}


\subsubsection{异步编程的难点}

\begin{itemize}
\item 异常处理.

Node在处理异常上形成了一种约定,将异常作为回调函数的第一个实参传回,如果为空值,
则表明异步调用没有异常抛出

\begin{JavaScript}
async(function (err, results) {
// TODO
});
\end{JavaScript}

在编写异步方法时,只要将异常正确地传递给用户的回调方法即可,无须过多处理。


\item 函数嵌套过深

\item 阻塞代码

\item 多线程编程

child_process是其基础API, cluster模块是更深层次的应用。

\item 异步转同步

对同步需求没有对应的API满足;
\end{itemize}


\subsection{异步编程解决方案}

\begin{itemize}
\item 事件发布/订阅模式
\item promise/deffered模式
\item 流程控制模式
\end{itemize}

\paragraph{xxx}