\chapter{Node}

事件驱动的非阻塞I/O.

\section{模块加载}

\section{异步I/O}


\subsubsection{轮询}

轮询技术:
\begin{itemize}
\item read 通过重复调用来检查I/O状态。

\item select ??? 改进在哪里??

\item 
\end{itemize}


\subsubsection{异步编程的难点}

\begin{itemize}
\item 异常处理.

Node在处理异常上形成了一种约定,将异常作为回调函数的第一个实参传回,如果为空值,
则表明异步调用没有异常抛出

\begin{JavaScript}
async(function (err, results) {
// TODO
});
\end{JavaScript}

在编写异步方法时,只要将异常正确地传递给用户的回调方法即可,无须过多处理。


\item 函数嵌套过深

\item 阻塞代码

\item 多线程编程

child\_process是其基础API, cluster模块是更深层次的应用。

\item 异步转同步

对同步需求没有对应的API满足;
\end{itemize}


\subsection{异步编程解决方案}

\begin{itemize}
\item 事件发布/订阅模式
\item promise/deffered模式
\item 流程控制模式
\end{itemize}


\paragraph{Deferreds}

deferreds,异步调用操作并不会block住调用,而是返回一个promise对象,我们可以往上面注册回调,当调用真正完成的时候,会回调注册的回调函数;

\begin{JavaScript}[伪码类似于这样]
var promise = callToAPI( arg1, arg2, ...);
 
promise.then(function( futureValue ) {
 
    // Handle futureValue
 
});
 
promise.then(function( futureValue ) {
 
    // Do something else
 
});
\end{JavaScript}

promise可能以两种方式结束,resolved和rejected


\begin{JavaScript}[伪码接受两个回调,处理两种情况]
promise.then(function( futureValue ) {
 
    // We got a value
 
}, function() {
 
    // Something went wrong
 
});
\end{JavaScript}

\begin{JavaScript}[当有多个promise的时候,可以注册等待多个promise都结束]
when(
    promise1,
    promise2,
    ...
).then(function( futureValue1, futureValue2, ... ) {
 
    // All promises have completed and are resolved
 
});
\end{JavaScript}

\begin{JavaScript}[比如有多个动画,我们要等待都完成再做其他操作]
var promise1 = $( "#id1" ).animate().promise();
var promise2 = $( "#id2" ).animate().promise();
when(
    promise1,
    promise2
).then(function() {
 
    // Once both animations have completed
    // we can then run our additional logic
 
});
\end{JavaScript}


\paragraph{Deferreds in jQuery}

jQuery的Deferreds遵循CommonJS Promise/A desgin。

\paragraph{xxx}