\chapter{JavaScript in Browser}



\section{History}

\begin{itemize}
\item he new parts in HTML5 include a way to add entries to the browser history, 
\item to visibly change the URL in the browser location bar (without triggering a page refresh), 
\item and an event that fires when those entries are removed from the stack by the user pressing the browser’s back button.
\end{itemize}



The HTML5 history API is actually designed to ensure that URLs continue to be useful in script-heavy web applications. 


如果改变URL,那么浏览器会刷新页面,即使新访问的页面和当前页面基本相同,浏览器还是会重新下载所有的东西。


HTML5 History API可以用来代替刷新整个页面,可以使用脚本来下载一部分页面。假设有A,B两个页面,基本相同,只有10\%的内容不一样,如果用户从A页面跳转到页面B.可以中断请求,并且

\begin{itemize}
\item 下载那10\%不同的内容(可能使用XMLHttpRequest来完成)。This will require some server-side changes to your web application. You will need to write code to return just the 10\% of page B that is different from page A. This can be a hidden URL or query parameter that the end user would not normally see. 

\item 交换改变的内容(使用innerHTML或者DOM methods)。You may also need to reset any event handlers on elements within the swapped-in content. 

\item 修改浏览器的地址栏到页面B。 using a particular method from the HTML5 history API that I’ll show you in a moment. 
\end{itemize}



\begin{JavaScript}
history.pushState(null, null, link.href);
\end{JavaScript}

\begin{JavaScript}
\item state 任何JSON格式的对象,他会被传递给popstate event hander.
\item title 任何string, 这个参数当前没有被主流的浏览器使用,如果要使用,可以再state中设定好,然后在popstate回调中手动设定。
\item url ,任何url,想要在地址栏显示url。
\end{JavaScript}

调用\lstinline$history.pushState$会马上改变地址栏的url。

在这之后,还需要fake向

\begin{JavaScript}
window.addEventListener("popstate", function(e) {
    swapPhoto(location.pathname);
});
\end{JavaScript}

\section{面试}

\subsection{浏览器兼容性问题}
* png24位的图片在iE6浏览器上出现背景,解决方案是做成PNG8.也可以引用一段脚本处理.

* 浏览器默认的margin和padding不同。解决方案是加一个全局的*{margin:0;padding:0;}来统一。

* IE6双边距bug:块属性标签float后,又有横行的margin情况下,在ie6显示margin比设置的大。 

* 浮动ie产生的双倍距离(IE6双边距问题:在IE6下,如果对元素设置了浮动,同时又设置了margin-left或margin-right,margin值会加倍。)
  #box{ float:left; width:10px; margin:0 0 0 100px;} 

 这种情况之下IE会产生20px的距离,解决方案是在float的标签样式控制中加入 ——_display:inline;将其转化为行内属性。(_这个符号只有ie6会识别)

*  渐进识别的方式,从总体中逐渐排除局部。 

  首先,巧妙的使用“\9”这一标记,将IE游览器从所有情况中分离出来。 
  接着,再次使用“+”将IE8和IE7、IE6分离开来,这样IE8已经独立识别。

  css
      .bb{
       background-color:#f1ee18;/*所有识别*/
      .background-color:#00deff\9; /*IE6、7、8识别*/
      +background-color:#a200ff;/*IE6、7识别*/
      _background-color:#1e0bd1;/*IE6识别*/ 
      } 

*  IE下,可以使用获取常规属性的方法来获取自定义属性,
   也可以使用getAttribute()获取自定义属性;
   Firefox下,只能使用getAttribute()获取自定义属性. 
   解决方法:统一通过getAttribute()获取自定义属性.

* IE下,event对象有x,y属性,但是没有pageX,pageY属性; 
  Firefox下,event对象有pageX,pageY属性,但是没有x,y属性.

* 解决方法:(条件注释)缺点是在IE浏览器下可能会增加额外的HTTP请求数。

* Chrome 中文界面下默认会将小于 12px 的文本强制按照 12px 显示, 
  可通过加入 CSS 属性 -webkit-text-size-adjust: none; 解决.

* 超链接访问过后hover样式就不出现了 被点击访问过的超链接样式不在具有hover和active了解决方法是改变CSS属性的排列顺序:
L-V-H-A :  a:link {} a:visited {} a:hover {} a:active {}

* 怪异模式问题:漏写DTD声明,Firefox仍然会按照标准模式来解析网页,但在IE中会触发怪异模式。为避免怪异模式给我们带来不必要的麻烦,最好养成书写DTD声明的好习惯。现在可以使用[html5](http://www.w3.org/TR/html5/single-page.html)推荐的写法:`<doctype html>`

* 上下margin重合问题
ie和ff都存在,相邻的两个div的margin-left和margin-right不会重合,但是margin-top和margin-bottom却会发生重合。
解决方法,养成良好的代码编写习惯,同时采用margin-top或者同时采用margin-bottom。
* ie6对png图片格式支持不好(引用一段脚本处理)


\subsection{减少加载时间}
 1.优化图片 
 
 2.图像格式的选择(GIF:提供的颜色较少,可用在一些对颜色要求不高的地方)
  
 3.优化CSS(压缩合并css,如margin-top,margin-left...) 
 
 4.网址后加斜杠(如www.campr.com/目录,会判断这个“目录是什么文件类型,或者是目录。) 
 
 5.标明高度和宽度(如果浏览器没有找到这两个参数,它需要一边下载图片一边计算大小,如果图片很多,浏览器需要不断地调整页面。这不但影响速度,也影响浏览体验。 
 
当浏览器知道了高度和宽度参数后,即使图片暂时无法显示,页面上也会腾出图片的空位,然后继续加载后面的内容。从而加载时间快了,浏览体验也更好了。) 

6.减少http请求(合并文件,合并图片)。


期待的解决方案包括:

 文件合并
 
 文件最小化/文件压缩
 
 使用 CDN 托管
 
 缓存的使用(多个域名来提供缓存)
 
 其他
 
 
 \subsection{你都使用哪些工具来测试代码的性能?}

Profiler, JSPerf(http://jsperf.com/nexttick-vs-setzerotimeout-vs-settimeout), Dromaeo



\subsection{HTTP缓存相关头}


\subsection{推送}

\subsection{Cookies}

\subsection{•}