\renewcommand\lstlistingname{Mocha}  
\chapter{JavaScript 语言}

面向对象,有两种实现面向对象的方式:基于类和基于原型的继承.

基于类的继承. 类是模板,继承的是行为。
	
基于原型的继承, 所有的都是对象(当然还有基本类型),继承的是状态和行为.JavaScript是基于原型的继承,这里只有对象,没有类的概念。

\begin{JavaScript}
var a = {};	
a.b = 10;
console.log(a.b); // 10
console.log(a["b"]); // 10
a.foo = function() {
	console.log("invoke foo");
};
a.foo();
\end{JavaScript}

JavasScript中对象的定义是: Object是Property的集合,property是一个值或者对象的引用. JavaScript对象是一组属性的集合,这些属性引用的是一个对象或者基本类型.
	
首先,第一行生成一个对象,并将它赋值给a。然后第二行,a.b = 10,这个时候,因为a没有b这个property,所以给他生成了一个b property,并将10复制给他,第三行,打印a.b,结果是10.第四行,对对象属性的访问有两种方式,.和[]。不同之处是.后面是直接跟标识符,而[]中是字符串,注意第四行的b是有引号的。
	
在在后面,将我们定义的函数赋值给foo property,这里又不同于Java,我们前面说了,JavaScript中一切皆对象,所以函数也是对象,他是一类比较特殊的对象,可以通过被调用。因为函数是对象,所以他也可以赋值给对象的属性。赋值之后,我们可以调用他。
	
\section{JavaScript的类型}

\subsection{基本类型}

JavaScript只有对象这种说法也不是太准确,JavaScript有基本类型和对象两类。只是在使用基本类型的时候,如果必要,基本类型会被自动转换为对象类型。

基本类型有Undefined, Null, Boolean, String和Number类型

\begin{tabular}{|r|l|}
\hline
类型 & 说明 \\
\hline
Undefined & 有且只有一个值,undefined。\\
\hline
Null & 有唯一的值null \\
\hline
Boolean & 两个值,true和false \\
\hline
String & 如 "abc" \\
\hline
Number & 只有浮点型表示。 NaN, Infinity, -Infinity \\
\hline
\end{tabular}

JavaScript中的undefined就是这个值,只是undefined不是关键字,而是一个全局变量。可能被赋其他值。所以可以使用void表达式来表示,如void 0 
\begin{JavaScript}
		it('void 0 is undefined', function(){
			should(void 0).be.exactly(undefined);
		});
\end{JavaScript}

\subsection{对象类型}

\begin{itemize}
\item 对象就是property的集合。
\item 给对象property赋值,如果存在,就修改值,如果不存在,就创建一个新的property,并赋值。
\item 函数也是对象,可以赋值给property。
\end{itemize}
	
对于property,我们可以先将他看成有两类property,一类就是我们上面看到的,实际上,他还要细分为data property 和 access property。上面和下面所说的也就是data property。我们平时写代码生成的也是data property类型。
	
再来有一类我们叫做internal property,他是在我们程序级别是看不到的,而是提供给语言内部实现级别所使用的。例如我们将要讨论的原型,每个对象都有原型属性,但是我们在程序中无法通过.的方式来访问他,我们在谈论到internal property的时候,会使用[[]]来表示,例如,原型属性我们会表示成[[prototype]]。
	
\subsection{Prototype}
	
js是通过原型的方式来实现继承的。原型实际上就是对象的一个internal property [[prototype]],每个对象都有[[prototype]]属性,他指向一个对象,而原型本身也有[[prototype]]属性,这样一直链接到最顶端的一个特殊对象,他的原型是空的(是null,不是undefined),所有的对象的原型链的顶端都是这个对象。这就是我们说的原型链。
\begin{JavaScript}[原型链的顶端是null]
		it('__proto__ of Object.prototype is null', function(){
			should(Object.prototype.__proto__).be.exactly(null);
		});
\end{JavaScript}
在浏览器中,我们可以通过属性名\lstinline!__proto__!来引用内部属性[[prototype]]。
	
然后我们说说原型的作用,他就是帮我们来解析.和[]取得什么样的值的。当我们通过.的方式来读取一个对象的property的时候,会在当前对象中查找看看是否有这个property,如果有,就返回,如果没有,就会尝试着在他的[[prototype]]上来查找,如果找到了就返回,如果没有找到,就继续在这个链往上找,直到找到最顶,如果还没有找到,那就会返回undefined。
	
	
\begin{JavaScript}
var proto = {bar : 10};
var foo = Object.create(proto);

console.log(foo.bar); // 10
console.log(foo.bar2); // undefined

proto.bar2 = 20; 
console.log(foo.bar2); // 20
\end{JavaScript}
	
这里的Object.create函数是用来创建一个对象,他的原型是传入的参数。在第四行,因为在原型对象上定义了bar,所以可以取到值10,而bar2没有定义,所以得不到值,返回undefined。然后在第七行,我们给proto的bar2赋值20,这样当我们再次调用foo.bar2的时候,在原型链上查找,可以找到值20了。
	
	
这里说到的是取值,实现了继承。接着是赋值,赋值很简单,如果存在就修改值,不存在就新增这个属性,这里所说的存在与不存在是指当前对象,而不是只原型链上存不存在。
	
\begin{JavaScript}
var proto = {x: 10};

var foo = Object.create(proto);
var bar = Object.create(proto);

console.log(foo.x); // 10
console.log(bar.x); // 10

foo.x = 20;

console.log(foo.x); // 20
console.log(bar.x); // 10
console.log(proto.x); // 10
\end{JavaScript}
		
这里可以看到foo和bar都是继承proto,所以x都等于10,而在foo.x = 20之后,foo上创建了x,并且赋值20。所以当取foo.x是等于20。而bar.x和proto.x还是10。

\subsection{Property}

\subsubsection{Data Property 和 Access Property}

从逻辑上来讲,Object是一系列property的集合,其中property可能是data property或者access property。
\begin{itemize}
\item data property就是一个key和一个ECMAScript语言类型的值和一系列boolean型的Attribute组成。
\item access property就是一个key和一个或者两个accessor function和一系列boolean型的Attribute组成。
\end{itemize}	
这里的key,要么是一个string,要么是一个symbol(ECMAScript6)。

ECMAScript的使用key来标示Property,有两中方式来访问Property。get和set。

\subsubsection{Attribute}
Attribute则是规范用来定义和解释Property的。

Data Property的Attribute

\begin{tabular}{|l|l|l|}
\hline
属性名 & 属性值 & 解释 \\
\hline
\lstinline![[Value]]! & 任意ECMAScript语言类型 & 属性的值 \\
\hline
\lstinline![[Writable]]! & Boolean & 如果是false,则通过[[Set]]来设定[[Value]]不会成功。 \\
\hline
\lstinline![[Enumerable]]! & Boolean & 如果是true,则可以在for-in中被迭代。 \\
\hline
\lstinline![[Configurable]]! & Boolean & \parbox[t]{8cm}{如果是false,则删除,从Data转换成accessor Property,修改除\lstinline![[Value]]! 之外的值都会失败}\\
\hline
\end{tabular}

\begin{JavaScript}[Data Property中Attribute的默认值]
		it(' should return correct default descriptor', function(){
			var foo  = Object.create({}, {
				bar: {}
			});
			var descriptor = Object.getOwnPropertyDescriptor(foo, 'bar');
			descriptor.should.eql(
				{
					value: undefined, 
					writable: false, 
					enumerable: false, 
					configurable: false
				});
		});
\end{JavaScript}

\begin{JavaScript}[writable为false时,不能修改Value]
		it(' should not able to change property value when writable is false', function(){
			var foo = Object.create({}, {
				bar: {value: 100, writable: false}
			})

			foo.bar = 200;
			foo.bar.should.be.exactly(100).and.be.Number;
		});
\end{JavaScript}

\begin{JavaScript}[enumerable为false时,key不会再for in语句中出现]
		it(' should not be able to enumerate in for in statement when enumerable is false', function(){
			var foo = Object.create({}, {
				bar: {value: 100, enumerable: false}
			});

			for (var key in foo)
			{
				key.should.not.be.exactly('bar');
			}
		});
\end{JavaScript}

\begin{JavaScript}[configurable为false时,不能修改除Value之外的Attributes]
		it(' should not be able to modify attributes of property excluded [[Value]] when configurable is false', function(){
			var foo = Object.create({}, {
				bar: {value: 100, writable: true, enumerable: false, configurable: false}
			});

			var descriptor = {value: 200, writable: false, enumerable: true, configurable: true};
			(function(){
				Object.defineProperty(foo, 'bar', descriptor);
			}).should.throw();

			foo.bar = 200;
			foo.bar.should.be.exactly(200).and.be.Number;			
		});
\end{JavaScript}

\begin{tabular}{|l|l|l|}
\hline
属性名 & 属性值 & 解释 \\
\hline
\lstinline![[Get]]! & & \\
\hline
\lstinline![[Set]]! & & \\
\hline
\lstinline![[Enumerable]]! & & \\
\hline
\lstinline![[Configurable]]! & Boolean & \\
\hline
\end{tabular}

\begin{JavaScript}[Accessor Property的Attributs的默认值]
		it('should get correct default descriptor', function(){
			var foo = Object.create({}, {
				bar : {get: undefined}
			});
			var descriptor = Object.getOwnPropertyDescriptor(foo, 'bar');
			descriptor.should.eql({
				get: undefined,
				set: undefined,
				enumerable: false,
				configurable: false
			});
		});
\end{JavaScript}

\begin{JavaScript}[对于Accessor Property,接受请求的对象会被当做this]
		it('base should be treat as this when assign or get property value', function(){
			var foo = Object.create({}, {
				bar: {get: function(){
					return this._bar;
				}, 
				set: function(value){
					this._bar = value;
				}}
			});
			foo.bar = 100;
			foo.should.have.ownProperty('_bar').be.exactly(100);
			foo.should.have.ownProperty('bar').be.exactly(100);
		});
\end{JavaScript}

\subsubsection{对Property进行赋值操作}
这里稍微比前面说如果对象上不存在某个被赋值的属性,就创建会要复杂点。对于对象上已经存在这个Property的情况,赋值的赋值,调用set方法的调用set方法。

而对于Property在对象上不存在的,可以分三种情况处理:
\begin{itemize}
\item 如果这个Property在原型链上不存在的话,则会创建一个Data Property
\begin{JavaScript}[Property在原型链上也不存在的话]
		it('should create a own data property when assignning a not exist property', function() {
			foo = Object.create({}, {});
			foo.bar = 100;
			foo.should.have.ownProperty('bar').be.exactly(100);
			var descriptor = Object.getOwnPropertyDescriptor(foo, 'bar');
			descriptor.should.eql({
				value: 100,
				writable: true,
				enumerable: true,
				configurable: true
			})
		});
\end{JavaScript}
\item 如果这个Property在原型链上存在一个Accessor Property的话,直接调用这个Accessor Property的\lstinline![[Set]]!方法,reciver会被当做this来传给这个方法。
\begin{JavaScript}[原型链上存在Accessor Property的情况]
		it('should not create own propert when assignning a inherited accessor property', function(){
			var x = 100
			var parent= {};
			Object.defineProperty(parent, 'bar', {set: function(value){x = value}});

			foo = Object.create(parent, {});

			foo.should.not.have.ownProperty('bar');
			foo.bar = 200;
			foo.should.not.have.ownProperty('bar');
			x.should.be.exactly(200);
		});
\end{JavaScript}

\item 如果原型链上存在一个Data Property的话,还是直接在当前对象上创建一个Data Property.
\begin{JavaScript}[原型链上存在Data Property的情况]
		it('should create a own data property when assignning a inherited data property', function(){
			foo = Object.create({bar: 200}, {});

			foo.should.not.have.ownProperty('bar');
			foo.bar = 100;
			foo.should.have.ownProperty('bar').be.exactly(100);
		});
\end{JavaScript}
\end{itemize}
	
\subsection{== 和 ===}

\begin{itemize}
\item ToNumber

\begin{tabular}{|l|l|}
\hline
参数 & 值 \\
\hline
Undefined & NaN \\
\hline
Null & 0 \\
\hline
Boolean & true转换为1,false转换为0 \\
\hline
String & 将string转换为数字 \\
\hline
Object & 先转换为基本类型,然后再按上面的转换 \\
\hline
\end{tabular}

\item PrimtiveValue

根据参数Hint来判断怎么执行,如果Hint是number,
对象转换为基本类型看valueOf是不是返回基本类型,不是再看看toString是不是返回基本类型,如果还不是则抛错。

如果Hint是String,
对象转换为基本类型看toString是不是返回基本类型,不是再看看valueOf啊、是不是返回基本类型,如果还不是则抛错。

如果不带Hint,则按Hint是number来执行。
\end{itemize}

\begin{itemize}
\item == 

\begin{enumerate}
\item 对左右引用类型求值,设左边是lval,右边是rval;
\item 如果lval和rval是相同类型的话。
	\begin{itemize}
	\item 基本类型的话,按正常的相等性进行比较.
	\item NaN不等于其他值.
	\item 不是基本类型时,如果不是引用同一个对象,则两者不相等。
	\end{itemize}
\item 当lval和rval是不同类型的时候
	\begin{itemize}
	\item undefined等于null。
	\item undefined和null不等于其他值
	\item 如果有boolean型,则将其转换为number,再比较
	\item 如果有一方是对象,一方是数字或者字符串,将对象转换为基本类型再比较
	\item 如果是数字和字符串比较,转换为数字在比较
	\item 其他情况一律不相等
	\end{itemize}
\end{enumerate}

\item ===
如果是不同类型,直接返回false,如果是相同类型,比较结果和 == 一样。
\end{itemize}

\begin{JavaScript}
		it('test between same type', function(){
			should(undefined == undefined).be.ok;
			should(null == null).be.ok;
			should(NaN == 1).not.be.ok;
			should(NaN == NaN).not.be.ok;
			should({} == {}).not.be.ok;
		});
\end{JavaScript}
对于相同类型的值得比较,NaN不和自己本身以及任何数相等。两个不同的对象是不相等的。

\begin{JavaScript}
		it('test between different type', function(){
			should(undefined == null).be.ok;
			should(0 == undefined).not.be.ok;
			should(0 == null).not.be.ok;
			should(2 == true).not.be.ok;
			should(1 == true).be.ok;
			should(0 == false).be.ok;

			should(1 == '1').be.ok;

			var a = {
				toString: function() {return "a"},
				valueOf: function(){return 1;}
			}
			should(a == 1).be.ok;
			should(a == "a").not.be.ok;

			should(true == a).be.ok;
		});
\end{JavaScript}
\begin{JavaScript}
		it('should throw exception', function(){
			(function(){
				var a = {toString: undefined, valueOf: undefined};
				a == 1;
			}).should.throw();
			// 先调用valueOf,再调用toString(一般来说), 
			// 如果都不是函数, 或者返回的都不是基本类型,报错
			(function(){
				var a = {toString: function(){ return {}}, valueOf: function(){return {}}};
				a == 1;
			}).should.throw();
		});
\end{JavaScript}
对于不同的类型,undefined和null是相等的,因为boolean的转化造成基本类型是1和0,所以2是不会等于boolean型的。对于有字符串或者数字参与的比较,需要将对象转变为基本类型。如果和valueOf和toString都不存在,则会在转换的过程抛出错误。

\subsection{typeof 和 instanceof}
typeof就是返回后面所跟引用的类型,这个值是固定的。

\begin{tabular}{|r|l|}
		\hline
		类型	& 	结果\\
		\hline
		Undefined 	& 	"undefined" \\
		\hline
		Null		&	"object" \\
		\hline
		Boolean		&	"boolean" \\
		\hline
		Number		&	"number" \\
		\hline
		String		&	"string" \\
		\hline
		Host object & Implementation-dependent \\
		\hline
		Function object & "function"\\
		\hline
		Any other object &	"object"\\
		\hline
\end{tabular}

\subsubsection{typeof的运算过程,为什么不会抛出异常?}

instanceof运算符用来检测constructor.prototype是否存在于参数object的原型链上.
\begin{JavaScript}
		it('test if Constructor.prototype on the prototype chain', function(){
			function Foo(){}

			var foo = Object.create(Foo.prototype);

			should(foo instanceof Foo).be.exactly(true);

			var foo2 = Object.create(Foo);

			should(foo2 instanceof Foo).be.exactly(false);
		});
\end{JavaScript}
	
\subsection{ECMA spec中和本节相关概念}
\section{关于函数部分的准备知识}
\subsection{Lexical Environment}

Lexical Environment是规范中使用定义变量和函数标识符的关系结构。Lexical Environement由Environment Record和一个指向外部Lexical Environment的引用(可空)。

Environment Record有两种类型,一种是Declarative Environment Record,他不提供this值,直接返回返回undefined。

一种是Object Environment Record。 Declarative Environment Record用于定义标识符,而Object Environment将他的binding对象和标示符绑定,标识符会被绑定到binding对象的对应Property上,如果自身的provideThis是true,绑定的object会被当做this值来提供,provideThis的默认值是false。

Global Environment是一个唯一的Lexical Environment,在代码执行之前被创建,他的Environment Record是一个Object Environment Record,这个Environment Record的绑定对象是Global Object,他的outer lexical Environment是null。


Execution Context,进入一段代码,就进入了一个execution context。execution context是栈结构,最顶上的execution context是running execution context.所有的操作都是对running execution context的操作。

execution context有三部分,
\begin{itemize}
\item 一个是ThisBinding,用来保存this这个keyword引用的对象;
\item lexicalEnvironment,用来解析标识符引用的,这个在执行过程中引用的lexicalEnvironment可能是会改变的;
\item variableEnvironment,用来保存执行过程中定义的函数和变量,这个在执行过程中是不会改变的。
\end{itemize}
在创建execution context时,lexicalEnvironment和variableEnvironment是引用同一个值。

\subsubsection{什么时候lexicalEnvironment会变化?}
\paragraph{with statement}
with statement会创建一个新ObjectEnvironment,绑定object,outer为原来的lexicalEnvironment。同时将objectEnvironment的bingThis设为true,表示提供this,this就是object,用于调用function的时候提供this。
\begin{JavaScript}
		it('should been add to with variable environment of running execution context', function(){
			var foo = {};

			(function(){
				bar;
			}).should.throw();
			
			with(foo){
				eval('var bar = 200');
			}
			bar.should.be.exactly(200);
			foo.should.eql({});
		});
\end{JavaScript}

此时lexicalEnvironment的envRec添加了一个object Environment record,但是variableEnvironment没有变,所以bar任然是添加到running execution context的variableEnvironment上。等出了with的范围,lexicalEnvironment又还原成和variableEnvironment为同一个对象。
\begin{JavaScript}
		it('should been add to with variable environment of running execution context', function(){
			var foo = {};
			should(bar).be.exactly(undefined);
			with(foo){
				var bar = 200;
			}
			bar.should.be.exactly(200);
			foo.should.eql({});
		});
\end{JavaScript}

\paragraph{catch statment}
catch(e)

会创建一个DeclarativeEnvironment(不同于with,with创建的是ObjectEnvironment), 将他的outer指向running execution context的lexicalEnvironment,将传入的对象绑定到e上。然后将新的DeclarativeEnvironment当做lexicalEnvironment。

返回执行结果,将running execution context的lexicalEnvironment回复成原来的那个。
\subsection{解析标识符}
\paragraph{reference Type} 是用来在规范中说明算法使用的,reference Type由三部分组成,base,name,strict。
对于标识符,base是标识符引用的对象真正所在的那个Environment record;name就是标识符名字,strict看看是否是strict mode。

GetIdentifierReference (lex, name, strict)算法
\begin{enumerate}
\item 如果lex是null, return {base: undefined, name: name, strict: strict}(unresolveReference)
\item let envRec = lex.environmentRecord
\item 如果envRec上有绑定name,return {base: envRec, name: name, strict: strict}
\item 否则let lex = lex.outer, call GetIdentifierReference (lex, name, strict)
\end{enumerate}


\subsection{建立Execution Context}
\paragraph{global execution context}
创建:
\begin{itemize}
\item this: global object
\item lexicalEnvironment: global environment
\item variableEnvironment: global environment
\end{itemize}
\paragraph{eval}
创建:

如果没有calling context,步骤同上;
否则
\begin{itemize}
\item this: this of calling context
\item lexicalEnvironment: lexicalEnvironment of calling context
\item variableEnvironment: variableEnvironment of calling context
\end{itemize}
\begin{JavaScript}
		it('should been add to environment of running execution context', function(){
			(function(){
				eval('var bar = 100');
				bar.should.be.exactly(100);
			})();
		});
\end{JavaScript}
如果是在strict模式下(调用代码或者eval代码是strict模式),生成新的DeclarativeEnvironment,outer environment指向上面定义的lexicalEnvironment。将lexicalEnvironment和variableEnvironment设定为新生成的DeclarativeEnvironment。
\begin{JavaScript}
		it('should use new declarative environment in strict mode', function(){
			(function(){
				'use strict';
				var bar = 200;
				eval('var bar = 100');
				bar.should.be.exactly(200);
			})();
		});

		it('should use new declarative environment in strict mode 2', function(){
			(function(){
				var bar = 200;
				eval('"use strict"; var bar = 100');
				bar.should.be.exactly(200);
			})();
		});
\end{JavaScript}
\paragraph{function}
创建:

调用的代码会传入this和参数。对于strict code,this就等于传入的this。

否则,如果传入的this是null或者undefined,this设定为global object。
如果this是基本类型,就转换为object。

新建DeclarativeEnvironment,将function的\lstinline![[Scope]]!作为outer Environment。将execution context的lexicalEnvironment和variableEnvironment设定为新生成的DeclarativeEnvironment。

之后调用\lstinline![[Code]]!的代码。

\begin{JavaScript}[this是object]
		it('this should be object', function(){
			function foo(){
				should(typeof this).be.exactly('object');
			}
			foo.call(1);
		});
\end{JavaScript}

\begin{JavaScript}[对于undefined this会被指定为Global object]
		it('this should be global object if null or undefined been treat as this', function(){
			function foo(){
				this.should.be.exactly(global);
			}

			foo.call(null);
			foo.call(undefined);
		});

		it('this should be global object if null or undefined been treat as this', function(){
			function foo(){
				this.should.be.exactly(global);
			}

			foo();
		});
\end{JavaScript}

当标识符解析时,lexicalEnvironment不会生成this,因为实现是中对于Lexical Environment Record的返回的this是undefined。

\section{函数}
函数也是对象,可以使用对象的地方就可以使用函数。

\subsection{函数申明和函数表达式}	
通过函数声明和函数表达是来创建Function Object,

两者不同之处在于函数声明白创建的function object的\lstinline![[Scope]]!保存的是running execution context的variableEnvironment,

本来我想写个例子,但是感觉实现不是这么实现的。下面可以访问到处于lexicalEnvironment链上的bar
\begin{JavaScript}
		it('does not work as expect', function(){
			var foo = {bar: 200};
			with(foo) {
				eval('function func(){return bar;}');
			}
			func().should.be.exactly(200);
		});
\end{JavaScript}
看上去,下面的代码可以说明,但是实际上function declaration是在执行代码之前就已经执行了。
\begin{JavaScript}
		it('[[Scope]] should bind with variableEnvironment', function(){
			// cannot prove
			var foo = {bar: 100};

			with(foo) {
				function func() {
					(function(){
						bar;
					}).should.throw();
				}
			}
			func();
		});
\end{JavaScript}

匿名函数表达式创建的function object的\lstinline![[Scope]]!保存的是running execution context的lexicalEnvironment。
\begin{JavaScript}
		it('[[Scope]] should bind with lexicalEnvironment', function(){

			try {
				throw {bar: 100};
			} catch(e) {
				var foo = function(){
					return e;
				}
			}

			should(foo()).be.eql({bar: 100});
		});
\end{JavaScript}

带名字的函数表达式创建的一个DeclarativeEnvironment,将running execution context的lexicalEnvironment,同时在新创建的DeclarativeEnvironment上绑定这个function object,将这个新创建DeclarativeEnvironment保存在function object的\lstinline![[Scope]]!上。(记得此处IE不是这么实现的)。

\begin{JavaScript}
		it('[[Scope]] should bind with a new DeclarativeEnvironment which outer is lexicalEnvironment of running execution context', function(){

			try {
				throw {bar: 100};
			} catch(e) {
				var foo = function e() {
					return e;
				}
				e.should.eql({bar: 100});
			}
			should(foo()).be.exactly(foo);
		});	
\end{JavaScript}


\subsection{函数对象的创建过程}
\begin{enumerate}
\item 创建ECMAScript Object;
\item \lstinline![[Class]]!设定为"function"
\item \lstinline![[Prototype]]!(可以通过\lstinline!__proto__!来访问)设定为Function.prototype。
\item \lstinline![[Scope]]!设定为上面所描述的。
\item 定义length属性
\item 定义prototype属性,为object对象,prototype上定义constructor属性
\end{enumerate}
\begin{JavaScript}[此处是函数表达式]
		it('[[Class]] is set to function', function(){
			(typeof function(){}).should.be.exactly('function');
		});
\end{JavaScript}

\begin{JavaScript}[所以 func instanceof Function 会返回true]
		it('__proto__ is set to Function.prototype', function(){
			(function(){}).__proto__.should.be.exactly(Function.prototype);
		});
\end{JavaScript}

\begin{JavaScript}
		it('length is the number of parameters', function(){
			(function(a, b, c, d, e, f){}).length.should.be.exactly(6);
		});
\end{JavaScript}

\begin{JavaScript}
		it('default prototype should be an object with constructor property', function(){
			function foo(){}
			foo.prototype.should.eql({constructor: foo});
		});
\end{JavaScript}

\subsection{函数作为构造函数}
使用new关键字来调用,对于无参数的构造函数Foo,new Foo()和new Foo是相同的效果。	
\begin{enumerate}
\item 创建对象;
\item \lstinline![[Class]]!设定为object;
\item 取得构造函数的prototype,如果他不是object,则取得Object.prototype,将他赋值给\lstinline![[Prototype]]!
\item 将创建的对象作为this来调用构造函数,如果返回值不是object,返回创建的对象,如果返回的是object,返回返回的对象。
\end{enumerate}
\begin{JavaScript}[作为构造函数生成对象的过程]
		it('steps for new operator', function(){
			function Foo(){
				this.bar = 200;
			}

			var newObj;
			var foo = {};
			foo.__proto__ = Foo.prototype;
			var result = Foo.call(foo);
			if(result === undefined || result === null || typeof result == 'number' || typeof result =='string' || typeof result == 'boolean') {
				newObj = foo;
			} else {
				newObj = result;
			}
\end{JavaScript}

\begin{JavaScript}[prototype是基本类型时,会使用Object.prototype代替]
		it('__proto__ is set to Object.prototyp if constructor.prototype is primitive or null', function(){
			function Foo(){}
			Foo.prototype = 1;

			var foo = new Foo;
			foo.__proto__.should.be.exactly(Object.prototype);
		});
\end{JavaScript}

\subsection{调用函数的过程}
\subsubsection{函数被调用,关注this的值}

这里先会evaluate表达式,得到一个ref,

当ref不是reference type的时候,this被设定为undefined(我的感觉比如匿名函数表达式,返回的就是一个function对象。)

当ref是reference type的时候,解析的reference type的的base是对象,这个对象就被当做this。

当reference type的base是Environment Record,则看看他是不是提供this值,如果提供就传入,否则传入undefined

当reference type的base是undefined时,this传入undefined(在函数执行过程中,undefined又会被global object代替)。

然后按上面提到的方式来创建execution context。

说起来,this有这样几种取值:
\begin{itemize}
\item 使用expresson[expression]或者identifies.identifies的格式来调用函数的时候,this是前面那部分的值;
\item 使用变量名的方式来调用的时候,this就是指向global object;
\item func.call,func.apply的方式调用的时候,第一个参数指定this,如果这个时候传入的是undefined或者null,则this是global object,如果传入的是基本类型,则this是对应的包装类型
\item 在with block中,使用变量名的方式来调用,如果这个变量是绑定在with传入的对象上时,this就是with传入的对象。因为:此处的LexicalEnvironment被替换成Object Environment Record,绑定了foo。bar得到的reference type是{base:objEnv(foo), name:bar, strict:false}。objEnv提供的this是foo,所以调用的时候,this被绑定为foo
\end{itemize}

\begin{JavaScript}
	describe("#this", function(){
		it('dot expression or square bracket', function(){
			var foo = {
				bar: function(){ return this;}
			}
			foo.bar().should.be.exactly(foo);
			foo['bar']().should.be.exactly(foo);
		});

		it('variable', function(){
			var foo = function(){return this;};
			foo().should.be.exactly(global);
		});

		it('variable in with', function(){
			var foo = {
				bar: function(){return this;}
			}
			with(foo){
				bar().should.be.exactly(foo);
			}
		});

		it('call, apply', function(){
			function foo(){return this;}
			var bar = {};
			
			foo.call(bar).should.be.exactly(bar);
			foo.apply(bar).should.be.exactly(bar);

			var res = foo.call(1);
			should(res).be.Object;
			(res == 1).should.be.true;

			foo.call(undefined).should.be.exactly(global);
		});
	});
\end{JavaScript}


\subsubsection{函数代码执行}
下面的步骤是将标示符绑定到当前的execution context的VariableEnvironment上。
\begin{enumerate}
\item 先创建arguments对象,同时在variableEnvironment上定义参数列表中的参数,参数列表中如果有重复的变量名,后面的会覆盖前面的值;
\item 绑定函数申明,后声明的覆盖前面申明的函数;
\item 如果没有定义arguments的函数,绑定第一步创建的arguments对象;
\item 绑定不存在的变量申明,值为undefined。
\end{enumerate}

\begin{JavaScript}
	describe('#arguments', function(){
		it('last should override first if the has same name', function(){
			function foo(a, b, c, a){return a;}
			foo(1,2,3,4).should.be.exactly(4);
		});

		it('arguments should be replace by function', function(){
			function foo(){
				arguments.should.be.type('function');
				function arguments(){}
				var arguments = {};
				arguments.should.eql({});
			}
			foo();
		});

		it('arguments should exist', function(){
			function foo(){
				arguments.should.be.arguments;
				var arguments= {};
				arguments.should.eql({});
			}
			foo();
		});

		it('should be', function(){
			function foo(){

				bar.should.be.type('function');

				var bar = 10;

				bar.should.be.exactly(10);

				function bar(){}

				bar.should.be.exactly(10);

			}

			foo();
		});

		it('should be', function(){
			function foo() {
				should(bar).be.exactly(undefined);
				var bar = 10;
				bar.should.be.exactly(10);
			}

			foo();
		});
	});
\end{JavaScript}
\subsubsection{变量的赋值}
通过GetIdentifierReference可以看到,找不到变量定义,reference type的base就是undefined,这个时候就会在global object上定义一个对应的property。

