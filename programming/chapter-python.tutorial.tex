\chapter{Python Tutorial}
在我了解了python的基本语法,如缩进来控制程序等之后,
\begin{itemize}
\item 我觉得第一个需要掌握的就是Python提供的一些基本的数据结构。这样方便熟练的写出高效的程序。
\item 然后我觉得就是基本类库了,比如一些系统的属性什么的,文件操作等等。
\item 在这之后,我觉得应该就是整个生态环境的了解,如pip等工具。也许需要python2和python3共存。
\end{itemize}
所以笔记也会用这样的方式来组成。在这个之后,我会最后记录一下其他一些语法知识。然后就是用python来实现那两本用Ruby写的书,Unix和Scoket相关的东西。

最后就是cocos2d python的学习了。
\section{Unit Test}
在开始正式的编程之前,我希望掌握一个合适的单元测试框架。首先我们要安装pip。这是一个包管理工具,这里我们只介绍怎么安装,之后介绍python相关工具时在做具体介绍。很简单,下载get-pip.py。然后运行他。

然后,这里学习一下python自带的unitest。这里我假设自己已经对单元测试很熟习了,如果想回顾一下单元测试,我会再写一篇关于单元测试的文章,他会基于Junit来说明,因为我之前花了一些力气来熟悉Junit3.8,它包括很多基本的概念和一些设计模式的知识,还有测试驱动开发的内容。

\begin{Python}
import random
import unittest

class TestSequenceFunctions(unittest.TestCase):

    def setUp(self):
        self.seq = list(range(10))

    def test_shuffle(self):
        # make sure the shuffled sequence does not lose any elements
        random.shuffle(self.seq)
        self.seq.sort()
        self.assertEqual(self.seq, list(range(10)))

        # should raise an exception for an immutable sequence
        self.assertRaises(TypeError, random.shuffle, (1,2,3))

    def test_choice(self):
        element = random.choice(self.seq)
        self.assertTrue(element in self.seq)

    def test_sample(self):
        with self.assertRaises(ValueError):
            random.sample(self.seq, 20)
        for element in random.sample(self.seq, 5):
            self.assertTrue(element in self.seq)

if __name__ == '__main__':
    unittest.main()
\end{Python}

我们可以通过python testModule来执行case,这里,是通过unittest.main()来触发。
也可以通过命令行 python -m unittest来执行,可以具体到某个case
\begin{itemize}
\item python -m unittest testModule
\item python -m unittest testModule.testClass
\item python -m unittest testModule.testClass.testMethod
\end{itemize}

也可以通过python -m unittest discover.所有的测试文件必须是在当前项目可以导入到顶层目录的module或者package。
而且因为默认情况下 discover查找的文件名是test*.py,所以保证test文件是以test开头的。

这里以我目前对python的肤浅理解是,文件名是标识符,而下一层目录必须包含文件\_\_init\_\_.py,这样python才会将这个目录当做package,而将这个目录里的module当做是submodule。

在这个简单的了解之后,开始我们的第一步,对数据结构的认识。


\section{Data Structure}
其实作为高级语言,也没有什么特别多的数据类型。因为我有Java基础,所以一下就想到有List,Set,Map这三种容器集合,这几个基本可以满足所有需求了。python还多一个元组tuple和range。而且python没有数组。

这就是python的基本数据结构。

\subsection{Sequence}
range, list, tuple都是sequence类型,也就是可以通过索引来找到对应的元素,string也是。sequence类型有很多方便的操作可以使用,如通过s[i]来找到对应位置的元素,通过s[i:j]来复制一份浅拷贝的子sequence。

\begin{itemize}
\item x in s, x not in s

判断x是或不是s的一个元素。

\item s + t, s*n

连接s和t,n个s的浅拷贝的连接。

\item s[i]

选择s中的第i个元素,从0开始。

\item s[i:j]

当i或者j是负值的时候,他们代表的index是 len(s)+ i or len(s) + j。但是-0还是表示0。


如果i或者j大于len(s),取值len(s)。如果i不存在,取0,如果j不存在,取len(s)。

如果i大于或等于j,返回空sequence。

返回顺序的k集合,所有的k有i <= k < j。

\item s[i:j:k] 

k不能为0。

从i到j,以k为步长的 x = i + n*k, 其中 0 <= n < (j - i)/k

如果i或者j大于len(s), 取len(s)

如果i或者j不存在,则根据k来取值结束的值,我推测是如果k > 0,i就取0,如果 k < 0,i就取len(s)

\item lens(s), min(s), max(s)

s的长度,s中的最小,做大元素的值

\item s.index(x, [i, [j]])

找到s中第一个x的位置,如果有i,表示从位置i开始,如果有j,表示到位置j之前结束。

\item s.count(x)

统计s中x的个数。

\end{itemize}

对于不变的sequence,他们实现了hash这个内建方法,这样,他们就可以作为dict类型的key值,set的值。

对于可以变的sequence,有下面一些特定操作:
\begin{itemize}
\item s[i] = x

\item s[i:j] = t

使用t替换掉从i到j之前的切片

\item del s[i:j]

删除切片

\item s[i:j:k] = t

这个时候左边和右边元素的个数必须相等

\item del s[i:j:k]

\item s.append(x), s.pop([i]), s.insert(i, x), s.remove(x), s.clear(), s.copy(), s.reverse(), s.exend(t)

其中,s.pop([i])中,i的默认值是-1,等于是弹出最后一个元素。

s.remove(x),如果x不存在,会报ValueError

s.reverse()会改变原来的sequence

\end{itemize}


\subsection{List}
class list([iterable])

几种常用来生成list的方法:
\begin{itemize}
\item [], [1,2,3,4]
\item [x for x in iterable]
\item list(), list(iterable)
\end{itemize}

除了上面的方法,list实现了内建的sort()方法。


\subsection{Tuple}
class tuple([iterable])

tuple是不可变的sequence类型,几种常用来生成tuple的方法:
\begin{itemize}
\item ()
\item a, or (a,)
\item a, b, c or (a, b, c)
\item tuple() or tuple(iterable)
\end{itemize}
\subsection{Range}
class range(stop)

class range(start, stop[, step])

我却傻傻的想着有两个构造函数,而且第一个参数不一样,想查出来怎么实现两个\_\_init\_\_方法,难道python有重载。这其实不就是js里常用的技巧么,判断一下第二个参数是不是等于默认值None,如果等于默认值,就将第一个参数的值赋给第二个参数。

为了证实一下我的想法,我写了个case。 结果证实我的猜想是错误的!!!!!!!


当step是正数时,range的取值为 r[i] = start + i*step, i >= 0, r[i] < stop

当setp是负数时,range的取值为 r[i] = start + i*step, i>=0, r[i] > stop

如果r[0]的取值不符合上面的条件,则为空range。

range只使用固定大小的memory,因为他只保存start,stop,step。在需要的时候计算单独的值或者子range。

\subsection{Set}
set和Java中的set有相同的作用,无序,不包含相同,采用hash来定位是否包含需要的元素.

有两个内建的set,set和frozenset。set是可变的,frozenset是不可变的。


\subsection{Map Type}
dict类型,类似Java的Map。

\subsection{String}
String是sequence类型,所以一般的sequence操作也适用于String

Python没有char类型。 使用index来取得的元素是长度为1的string

string是不可变类型,为了高效处理string,可以使用str.join(iterable),或者io.StringIO



\section{Standard Lib}

\subsection{Path}


\section{Class}
\subsection{Scopes And Namespaces}
namespace是从名字到object的映射。Python中大多数的namespace是通过Python的dictionaries来实现的。

在Module中定义的全局名字对应module对象的attribute。这里的attribute是"module."后面所接的名字。

对于module的attribute我们可以赋值,也可以用del关键词删除掉他。

Namespace的生命周期
\begin{itemize}
\item 对于built-in名字,在Python解释器启动的时候创建。
\item 对于module的全局名字,在module读入的时候创建。
\item 顶层调用语句定义的名字,都定义在在module "\_\_name\_\_"中。
\item 函数中的局部namespace在函数调用时创建。
\end{itemize}

scope
\begin{itemize}
\item 最里面的一层scope
\item 包裹这层最里层的scope的函数的scope
\item module的全局名字
\item built-in名字
\end{itemize}

\section{Protocol}

\section{Concurrency}
\subsection{Thread}

\subsection{Process}

