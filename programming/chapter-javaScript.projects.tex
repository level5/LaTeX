\chapter{练习项目}

  \section{NuoNuo Picture Gallary}

  初步的设想是开发一个图片管理的website,包括导入图片,给图片加标注,加标签。按日期分类,安标签分类,浏览图片。

  服务器采用NodeJs和Express框架。数据库采用MongoDB.前端框架使用AngularJs,然后需要使用这个地方可以考虑是不是需要使用requireJs之类的加载。

  然后可以练习的包括js加载的优化。

  最先要考虑的是图片保存的策略,这里先采用元数据存储在DB中,查询返回所需要图片的URL,然后浏览器从对应的URL来读取图片。。

  首先,我们需要建立一个angularJS的项目,采用网上的Angular-seed,貌似这是一个为开发搭建的框架,包括了单元测试,e2e测试,包得依赖。目前已经通过angularJS的教程,初步学习了怎么开发AngularJS应用,接着,要学习一下怎么来组织AngularJS的代码。


  第一篇文章,怎么组织AngularJS的代码。

  1. 不要开发大的App,写小的,模块化的,将他们组合起来形成大的应用。

