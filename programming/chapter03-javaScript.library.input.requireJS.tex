\section{Require.js}
    Require.js的目的是为了让我门编写模块化的代码,他遵循AMD规范。

    \subsection{BaseUrl}
    Require.js根据BaseUrl来加载所有的js文件。

    \begin{lstlisting}
<!--This sets the baseUrl to the "scripts" directory, and
    loads a script that will have a module ID of 'main'-->
<script data-main="scripts/main.js" src="scripts/require.js"></script>
    \end{lstlisting}

    对于BaseUrl,一般来说,他是data-main属性引用的文件所在的目录,在上面的例子中,目录为scripts。
    同时,我们也可以在config中显示的指定,如果我们在config中指定了,将不再使用data-main属性的目录了。
    如果既没有main-data属性,也没有在config中指定,那么BaseUrl将是当前运行require.js的HTML文件所在的目录。

    \subsection{BaseUrl + path}
    一般情况下,对js文件的查找,是通过BaseUrl + path的方式来查找。在下面的配置中,如果我们查找some/module.
    require.js会查找/another/path/some/v1.0/module.js。
    \begin{lstlisting}
<script src="scripts/require.js"></script>
<script>
  require.config({
    baseUrl: "/another/path",
    paths: {
        "some": "some/v1.0"
    },
    waitSeconds: 15
  });
  require( ["some/module", "my/module", "a.js", "b.js"],
    function(someModule,    myModule) {
        //This function will be called when all the dependencies
        //listed above are loaded. Note that this function could
        //be called before the page is loaded.
        //This callback is optional.
    }
  );
</script>
    \end{lstlisting}

    而在下面的三种情况下,Require.js不会采用BaseUrl + path的方式来查找js文件。
    \begin{enumerate}
    \item 以".js"结束的;
    \item 以"/"开头的;
    \item 以协议头,如"http:"开头的。
    \end{enumerate}

    \subsection{模块定义}

    模块应该遵循js的模块模式[先省略,日后补充][http://www.adequatelygood.com/JavaScript-Module-Pattern-In-Depth.html]

    简单的键值对,没有依赖,那么将键值对对象作为参数传递给define()
    \begin{lstlisting}
//Inside file my/shirt.js:
define({
    color: "black",
    size: "unisize"
});
    \end{lstlisting}

    如果没有依赖,只是需要函数来做一些先期的设定,那么讲一个函数传递给define()
    \begin{lstlisting}
//my/shirt.js now does setup work
//before returning its module definition.
define(function () {
    //Do setup work here

    return {
        color: "black",
        size: "unisize"
    }
});
    \end{lstlisting}

    如果有依赖,那么define()的一个参数是一个数组,数组元素是依赖的module的muduleID,第二个参数是一个函数,这个函数接受的参数就是你的module依赖的module,顺序和第一个参数数组元素顺序相同,这个函数必须返回一个对象,就是你定义的module。

    \begin{lstlisting}
//my/shirt.js now has some dependencies, a cart and inventory
//module in the same directory as shirt.js
define(["./cart", "./inventory"], function(cart, inventory) {
        //return an object to define the "my/shirt" module.
        return {
            color: "blue",
            size: "large",
            addToCart: function() {
                inventory.decrement(this);
                cart.add(this);
            }
        }
    }
);
    \end{lstlisting}

    在上面的例子中,/my/shirt这个module被创建了,他依赖于/my/cart和/my/inventory.第二个参数的function会在依赖的两个module加载起来之后再被执行。

    当然,这个返回的module不一定需要是个object,返回function对象也同样可以的。

    对于遵循CommonJS的代码,可以采用下面的方式来重用以后代码。

    \begin{lstlisting}
define(function(require, exports, module) {
        var a = require('a'),
            b = require('b');

        //Return the module value
        return function () {};
    }
);
    \end{lstlisting}

    有时候,你会看到define()的第一个参数是你定义的module的moduleID,但是不推荐这么写,因为如果你改变你的module的目录,你需要自己来更改moduleID,所以推荐由Require.js来自动定义你的moduleID。

    \begin{lstlisting}
//Explicitly defines the "foo/title" module:
    define("foo/title",
        ["my/cart", "my/inventory"],
        function(cart, inventory) {
            //Define foo/title object in here.
       }
    );
    \end{lstlisting}

    \subsection{Configuration}

    有几种方式来调用require.config(),先通过script标签来加载Require.js,然后在下一个script标签中包含对require.config()的调用的代码。
    \begin{lstlisting}
<script src="scripts/require.js"></script>
<script>
  require.config({
    baseUrl: "/another/path",
    paths: {
        "some": "some/v1.0"
    },
    waitSeconds: 15
  });
  require( ["some/module", "my/module", "a.js", "b.js"],
    function(someModule,    myModule) {
        //This function will be called when all the dependencies
        //listed above are loaded. Note that this function could
        //be called before the page is loaded.
        //This callback is optional.
    }
  );
</script>
    \end{lstlisting}
    也可以在main-data属性指定的js文件中调用。在这种情况下,需要注意script标签的异步加载,下面的标签引用的代码可能在main.js之前执行。
    \begin{lstlisting}
<script data-main="scripts/main" src="scripts/require.js"></script>
<script src="scripts/other.js"></script>
    \end{lstlisting}
    \begin{lstlisting}
// contents of main.js:
require.config({
    paths: {
        foo: 'libs/foo-1.1.3'
    }
});
    \end{lstlisting}
    \begin{lstlisting}
// contents of other.js:

// This code might be called before the require.config() in main.js
// has executed. When that happens, require.js will attempt to
// load 'scripts/foo.js' instead of 'scripts/libs/foo-1.1.3.js'
require( ['foo'], function( foo ) {

});
    \end{lstlisting}

    当然,你也可以先定义一个Require的全局config对象,然后再用script标签加载Require.js。使用 require = {},而不是window.require = {}。这样保证在IE下得正确性。

    \begin{lstlisting}
<script>
    var require = {
        deps: ["some/module1", "my/module2", "a.js", "b.js"],
        callback: function(module1, module2) {
            //This function will be called when all the dependencies
            //listed above in deps are loaded. Note that this
            //function could be called before the page is loaded.
            //This callback is optional.
        }
    };
</script>
<script src="scripts/require.js"></script>
    \end{lstlisting}

