\section{Require.js}
Require.js的目的是为了让我门编写模块化的代码,他遵循AMD规范。

项目的目录结构
\renewcommand{\labelitemi}{$\triangleright$}
\renewcommand{\labelitemii}{$\triangleright$}
\renewcommand{\labelitemiii}{$\triangleright$}
\renewcommand{\labelitemiv}{$\triangleright$}
\begin{itemize}
\item project-directory/
	\begin{itemize}
	\item project.html
	\item scripts/
		\begin{itemize}
		\item main.js
		\item require.js
		\item helper/
			\begin{itemize}
			\item util.js
			\end{itemize}
		\end{itemize}			
	\end{itemize}
\end{itemize}

\begin{HTML5}[project.html]
<!DOCTYPE html>
<html>
    <head>
        <title>My Sample Project</title>
        <!-- data-main attribute tells require.js to load
             scripts/main.js after require.js loads. -->
        <script data-main="scripts/main" src="scripts/require.js"></script>
    </head>
    <body>
        <h1>My Sample Project</h1>
    </body>
</html>
\end{HTML5}
可以将script tag放在body的结尾,这样require.js的加载就不会block页面的DOM的渲染了。如果浏览器支持,也可使用script标签的async属性。

\begin{JavaScript}[main.js 程序的入口]
require(["helper/util"], function(util) {
    //This function is called when scripts/helper/util.js is loaded.
    //If util.js calls define(), then this function is not fired until
    //util's dependencies have loaded, and the util argument will hold
    //the module value for "helper/util".
});
\end{JavaScript}

\subsection{加载}

\subsubsection{data-main Entry Point}

\begin{HTML5}[require.js会inject一个异步的script标签来加载main.js]
<!--when require.js loads it will inject another script tag
    (with async attribute) for scripts/main.js-->
<script data-main="scripts/main" src="scripts/require.js"></script>
\end{HTML5}

\begin{HTML5}[由于main是异步加载的,不能假设这个页面引用的其他script标签在main后执行]
<script data-main="scripts/main" src="scripts/require.js"></script>
<script src="scripts/other.js"></script>

// main.js
require.config({
    paths: {
        foo: 'libs/foo-1.1.3'
    }
});

// other.js
// This code might be called before the require.config() in main.js
// has executed. When that happens, require.js will attempt to
// load 'scripts/foo.js' instead of 'scripts/libs/foo-1.1.3.js'
require(['foo'], function(foo) {

});
\end{HTML5}

data-main一般用于但入口的应用。如果想使用require()。最好使用下面的方式
\begin{HTML5}
<script src="scripts/require.js"></script>
<script>
require(['scripts/config'], function() {
    // Configuration loaded now, safe to do other require calls
    // that depend on that config.
    require(['foo'], function(foo) {

    });
});
</script>
\end{HTML5}

\subsubsection{config}

\begin{HTML5}[在top-level的html页面,或者top-level的没有定义模块的script文件中。配置对象可以像下面的方式来传入]
<script src="scripts/require.js"></script>
<script>
  require.config({
    baseUrl: "/another/path",
    paths: {
        "some": "some/v1.0"
    },
    waitSeconds: 15
  });
  require( ["some/module", "my/module", "a.js", "b.js"],
    function(someModule,    myModule) {
        //This function will be called when all the dependencies
        //listed above are loaded. Note that this function could
        //be called before the page is loaded.
        //This callback is optional.
    }
  );
</script>
\end{HTML5}

You may also call require.config from your data-main Entry Point, but be aware that the data-main script is loaded asynchronously. Avoid other entry point scripts which wrongly assume that data-main and its require.config will always execute prior to their script loading.


\begin{HTML5}[也可以先于加载require.js之前定义一个全局的require配置文件]
<script>
    var require = {
        deps: ["some/module1", "my/module2", "a.js", "b.js"],
        callback: function(module1, module2) {
            //This function will be called when all the dependencies
            //listed above in deps are loaded. Note that this
            //function could be called before the page is loaded.
            //This callback is optional.
        }
    };
</script>
<script src="scripts/require.js"></script>
\end{HTML5}

最好使用\lstinline$var require = {...}$,而不是\lstinline$window.require = {...}$。在IE中的行为不正确。


requirejs假设所有的依赖都是script,所以moduleID不需要后缀.js。requirejs会自动补上.js将moduleID转换为path。

当不想按照baseUrl + path的方式来查找module的话,可以按下面的方式
\begin{itemize}
\item 以.js结尾
\item 以"/"开始
\item 包含URL协议,比如"http:"或者"https:"
\end{itemize}

推荐包结构:
\begin{itemize}
\item www/
	\begin{itemize}
	\item index.html
	\item js/
		\begin{itemize}
		\item app/
			\begin{itemize}
			\item sub.js
			\end{itemize}
		\item lib/
			\begin{itemize}
			\item jquery.js
			\item canvas.js
			\end{itemize}	
		\item app.js
		\item require.js						
		\end{itemize}
	\end{itemize}
\end{itemize}


\begin{HTML5}[index.html中]
<script data-main="js/app.js" src="js/require.js"></script>
\end{HTML5}

\begin{JavaScript}[app.js]
requirejs.config({
    //By default load any module IDs from js/lib
    baseUrl: 'js/lib',
    //except, if the module ID starts with "app",
    //load it from the js/app directory. paths
    //config is relative to the baseUrl, and
    //never includes a ".js" extension since
    //the paths config could be for a directory.
    paths: {
        app: '../app'
    }
});

// Start the main app logic.
requirejs(['jquery', 'canvas', 'app/sub'],
function   ($,        canvas,   sub) {
    //jQuery, canvas and the app/sub module are all
    //loaded and can be used here now.
});
\end{JavaScript}

%$

\subsection{BaseUrl}
Require.js根据BaseUrl来加载所有的js文件。

一般来说,baseUrl被设定和data-main属性中的script相同的目录。require.js会从data-main中设定的script开始加载。
\begin{HTML5}
<!--This sets the baseUrl to the "scripts" directory, and
    loads a script that will have a module ID of 'main'-->
<script data-main="scripts/main.js" src="scripts/require.js"></script>
\end{HTML5}

\begin{itemize}
\item 对于BaseUrl,一般来说,他是data-main属性引用的文件所在的目录,在上面的例子中,目录为scripts。
\item 同时,我们也可以在config中显示的指定,如果我们在config中指定了,将不再使用data-main属性的目录了。
\item 如果既没有main-data属性,也没有在config中指定,那么BaseUrl将是当前运行require.js的HTML文件所在的目录。
\end{itemize}


而在下面的三种情况下,Require.js不会采用baseUrl + path的方式来查找js文件。
\begin{itemize}
\item 以".js"结束的;
\item 以"/"开头的;
\item 以协议头,如"http:"开头的。
\end{itemize}


The baseUrl can be a URL on a different domain as the page that will load require.js. RequireJS script loading works across domains. 

The only restriction is on text content loaded by text! plugins: those paths should be on the same domain as the page, at least during development. The optimization tool will inline text! plugin resources so after using the optimization tool, you can use resources that reference text! plugin resources from another domain. ???


\subsection{BaseUrl + path}

path用来mapping在baseUrl下不能直接找到的module名字。一般和baseUrl配合使用。在下面的配置中,如果我们查找some/module。require.js会查找/another/path/some/v1.0/module.js。


\begin{HTML5}
<script src="scripts/require.js"></script>
<script>
  require.config({
    baseUrl: "/another/path",
    paths: {
        "some": "some/v1.0"
    },
    waitSeconds: 15
  });
  require( ["some/module", "my/module", "a.js", "b.js"],
    function(someModule,    myModule) {
        //This function will be called when all the dependencies
        //listed above are loaded. Note that this function could
        //be called before the page is loaded.
        //This callback is optional.
    }
  );
</script>
\end{HTML5}

而在下面的三种情况下,Require.js不会采用BaseUrl + path的方式来查找js文件。
\begin{enumerate}
\item 以".js"结束的;
\item 以"/"开头的;
\item 以协议头,如"http:"开头的。
\end{enumerate}

上面的例子中的a.js和b.js都会和从上面的html的相同的目录中读取。



??(这里应该是上面的some既可以做为路径的一部分,又可以直接作为module名字。)
The path that is used for a module name should not include an extension, since the path mapping could be for a directory. The path mapping code will automatically add the .js extension when mapping the module name to a path. 

 If require.toUrl() is used, it will add the appropriate extension, if it is for something like a text template.
 
\subsubsection{require.toUrl}

\subsubsection{path fallback}


\subsection{定义模块}

\subsubsection{JavaScript Module Pattern}

js模块模式。

基础

匿名闭包,定义一个立即执行的匿名函数,那么在这个匿名函数中的代码都在一个闭包中,在整个应用的生命周期中,提供了私有性。


\begin{JavaScript}[有没有括号是函数声明和函数表达式的区别]
(function () {
	// ... all vars and functions are in this scope only
	// still maintains access to all globals
}());
\end{JavaScript}

全局引入,通过给匿名函数传入参数的方式来引入全局变量,可以保持良好的代码风格,这样就不需要直接在匿名函数中使用全局变量。

\begin{JavaScript}
(function ($, YAHOO) {
	// now have access to globals jQuery (as $) and YAHOO in this code
}(jQuery, YAHOO));
\end{JavaScript}

模块导出,通过匿名函数的返回值来导出模块。

\begin{JavaScript}
var MODULE = (function () {
	var my = {},
		privateVariable = 1;

	function privateMethod() {
		// ...
	}

	my.moduleProperty = 1;
	my.moduleMethod = function () {
		// ...
	};

	return my;
}());
\end{JavaScript}

进阶

增强(augmentation),通过导入模块,然后在匿名函数中添加属性,这样就可以将module的代码分割到不同的文件中。

\begin{JavaScript}[var不是必须的,这样写是为了保持统一]
var MODULE = (function (my) {
	my.anotherMethod = function () {
		// added method...
	};

	return my;
}(MODULE));
\end{JavaScript}

松增强(loose augmentation),在第一次请求module的时候需要初始化module,augmentation发生在第二次。js的script可以异步加载。为了让multi-part的module可以按任意顺序加载,使用下面的结构来加载,这样多个文件就可以并行加载:

\begin{JavaScript}[这个结构下,var是必须的]
var MODULE = (function (my) {
	// add capabilities...

	return my;
}(MODULE || {}));
\end{JavaScript}


紧加载(Tight Augmentation),loose augmentation不能够安全的override module的properties,也不能在初始化阶段使用其他文件的module properties(但是可以在初始化之后的运行时这么做)。

\begin{JavaScript}[保留一个原始方法的引用的方式来override方法]
var MODULE = (function (my) {
	var old_moduleMethod = my.moduleMethod;

	my.moduleMethod = function () {
		// method override, has access to old through old_moduleMethod...
	};

	return my;
}(MODULE));
\end{JavaScript}
克隆和继承(Cloning \& Inheritance)

\begin{JavaScript}
var MODULE_TWO = (function (old) {
	var my = {},
		key;

	for (key in old) {
		if (old.hasOwnProperty(key)) {
			my[key] = old[key];
		}
	}

	var super_moduleMethod = old.moduleMethod;
	my.moduleMethod = function () {
		// override method on the clone, access to super through super_moduleMethod
	};

	return my;
}(MODULE));
\end{JavaScript}


跨文件私有状态(Cross File Private State)

\begin{JavaScript}
var MODULE = (function (my) {
	var _private = my._private = my._private || {},
		_seal = my._seal = my._seal || function () {
			delete my._private;
			delete my._seal;
			delete my._unseal;
		},
		_unseal = my._unseal = my._unseal || function () {
			my._private = _private;
			my._seal = _seal;
			my._unseal = _unseal;
		};

	// permanent access to _private, _seal, and _unseal

	return my;
}(MODULE || {}));
\end{JavaScript}


当某个文件设定好\lstinline$_private$之后,其他的文件马上都可以使用。一旦module完成加载,应该调用\lstinline$_unseal()$方法,这样可以阻止外面访问\lstinline$_private$。如果想要在augment这个module,应该在augment之前先通过某个文件的内部方法调用\lstinline$_unseal$,然后在其之后调用\lstinline$_seal$方法。	


子模块(Sub-modules)

\begin{JavaScript}
MODULE.sub = (function () {
	var my = {};
	// ...

	return my;
}());
\end{JavaScript}

\subsubsection{简单模块的定义}

\begin{JavaScript}[如果模块没有依赖,可以这样定义]
//Inside file my/shirt.js:
define({
    color: "black",
    size: "unisize"
});
\end{JavaScript}

\begin{JavaScript}[虽然没有依赖,但是需要做一些setup的工作,可以这样定义]
//my/shirt.js now does setup work
//before returning its module definition.
define(function () {
    //Do setup work here

    return {
        color: "black",
        size: "unisize"
    }
});
\end{JavaScript}

\begin{JavaScript}[如果有依赖,第一个参数是依赖的module的数组,函数将会按相同的顺序传入module来调用]
//my/shirt.js now has some dependencies, a cart and inventory
//module in the same directory as shirt.js
define(["./cart", "./inventory"], function(cart, inventory) {
        //return an object to define the "my/shirt" module.
        return {
            color: "blue",
            size: "large",
            addToCart: function() {
                inventory.decrement(this);
                cart.add(this);
            }
        }
    }
);
\end{JavaScript}

\begin{JavaScript}[任何有效的函数返回值都可以返回]
//A module definition inside foo/title.js. It uses
//my/cart and my/inventory modules from before,
//but since foo/title.js is in a different directory than
//the "my" modules, it uses the "my" in the module dependency
//name to find them. The "my" part of the name can be mapped
//to any directory, but by default, it is assumed to be a
//sibling to the "foo" directory.
define(["my/cart", "my/inventory"],
    function(cart, inventory) {
        //return a function to define "foo/title".
        //It gets or sets the window title.
        return function(title) {
            return title ? (window.title = title) :
                   inventory.storeName + ' ' + cart.name;
        }
    }
);
\end{JavaScript}


\begin{JavaScript}[定义module的名字,一般优化工具来做这个事,自己不应该这么做]
    //Explicitly defines the "foo/title" module:
    define("foo/title",
        ["my/cart", "my/inventory"],
        function(cart, inventory) {
            //Define foo/title object in here.
       }
    );
\end{JavaScript}

\begin{JavaScript}[取得Module对应的URL]
define(["require"], function(require) {
    var cssUrl = require.toUrl("./style.css");
});
\end{JavaScript}

\begin{JavaScript}[控制台取得Module]
require("module/name").callSomeFunction()
\end{JavaScript}

a依赖b,b依赖a,这是在b中会取得一个a的依赖为undefined。b可以在随后通过require()方法来取得a的引用
\begin{JavaScript}[循环依赖时]
//Inside b.js:
define(["require", "a"],
    function(require, a) {
        //"a" in this case will be null if "a" also asked for "b",
        //a circular dependency.
        return function(title) {
            return require("a").doSomething();
        }
    }
);
\end{JavaScript}

使用commonjs风格的exports的话,我们可以取得一个真正的a的引用,但是在b返回值之前,我们是不能够使用a的方法的。

\begin{JavaScript}
//Inside b.js:
define(function(require, exports, module) {
    //If "a" has used exports, then we have a real
    //object reference here. However, we cannot use
    //any of "a"'s properties until after "b" returns a value.
    var a = require("a");

    exports.foo = function () {
        return a.bar();
    };
});
\end{JavaScript}

\begin{JavaScript}
//Inside b.js:
define(['a', 'exports'], function(a, exports) {
    //If "a" has used exports, then we have a real
    //object reference here. However, we cannot use
    //any of "a"'s properties until after "b" returns a value.

    exports.foo = function () {
        return a.bar();
    };
});
\end{JavaScript}



??? 相对module名好像是先相对于module明来说的,然后再将module名转换为对应的Path


\subsection{配置require}

\begin{JavaScript}[在top-level HTML或者没有模块定义的script文件中]
<script src="scripts/require.js"></script>
<script>
  require.config({
    baseUrl: "/another/path",
    paths: {
        "some": "some/v1.0"
    },
    waitSeconds: 15
  });
  require( ["some/module", "my/module", "a.js", "b.js"],
    function(someModule,    myModule) {
        //This function will be called when all the dependencies
        //listed above are loaded. Note that this function could
        //be called before the page is loaded.
        //This callback is optional.
    }
  );
</script>
\end{JavaScript}

也可以使用data-main Entry Point来读取配置,不过要注意这样是异步加载的,要避免其他入口程序在这个配置之前执行。

\begin{JavaScript}[也可以在加载requireJS之前先定义一个require对象]
<script>
    var require = {
        deps: ["some/module1", "my/module2", "a.js", "b.js"],
        callback: function(module1, module2) {
            //This function will be called when all the dependencies
            //listed above in deps are loaded. Note that this
            //function could be called before the page is loaded.
            //This callback is optional.
        }
    };
</script>
<script src="scripts/require.js"></script>
\end{JavaScript}

\subsubsection{baseUrl} 所有模块查找的root目录。

下面的情况下不会使用baseUrl,这个时候就如同文档中其他地方的url是一样:
\begin{itemize}
\item 以/开头
\item 以.js结尾
\item 包含网络协议头,如"http:"或者"https:"
\end{itemize}

这里涉及到绝对路径和相对路径【http://stackoverflow.com/questions/10659459/starting-with-a-forward-slash-in-html-for-href】

如果没有显示的设置baseUrl,则默认为当前html所在的目录,如果有main-data属性,则是这里设定的目录。

baseUrl可以和当前html处于不同的domain。requirejs可以跨domain加载脚本。唯一的限制是如果使用text! plugins。 那么就得和当前html处于同一个domain。

The optimization tool will inline text! plugin resources so after using the optimization tool, you can use resources that reference text! plugin resources from another domain.

\subsubsection{path}
path mappings for module names not found directly under baseUrl.

path用于mapping在baseUrl下找不到的module名字, 比如上面的some/module。

path不应该包含扩展名,因为路径可能用来匹配目录(???)。The path mapping code will automatically add the .js extension when mapping the module name to a path. 

\subsubsection{bundle}
Introduced in RequireJS 2.1.10: allows configuring multiple module IDs to be found in another script.

\begin{JavaScript}
requirejs.config({
    bundles: {
        'primary': ['main', 'util', 'text', 'text!template.html'],
        'secondary': ['text!secondary.html']
    }
});

require(['util', 'text'], function(util, text) {
    //The script for module ID 'primary' was loaded,
    //and that script included the define()'d
    //modules for 'util' and 'text'
});
\end{JavaScript}

That config states: modules 'main', 'util', 'text' and 'text!template.html' will be found by loading module ID 'primary'. Module 'text!secondary.html' can be found by loading module ID 'secondary'.


This only sets up where to find a module inside a script that has multiple define()'d modules in it. It does not automatically bind those modules to the bundle's module ID. The bundle's module ID is just used for locating the set of modules.

Something similar is possible with paths config, but it is much wordier, and the paths config route does not allow loader plugin resource IDs in its configuration, since the paths config values are path segments, not IDs.


bundles config is useful if doing a build and that build target was not an existing module ID, or if you have loader plugin resources in built JS files that should not be loaded by the loader plugin. Note that the keys and values are module IDs, not path segments. They are absolute module IDs, not a module ID prefix like paths config or map config. Also, bundle config is different from map config in that map config is a one-to-one module ID relationship, where bundle config is for pointing multiple module IDs to a bundle's module ID.


\subsubsection{shim}
Here is an example. It requires RequireJS 2.1.0+, and assumes backbone.js, underscore.js and jquery.js have been installed in the baseUrl directory. If not, then you may need to set a paths config for them:
\begin{JavaScript}[假设jquery.js, backbone.js,underscore.js都在basUrl的目录里]
requirejs.config({
    //Remember: only use shim config for non-AMD scripts,
    //scripts that do not already call define(). The shim
    //config will not work correctly if used on AMD scripts,
    //in particular, the exports and init config will not
    //be triggered, and the deps config will be confusing
    //for those cases.
    shim: {
        'backbone': {
            //These script dependencies should be loaded before loading
            //backbone.js
            deps: ['underscore', 'jquery'],
            //Once loaded, use the global 'Backbone' as the
            //module value.
            exports: 'Backbone'
        },
        'underscore': {
            exports: '_'
        },
        'foo': {
            deps: ['bar'],
            exports: 'Foo',
            init: function (bar) {
                //Using a function allows you to call noConflict for
                //libraries that support it, and do other cleanup.
                //However, plugins for those libraries may still want
                //a global. "this" for the function will be the global
                //object. The dependencies will be passed in as
                //function arguments. If this function returns a value,
                //then that value is used as the module export value
                //instead of the object found via the 'exports' string.
                //Note: jQuery registers as an AMD module via define(),
                //so this will not work for jQuery. See notes section
                //below for an approach for jQuery.
                return this.Foo.noConflict();
            }
        }
    }
});

//Then, later in a separate file, call it 'MyModel.js', a module is
//defined, specifying 'backbone' as a dependency. RequireJS will use
//the shim config to properly load 'backbone' and give a local
//reference to this module. The global Backbone will still exist on
//the page too.
define(['backbone'], function (Backbone) {
  return Backbone.Model.extend({});
});
\end{JavaScript}

For "modules" that are just jQuery or Backbone plugins that do not need to export any module value, the shim config can just be an array of dependencies:

\begin{JavaScript}
requirejs.config({
    shim: {
        'jquery.colorize': ['jquery'],
        'jquery.scroll': ['jquery'],
        'backbone.layoutmanager': ['backbone']
    }
});
\end{JavaScript}

Note however if you want to get 404 load detection in IE so that you can use paths fallbacks or errbacks, then a string exports value should be given so the loader can check if the scripts actually loaded (a return from init is not used for enforceDefine checking):

\begin{JavaScript}
requirejs.config({
    shim: {
        'jquery.colorize': {
            deps: ['jquery'],
            exports: 'jQuery.fn.colorize'
        },
        'jquery.scroll': {
            deps: ['jquery'],
            exports: 'jQuery.fn.scroll'
        },
        'backbone.layoutmanager': {
            deps: ['backbone']
            exports: 'Backbone.LayoutManager'
        }
    }
});
\end{JavaScript}


shim需要注意的内容:
\begin{itemize}
\item shim config只是配置了code的关系,为了加载使用shim config的module,需要调用require/define。设置shim不会触发code的加载。
\item Only use other "shim" modules as dependencies for shimmed scripts, or AMD libraries that have no dependencies and call define() after they also create a global (like jQuery or lodash). Otherwise, if you use an AMD module as a dependency for a shim config module, after a build, that AMD module may not be evaluated until after the shimmed code in the build executes, and an error will occur. The ultimate fix is to upgrade all the shimmed code to have optional AMD define() calls.
\end{itemize}
\item shimed的script只能依赖shim modules,或者是那些没有依赖,并且盗用define()后创建全局对象。否则如果你使用AMD module作为依赖的话,AMD module可能在你使用shimmed code的时候还没有执行。

\item If it is not possible to upgrade the shimmed code to use AMD define() calls, as of RequireJS 2.1.11, the optimizer has a wrapShim build option that will try to automatically wrap the shimmed code in a define() for a build. This changes the scope of shimmed dependencies, so it is not guaranteed to always work, but, for example, for shimmed dependencies that depend on an AMD version of Backbone, it can be helpful.

\item The init function will not be called for AMD modules. For example, you cannot use a shim init function to call jQuery's noConflict. See Mapping Modules to use noConflict for an alternate approach to jQuery.   ???为啥不能用呢?

\item Shim config is not supported when running AMD modules in node via RequireJS (it works for optimizer use though). Depending on the module being shimmed, it may fail in Node because Node does not have the same global environment as browsers. As of RequireJS 2.1.7, it will warn you in the console that shim config is not supported, and it may or may not work. If you wish to suppress that message, you can pass requirejs.config({ suppress: { nodeShim: true }});.



Important optimizer notes for "shim" config:
\begin{itemize}
\item You should use the mainConfigFile build option to specify the file where to find the shim config. Otherwise the optimizer will not know of the shim config. The other option is to duplicate the shim config in the build profile.
\item Do not mix CDN loading with shim config in a build. Example scenario: you load jQuery from the CDN but use the shim config to load something like the stock version of Backbone that depends on jQuery. When you do the build, be sure to inline jQuery in the built file and do not load it from the CDN. Otherwise, Backbone will be inlined in the built file and it will execute before the CDN-loaded jQuery will load. This is because the shim config just delays loading of the files until dependencies are loaded, but does not do any auto-wrapping of define. After a build, the dependencies are already inlined, the shim config cannot delay execution of the non-define()'d code until later. define()'d modules do work with CDN loaded code after a build because they properly wrap their source in define factory function that will not execute until dependencies are loaded. So the lesson: shim config is a stop-gap measure for non-modular code, legacy code. define()'d modules are better.
\item For local, multi-file builds, the above CDN advice also applies. For any shimmed script, its dependencies must be loaded before the shimmed script executes. This means either building its dependencies directly in the buid layer that includes the shimmed script, or loading its dependencies with a require([], function (){}) call, then doing a nested require([]) call for the build layer that has the shimmed script.
\item If you are using uglifyjs to minify the code, do not set the uglify option toplevel to true, or if using the command line do not pass -mt. That option mangles the global names that shim uses to find exports.
\end{itemize}





\subsection{模块定义}
模块应该遵循js的模块模式[先省略,日后补充][http://www.adequatelygood.com/JavaScript-Module-Pattern-In-Depth.html]

简单的键值对,没有依赖,那么将键值对对象作为参数传递给define()

\begin{lstlisting}
//Inside file my/shirt.js:
define({
    color: "black",
    size: "unisize"
});
\end{lstlisting}

    如果没有依赖,只是需要函数来做一些先期的设定,那么讲一个函数传递给define()
    \begin{lstlisting}
//my/shirt.js now does setup work
//before returning its module definition.
define(function () {
    //Do setup work here

    return {
        color: "black",
        size: "unisize"
    }
});
    \end{lstlisting}

    如果有依赖,那么define()的一个参数是一个数组,数组元素是依赖的module的muduleID,第二个参数是一个函数,这个函数接受的参数就是你的module依赖的module,顺序和第一个参数数组元素顺序相同,这个函数必须返回一个对象,就是你定义的module。

    \begin{lstlisting}
//my/shirt.js now has some dependencies, a cart and inventory
//module in the same directory as shirt.js
define(["./cart", "./inventory"], function(cart, inventory) {
        //return an object to define the "my/shirt" module.
        return {
            color: "blue",
            size: "large",
            addToCart: function() {
                inventory.decrement(this);
                cart.add(this);
            }
        }
    }
);
    \end{lstlisting}

    在上面的例子中,/my/shirt这个module被创建了,他依赖于/my/cart和/my/inventory.第二个参数的function会在依赖的两个module加载起来之后再被执行。

    当然,这个返回的module不一定需要是个object,返回function对象也同样可以的。

    对于遵循CommonJS的代码,可以采用下面的方式来重用以后代码。

    \begin{lstlisting}
define(function(require, exports, module) {
        var a = require('a'),
            b = require('b');

        //Return the module value
        return function () {};
    }
);
    \end{lstlisting}

    有时候,你会看到define()的第一个参数是你定义的module的moduleID,但是不推荐这么写,因为如果你改变你的module的目录,你需要自己来更改moduleID,所以推荐由Require.js来自动定义你的moduleID。

    \begin{lstlisting}
//Explicitly defines the "foo/title" module:
    define("foo/title",
        ["my/cart", "my/inventory"],
        function(cart, inventory) {
            //Define foo/title object in here.
       }
    );
    \end{lstlisting}

    \subsection{Configuration}

    有几种方式来调用require.config(),先通过script标签来加载Require.js,然后在下一个script标签中包含对require.config()的调用的代码。
    \begin{lstlisting}
<script src="scripts/require.js"></script>
<script>
  require.config({
    baseUrl: "/another/path",
    paths: {
        "some": "some/v1.0"
    },
    waitSeconds: 15
  });
  require( ["some/module", "my/module", "a.js", "b.js"],
    function(someModule,    myModule) {
        //This function will be called when all the dependencies
        //listed above are loaded. Note that this function could
        //be called before the page is loaded.
        //This callback is optional.
    }
  );
</script>
    \end{lstlisting}
    也可以在main-data属性指定的js文件中调用。在这种情况下,需要注意script标签的异步加载,下面的标签引用的代码可能在main.js之前执行。
    \begin{lstlisting}
<script data-main="scripts/main" src="scripts/require.js"></script>
<script src="scripts/other.js"></script>
    \end{lstlisting}
    \begin{lstlisting}
// contents of main.js:
require.config({
    paths: {
        foo: 'libs/foo-1.1.3'
    }
});
    \end{lstlisting}
    \begin{lstlisting}
// contents of other.js:

// This code might be called before the require.config() in main.js
// has executed. When that happens, require.js will attempt to
// load 'scripts/foo.js' instead of 'scripts/libs/foo-1.1.3.js'
require( ['foo'], function( foo ) {

});
    \end{lstlisting}

    当然,你也可以先定义一个Require的全局config对象,然后再用script标签加载Require.js。使用 require = {},而不是window.require = {}。这样保证在IE下得正确性。

    \begin{lstlisting}
<script>
    var require = {
        deps: ["some/module1", "my/module2", "a.js", "b.js"],
        callback: function(module1, module2) {
            //This function will be called when all the dependencies
            //listed above in deps are loaded. Note that this
            //function could be called before the page is loaded.
            //This callback is optional.
        }
    };
</script>
<script src="scripts/require.js"></script>
    \end{lstlisting}

