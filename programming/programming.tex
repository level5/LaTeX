\documentclass[12pt, oneside, b5paper]{book}  


\usepackage[a3paper]{geometry}

\usepackage{listings}
\usepackage{CJKutf8}  

\usepackage{color}
\definecolor{lightgray}{rgb}{.9,.9,.9}
\definecolor{darkgray}{rgb}{.4,.4,.4}
\definecolor{purple}{rgb}{0.65, 0.12, 0.82}

\lstdefinelanguage{JavaScript}{
  keywords={typeof, new, true, false, catch, function, return, null, catch, switch, var, if, in, while, do, else, case, break},
  keywordstyle=\color{blue}\bfseries,
  ndkeywords={class, export, boolean, throw, implements, import, this},
  ndkeywordstyle=\color{darkgray}\bfseries,
  identifierstyle=\color{black},
  sensitive=false,
  comment=[l]{//},
  morecomment=[s]{/*}{*/},
  morecomment=[s]{<!--}{-->},
  commentstyle=\color{purple}\ttfamily,
  stringstyle=\color{red}\ttfamily,
  morestring=[b]',
  morestring=[b]"
}

\lstset{
   language=JavaScript,
   backgroundcolor=\color{lightgray},
   extendedchars=true,
   basicstyle=\footnotesize\ttfamily,
   showstringspaces=false,
   showspaces=false,
   numbers=left,
   numberstyle=\footnotesize,
   numbersep=9pt,
   tabsize=2,
   breaklines=true,
   showtabs=false,
   captionpos=b
}

\pagestyle{headings}
%\pagenumbering{arabic}

\title{编程}
\author{石峰\\
			上海 长宁}
\date{\today}
	
\begin{document}  
\begin{CJK}{UTF8}{gkai}
\maketitle
%\tableofcontents

\chapter{JavaScript}
	\section{JavaScript 语言}
	JavaScript语言有一些重要的概念,原型,函数,作用域和闭包.掌握了这些概念,基本就掌握了JavaScript这门语言,其他的就是一些语言细节特性.
		\subsection{原型}
		我们所说的面向对象.继承,封装和多态.从语言上来说,有两种实现面向对象的方式:基于类和基于原型的继承.
	
		基于类的继承. 类是模板,继承的是行为.这类语言有Java.
		
		基于原型的继承, 所有的都是对象(当然还有基本类型),继承的是状态和行为.这类语言有JavaScript

		在JavaScript的世界中,只有对象这个概念.当然还有基本类型,但是在需要对象的时候,会自动将基本类型转换为对象的.

		先来看看简单的代码,
	
		\begin{lstlisting}
var a = {};	
a.b = 10;
console.log(a.b); // 10
		\end{lstlisting}

		JavasScript中对象的定义是: Object是Property的集合. JavaScript对象是一组属性的集合,这些属性引用的是一个对象或者基本类型.
	
		\subsection{函数}
		函数在JavaScript中是第一等公民。
		\subsection{作用域}
		函数中的变量的定义和取值和函数的作用域有关。
		\subsection{闭包}
		函数和作用域,就构成了闭包。
		\subsection{语言细节}
		这部分可以讨论一下很多关于JavaScript的细节.
		\section{JavaScript环境}
		\section{JavaScript类库}
		\subsection{Backbone.js}	
		这是一个轻量级的MVP框架.分析一下源码.

    \paragraph{加载}
		
		\begin{lstlisting}
(function(root, factory) {

  // Set up Backbone appropriately for the environment. Start with AMD.
  if (typeof define === 'function' && define.amd) {
    define(['underscore', 'jquery', 'exports'], function(_, $, exports) {
      // Export global even in AMD case in case this script is loaded with
      // others that may still expect a global Backbone.
      root.Backbone = factory(root, exports, _, $);
    });

  // Next for Node.js or CommonJS. jQuery may not be needed as a module.
  } else if (typeof exports !== 'undefined') {
    var _ = require('underscore');
    factory(root, exports, _);

  // Finally, as a browser global.
  } else {
    root.Backbone = factory(root, {}, root._, (root.jQuery || root.Zepto || root.ender || root.$));
  }

}(this, function(root, Backbone, _, $) {
	...
}));
		\end{lstlisting}

		在这段代码中,定义的匿名函数是用来在各种不同的情况下加载加载Backbone. 在支持AMD的情况,在Node.js,在browser中。factory这个函数才是定义Backbone的地方。接着让我们看看factory函数。


		在factory函数的开始,是一些基本的设置,如版本,引用一些常用的方法,解决冲突的noConfilict。

		\begin{lstlisting}
  // Initial Setup
  // -------------

  // Save the previous value of the `Backbone` variable, so that it can be
  // restored later on, if `noConflict` is used.
  var previousBackbone = root.Backbone;

  // Create local references to array methods we'll want to use later.
  var array = [];
  var push = array.push;
  var slice = array.slice;
  var splice = array.splice;

  // Current version of the library. Keep in sync with `package.json`.
  Backbone.VERSION = '1.1.2';

  // For Backbone's purposes, jQuery, Zepto, Ender, or My Library (kidding) owns
  // the `$` variable.
  Backbone.$ = $;

  // Runs Backbone.js in *noConflict* mode, returning the `Backbone` variable
  // to its previous owner. Returns a reference to this Backbone object.
  Backbone.noConflict = function() {
    root.Backbone = previousBackbone;
    return this;
  };
		\end{lstlisting}

		再继续,就是Backbone的几个重要的模块了。Event,Model,Collection和View。

		\paragraph{Event}
		Event模块有几个很常用的方法,on, once, off, trigger. 他们都使用了一个方法eventApi。我们先来看这个方法,然后再去理解上面说到的这些方法。这个方法是用来将key-value形式的name和listener分别调用对应API。或者是空格分割的多个name。如果上述两种情况存在,就返回false,如果什么都没有做,则返回true。

	\begin{lstlisting}
  // Implement fancy features of the Events API such as multiple event
  // names `"change blur"` and jQuery-style event maps `{change: action}`
  // in terms of the existing API.
  var eventsApi = function(obj, action, name, rest) {
    if (!name) return true;

    // Handle event maps.
    if (typeof name === 'object') {
      for (var key in name) {
        obj[action].apply(obj, [key, name[key]].concat(rest));
      }
      return false;
    }

    // Handle space separated event names.
    if (eventSplitter.test(name)) {
      var names = name.split(eventSplitter);
      for (var i = 0, l = names.length; i < l; i++) {
        obj[action].apply(obj, [names[i]].concat(rest));
      }
      return false;
    }

    return true;
  };
	\end{lstlisting}

  \paragraph{Model}
  这里,在第四行,给每个Model定义了一个唯一ID。第九行,将属性通过set方法赋值给model。最后,调用initialize方法,这个方法在定义自己的Model的时候可以重写,其他的行暂时不去管它,因为我现在也不知道作用。
  \begin{lstlisting}
  var Model = Backbone.Model = function(attributes, options) {
    var attrs = attributes || {};
    options || (options = {});
    this.cid = _.uniqueId('c');
    this.attributes = {};
    if (options.collection) this.collection = options.collection;
    if (options.parse) attrs = this.parse(attrs, options) || {};
    attrs = _.defaults({}, attrs, _.result(this, 'defaults'));
    this.set(attrs, options);
    this.changed = {};
    this.initialize.apply(this, arguments);
  };
  \end{lstlisting}

  接下来,重点看看set方法。
  \begin{lstlisting}
    // Set a hash of model attributes on the object, firing `"change"`. This is
    // the core primitive operation of a model, updating the data and notifying
    // anyone who needs to know about the change in state. The heart of the beast.
    set: function(key, val, options) {
      var attr, attrs, unset, changes, silent, changing, prev, current;
      if (key == null) return this;

      // Handle both `"key", value` and `{key: value}` -style arguments.
      if (typeof key === 'object') {
        attrs = key;
        options = val;
      } else {
        (attrs = {})[key] = val;
      }

      options || (options = {});
  \end{lstlisting}
  这一段,是对key,value和{key:value}情况的处理,如果是后一种情况,那么第二个参数val引用的就是实际上的options,所以这个地方需要重新赋值一下,最后如果options是空,就给他一个空对象。

  \begin{lstlisting}
      // Run validation.
      if (!this._validate(attrs, options)) return false;
  \end{lstlisting}

  如果需要验证,进行验证,失败了的话直接跳出方法。下面是这个方法最复杂的逻辑部分了。在下面开始的部分,会将this.\_changing设为true,然后在方法的最后,将this.\_changing设为false。并且在改变值之前,将她赋值为局部变量changing.这么做的原因是设定一个flag,用来表示这是set方法最开始进入的,还是在调用set方法时,因为一些事件啊,什么的,触发的第二次,第三次进入set。因为如果是第一次进入set,就将原来的attributes保存在this.\_previousAttributes中,并且将this.changed清空。用来保存将要发生改变了的attributes的名字。

  \begin{lstlisting}

      // Extract attributes and options.
      unset           = options.unset;
      silent          = options.silent;
      changes         = [];
      changing        = this._changing;
      this._changing  = true;

      if (!changing) {
        this._previousAttributes = _.clone(this.attributes);
        this.changed = {};
      }

  \end{lstlisting}
  下面的代码对于第一次进入,第二次进入set都是相同的逻辑
  \begin{lstlisting}
      current = this.attributes, prev = this._previousAttributes;

  \end{lstlisting}
  这里,局部变量current保存当前this.attributes,局部变量prev保存的是第一次调用set前的attributes的值。
  \begin{lstlisting}
      // Check for changes of `id`.
      if (this.idAttribute in attrs) this.id = attrs[this.idAttribute];
  \end{lstlisting}
  首先改变id的值,这一步目前我还不知道是为啥。留在后面来更行。

  \begin{lstlisting}
      // For each `set` attribute, update or delete the current value.
      for (attr in attrs) {
        val = attrs[attr];
        if (!_.isEqual(current[attr], val)) changes.push(attr);
        if (!_.isEqual(prev[attr], val)) {
          this.changed[attr] = val;
        } else {
          delete this.changed[attr];
        }
        unset ? delete current[attr] : current[attr] = val;
      }
  \end{lstlisting}

  在这里,如果val和当前值比发生变化,就加入到局部变量changes这个数组中,如果和pre比起来发生变化,就加入到this.changed中,相同的话,就将她从this.changed中删除掉。这里有可能是第一次set将值改变了,第二次set又将值改回来了,这个时候就需要从this.changed中删除掉。最后,如果option中带有unset标志,就删除这个值,否则,修改这个值。

  接下来就是属性修改之后的事件的触发了。
 
  \begin{lstlisting}
      // Trigger all relevant attribute changes.
      if (!silent) {
        if (changes.length) this._pending = options;
        for (var i = 0, l = changes.length; i < l; i++) {
          this.trigger('change:' + changes[i], this, current[changes[i]], options);
        }
      }
  \end{lstlisting}
  如果需要触发事件,那么就出发每个值发生改变了属性的对应事件。
  \begin{lstlisting}
      // You might be wondering why there's a `while` loop here. Changes can
      // be recursively nested within `"change"` events.
      if (changing) return this;
  \end{lstlisting}
  到这一步,如果不是第一次调用set,他们的changing都会是true,到此这里为止。

  而如果是第一次调用set,执行下面的代码。来触发change事件,这里之所以用while,是因为在change事件中,可能又会有属性发生变法,需要再次触发change事件。最后充值\_pending和\_changing。
  \begin{lstlisting}
      if (!silent) {
        while (this._pending) {
          options = this._pending;
          this._pending = false;
          this.trigger('change', this, options);
        }
      }
      this._pending = false;
      this._changing = false;
      return this;
    },

  \end{lstlisting}

  接下来的两个方法很好理解,只是对上面的set方法的调用,并设定好unset,value设置为undefined,这里注意的void 0,这个会在另外一个部分来解释为什么void 0的值是undefined,这样做的一个更好的原因是void 0肯定是undefined,但是undefined不是关键字,可以定义一个变量叫做undefined。

  \begin{lstlisting}
    // Remove an attribute from the model, firing `"change"`. `unset` is a noop
    // if the attribute doesn't exist.
    unset: function(attr, options) {
      return this.set(attr, void 0, _.extend({}, options, {unset: true}));
    },

    // Clear all attributes on the model, firing `"change"`.
    clear: function(options) {
      var attrs = {};
      for (var key in this.attributes) attrs[key] = void 0;
      return this.set(attrs, _.extend({}, options, {unset: true}));
    },

  \end{lstlisting}

  接下来是一个很有用的方法,判断那些属性发生变化,这样可判断view的那部分需要更新,那部分数据需要push到sever上。
  如果没有传入参数,那么就直接返回this.changed的一个克隆或者false。

  如果传入了diff,那就返回Model和diff中不同的属性。
  \begin{lstlisting}
    // Return an object containing all the attributes that have changed, or
    // false if there are no changed attributes. Useful for determining what
    // parts of a view need to be updated and/or what attributes need to be
    // persisted to the server. Unset attributes will be set to undefined.
    // You can also pass an attributes object to diff against the model,
    // determining if there *would be* a change.
    changedAttributes: function(diff) {
      if (!diff) return this.hasChanged() ? _.clone(this.changed) : false;
      var val, changed = false;
      var old = this._changing ? this._previousAttributes : this.attributes;
      for (var attr in diff) {
        if (_.isEqual(old[attr], (val = diff[attr]))) continue;
        (changed || (changed = {}))[attr] = val;
      }
      return changed;
    },
  \end{lstlisting}

  接下来,是重要的方法,fetch,save。这两个方法会调用Restful接口。

  首先处理的是key, value和{key: value}的情况。

  \begin{lstlisting}
    // Set a hash of model attributes, and sync the model to the server.
    // If the server returns an attributes hash that differs, the model's
    // state will be `set` again.
    save: function(key, val, options) {
      var attrs, method, xhr, attributes = this.attributes;

      // Handle both `"key", value` and `{key: value}` -style arguments.
      if (key == null || typeof key === 'object') {
        attrs = key;
        options = val;
      } else {
        (attrs = {})[key] = val;
      }
  \end{lstlisting}

  \begin{lstlisting}
      options = _.extend({validate: true}, options);

      // If we're not waiting and attributes exist, save acts as
      // `set(attr).save(null, opts)` with validation. Otherwise, check if
      // the model will be valid when the attributes, if any, are set.
      if (attrs && !options.wait) {
        if (!this.set(attrs, options)) return false;
      } else {
        if (!this._validate(attrs, options)) return false;
      }

      // Set temporary attributes if `{wait: true}`.
      if (attrs && options.wait) {
        this.attributes = _.extend({}, attributes, attrs);
      }

      // After a successful server-side save, the client is (optionally)
      // updated with the server-side state.
      if (options.parse === void 0) options.parse = true;
      var model = this;
      var success = options.success;
      options.success = function(resp) {
        // Ensure attributes are restored during synchronous saves.
        model.attributes = attributes;
        var serverAttrs = model.parse(resp, options);
        if (options.wait) serverAttrs = _.extend(attrs || {}, serverAttrs);
        if (_.isObject(serverAttrs) && !model.set(serverAttrs, options)) {
          return false;
        }
        if (success) success(model, resp, options);
        model.trigger('sync', model, resp, options);
      };
      wrapError(this, options);

      method = this.isNew() ? 'create' : (options.patch ? 'patch' : 'update');
      if (method === 'patch') options.attrs = attrs;
      xhr = this.sync(method, this, options);

      // Restore attributes.
      if (attrs && options.wait) this.attributes = attributes;

      return xhr;
    },
  \end{lstlisting}

  \begin{lstlisting}
    // Destroy this model on the server if it was already persisted.
    // Optimistically removes the model from its collection, if it has one.
    // If `wait: true` is passed, waits for the server to respond before removal.
    destroy: function(options) {
      options = options ? _.clone(options) : {};
      var model = this;
      var success = options.success;

      var destroy = function() {
        model.trigger('destroy', model, model.collection, options);
      };

      options.success = function(resp) {
        if (options.wait || model.isNew()) destroy();
        if (success) success(model, resp, options);
        if (!model.isNew()) model.trigger('sync', model, resp, options);
      };

      if (this.isNew()) {
        options.success();
        return false;
      }
      wrapError(this, options);

      var xhr = this.sync('delete', this, options);
      if (!options.wait) destroy();
      return xhr;
    },
  \end{lstlisting}

  下面是Collection的源码。Model表示的是数据库的一行数据,而Collection表示的则是整个表,或者多行的数据。
  \begin{lstlisting} 

  \end{lstlisting}

  \paragraph{}


	\subsection{Require.js}
	Require.js的目的是为了让我门编写模块化的代码,他遵循AMD规范。

	\paragraph{BaseUrl}
	Require.js根据BaseUrl来加载所有的js文件。

	\begin{lstlisting}
<!--This sets the baseUrl to the "scripts" directory, and
    loads a script that will have a module ID of 'main'-->
<script data-main="scripts/main.js" src="scripts/require.js"></script>
	\end{lstlisting}

	对于BaseUrl,一般来说,他是data-main属性引用的文件所在的目录,在上面的例子中,目录为scripts。
	同时,我们也可以在config中显示的指定,如果我们在config中指定了,将不再使用data-main属性的目录了。
	如果既没有main-data属性,也没有在config中指定,那么BaseUrl将是当前运行require.js的HTML文件所在的目录。

	\paragraph{BaseUrl + path}
	一般情况下,对js文件的查找,是通过BaseUrl + path的方式来查找。在下面的配置中,如果我们查找some/module.
	require.js会查找/another/path/some/v1.0/module.js。
	\begin{lstlisting}
<script src="scripts/require.js"></script>
<script>
  require.config({
    baseUrl: "/another/path",
    paths: {
        "some": "some/v1.0"
    },
    waitSeconds: 15
  });
  require( ["some/module", "my/module", "a.js", "b.js"],
    function(someModule,    myModule) {
        //This function will be called when all the dependencies
        //listed above are loaded. Note that this function could
        //be called before the page is loaded.
        //This callback is optional.
    }
  );
</script>
	\end{lstlisting}

	而在下面的三种情况下,Require.js不会采用BaseUrl + path的方式来查找js文件。
	\begin{enumerate}
	\item 以".js"结束的;
	\item 以"/"开头的;
	\item 以协议头,如"http:"开头的。
	\end{enumerate}

	\paragraph{模块定义}

	模块应该遵循js的模块模式[先省略,日后补充][http://www.adequatelygood.com/JavaScript-Module-Pattern-In-Depth.html]

	简单的键值对,没有依赖,那么将键值对对象作为参数传递给define()
	\begin{lstlisting}
//Inside file my/shirt.js:
define({
    color: "black",
    size: "unisize"
});
	\end{lstlisting}

	如果没有依赖,只是需要函数来做一些先期的设定,那么讲一个函数传递给define()
	\begin{lstlisting}
//my/shirt.js now does setup work
//before returning its module definition.
define(function () {
    //Do setup work here

    return {
        color: "black",
        size: "unisize"
    }
});
	\end{lstlisting}

	如果有依赖,那么define()的一个参数是一个数组,数组元素是依赖的module的muduleID,第二个参数是一个函数,这个函数接受的参数就是你的module依赖的module,顺序和第一个参数数组元素顺序相同,这个函数必须返回一个对象,就是你定义的module。

	\begin{lstlisting}
//my/shirt.js now has some dependencies, a cart and inventory
//module in the same directory as shirt.js
define(["./cart", "./inventory"], function(cart, inventory) {
        //return an object to define the "my/shirt" module.
        return {
            color: "blue",
            size: "large",
            addToCart: function() {
                inventory.decrement(this);
                cart.add(this);
            }
        }
    }
);
	\end{lstlisting}

	在上面的例子中,/my/shirt这个module被创建了,他依赖于/my/cart和/my/inventory.第二个参数的function会在依赖的两个module加载起来之后再被执行。

	当然,这个返回的module不一定需要是个object,返回function对象也同样可以的。

	对于遵循CommonJS的代码,可以采用下面的方式来重用以后代码。

	\begin{lstlisting}
define(function(require, exports, module) {
        var a = require('a'),
            b = require('b');

        //Return the module value
        return function () {};
    }
);
	\end{lstlisting}

	有时候,你会看到define()的第一个参数是你定义的module的moduleID,但是不推荐这么写,因为如果你改变你的module的目录,你需要自己来更改moduleID,所以推荐由Require.js来自动定义你的moduleID。

	\begin{lstlisting}
//Explicitly defines the "foo/title" module:
    define("foo/title",
        ["my/cart", "my/inventory"],
        function(cart, inventory) {
            //Define foo/title object in here.
       }
    );
	\end{lstlisting}

	\paragraph{Configuration}

	有几种方式来调用require.config(),先通过script标签来加载Require.js,然后在下一个script标签中包含对require.config()的调用的代码。
	\begin{lstlisting}
<script src="scripts/require.js"></script>
<script>
  require.config({
    baseUrl: "/another/path",
    paths: {
        "some": "some/v1.0"
    },
    waitSeconds: 15
  });
  require( ["some/module", "my/module", "a.js", "b.js"],
    function(someModule,    myModule) {
        //This function will be called when all the dependencies
        //listed above are loaded. Note that this function could
        //be called before the page is loaded.
        //This callback is optional.
    }
  );
</script>
	\end{lstlisting}
	也可以在main-data属性指定的js文件中调用。在这种情况下,需要注意script标签的异步加载,下面的标签引用的代码可能在main.js之前执行。
	\begin{lstlisting}
<script data-main="scripts/main" src="scripts/require.js"></script>
<script src="scripts/other.js"></script>
	\end{lstlisting}
	\begin{lstlisting}
// contents of main.js:
require.config({
    paths: {
        foo: 'libs/foo-1.1.3'
    }
});
	\end{lstlisting}
	\begin{lstlisting}
// contents of other.js:

// This code might be called before the require.config() in main.js
// has executed. When that happens, require.js will attempt to
// load 'scripts/foo.js' instead of 'scripts/libs/foo-1.1.3.js'
require( ['foo'], function( foo ) {

});
	\end{lstlisting}

	当然,你也可以先定义一个Require的全局config对象,然后再用script标签加载Require.js。使用 require = {},而不是window.require = {}。这样保证在IE下得正确性。

	\begin{lstlisting}
<script>
    var require = {
        deps: ["some/module1", "my/module2", "a.js", "b.js"],
        callback: function(module1, module2) {
            //This function will be called when all the dependencies
            //listed above in deps are loaded. Note that this
            //function could be called before the page is loaded.
            //This callback is optional.
        }
    };
</script>
<script src="scripts/require.js"></script>
	\end{lstlisting}

	\subsection{Underscore.js}

  \subsection{AngularJS}

  

  \section{练习项目}

  \subsection{NuoNuo Picture Gallary}

  初步的设想是开发一个图片管理的website,包括导入图片,给图片加标注,加标签。按日期分类,安标签分类,浏览图片。

  服务器采用NodeJs和Express框架。数据库采用MongoDB.前端框架使用AngularJs,然后需要使用这个地方可以考虑是不是需要使用requireJs之类的加载。

  然后可以练习的包括js加载的优化。

  最先要考虑的是图片保存的策略,这里先采用元数据存储在DB中,查询返回所需要图片的URL,然后浏览器从对应的URL来读取图片。。

  
		
\newpage
\end{CJK}  
\end{document}