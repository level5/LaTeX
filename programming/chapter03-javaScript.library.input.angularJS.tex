\section{AngularJS}
  这里,通过AngularJS自带的例子来学习他。

  首先,从github上clone下整个项目。然后使用npm安装依赖,在这里,即使使用镜像,也可能报错,所以最好使用vpn来翻墙安装依赖包。然后启动应用,这里如果第一次启动了应用,再启动报错的话,可以关闭开关命令窗口,重新启动。原因可能是因为是使用的是ctrl+z来关闭的nodejs,而不是使用的ctrl+c来关闭。

  在读完整个教程,对AngularJS有一个初步的理解之后,开始学习一下AngularJS的基本概念。


  \subsection{Template}
  可以看做HTML的增强。在HTML上增加了angular的一些元素和属性。angular使用template,model和controller来渲染用户在浏览器中看到的view。

  有这样一些元素和属性:
  \begin{itemize}
  \item Directive --
  \item Markup -- \{\{\}\}就是markup,将表达式和元素绑定。 
  \item Filter -- 格式化数据为想要显示的样子。
  \item Form controls -- 验证用户输入。
  \end{itemize}


  \subsection{Controller}

  Controller是一个构造函数,用来提升Angular Scope,稍后解释什么是Angular Scope。

  当一个Controller通过ng-controller和一个DOM绑定,Angular将会通过这个构造方法实例化一个Controller的对象。一个新的子scope将最为参数传递给构造方法,参数名是\$scope。

  可以通过Controller来
  \begin{itemize}
  \item 初始化\$scope对象的属性;
  \item 给\$scope对象添加行为。
  \end{itemize}

  但是不要通过Controller来
  \begin{itemize}
  \item 操作DOM,Controller只应该包含业务逻辑。在Controller中加入展现的逻辑将影响他的可测试性。angular有databinding来处理绝大情况,也可以使用directive来封装DOM操作。
  \item 格式化输入,使用form control来做。
  \item 过滤输出,使用filter来做。
  \item 在Controller共享代码和状态,使用service来完成。
  \item 管理其他部件的生命周期。例如实例化service。
  \end{itemize}

  保持Controller简单,只做单一的view需要的业务逻辑,而将不属于Contoller的逻辑分装在service中,通过依赖注入来一来service。

  You can associate Controllers with scope object implicitly via the ngController directive or \$route service. ??? 这句话真不知道什么意思!!!

  可以这么理解,碰到ngController directive,这个directive会产生一个\$scope,然后调用controller这个构造函数,而对于\$route service,当跳转到相应的hash url时,也会产生一个新的\$scope,并且绑定controller和template,这也是为什么在这种情况下,template中不需要使用ngController这个directive。

  在ngController是嵌套的情况下,他们的scope是可以访问父scope的property,如果他们本身的property没有因为名字相同隐藏了父scope的property的话。

  
  
