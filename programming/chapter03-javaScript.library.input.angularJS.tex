	\section{AngularJS}
	  这里,通过AngularJS自带的例子来学习他。
	
	  首先,从github上clone下整个项目。然后使用npm安装依赖,在这里,即使使用镜像,也可能报错,所以最好使用vpn来翻墙安装依赖包。然后启动应用,这里如果第一次启动了应用,再启动报错的话,可以关闭开关命令窗口,重新启动。原因可能是因为是使用的是ctrl+z来关闭的nodejs,而不是使用的ctrl+c来关闭。
	
	  在读完整个教程,对AngularJS有一个初步的理解之后,开始学习一下AngularJS的基本概念。
	
	
	  \subsection{Template}
	  可以看做HTML的增强。在HTML上增加了angular的一些元素和属性。angular使用template,model和controller来渲染用户在浏览器中看到的view。
	
	  有这样一些元素和属性:
	  \begin{itemize}
	  \item Directive --
	  \item Markup -- \{\{\}\}就是markup,将表达式和元素绑定。 
	  \item Filter -- 格式化数据为想要显示的样子。
	  \item Form controls -- 验证用户输入。
	  \end{itemize}
	
	
	  \subsection{Controller}
	
	  Controller是一个构造函数,用来提升Angular Scope,稍后解释什么是Angular Scope。
	
	  当一个Controller通过ng-controller和一个DOM绑定,Angular将会通过这个构造方法实例化一个Controller的对象。一个新的子scope将最为参数传递给构造方法,参数名是\$scope。
	
	  可以通过Controller来
	  \begin{itemize}
	  \item 初始化\$scope对象的属性;
	  \item 给\$scope对象添加行为。
	  \end{itemize}
	
	  但是不要通过Controller来
	  \begin{itemize}
	  \item 操作DOM,Controller只应该包含业务逻辑。在Controller中加入展现的逻辑将影响他的可测试性。angular有databinding来处理绝大情况,也可以使用directive来封装DOM操作。
	  \item 格式化输入,使用form control来做。
	  \item 过滤输出,使用filter来做。
	  \item 在Controller共享代码和状态,使用service来完成。
	  \item 管理其他部件的生命周期。例如实例化service。
	  \end{itemize}
	
	  保持Controller简单,只做单一的view需要的业务逻辑,而将不属于Contoller的逻辑分装在service中,通过依赖注入来一来service。
	
	  You can associate Controllers with scope object implicitly via the ngController directive or \$route service. ??? 这句话真不知道什么意思!!!
	
	  可以这么理解,碰到ngController directive,这个directive会产生一个\$scope,然后调用controller这个构造函数,而对于\$route service,当跳转到相应的hash url时,也会产生一个新的\$scope,并且绑定controller和template,这也是为什么在这种情况下,template中不需要使用ngController这个directive。
	
	  在ngController是嵌套的情况下,他们的scope是可以访问父scope的property,如果他们本身的property没有因为名字相同隐藏了父scope的property的话。
	  
	  
	 \subsection{Service}
	 
	 \subsection{Provider}
	 
	 对于AngularJS应用,我们使用injector来初始话我们需要的一些有用的对象,service和一些特殊的object,如controller,directive。但是injector需要知道怎么来创建这些对象,这就需要我们来注册一些``菜单''来告诉injector怎么来创建这些对象。总共有五种类型的``菜单'',这里,最啰嗦,但是也是最全面的provider,其他的value,Factory, Service和constant,只是一些provider的语法糖而已。
	 
	 对于被注册的recipe,都有一个对象的标识符,以及怎么来创建这个对象的描述。并且,任何一个recipe都属于一个angular的module,angular的module就是一个用来盛装一个或多个recipe的包裹。而且module还有对其他module依赖的信息。
	  
	  当一个angular应用启动的时候,angular将创建一个injector的实例,他将按顺序穿件一个recipe的登记档,所有定义在core ng module,应用module,以及应用module的依赖的recipe作为一个union登记在其中,当你需要创建一个对象给你的应用时,injector就会查询这个recipe登记档。
	  
	  \paragraph{Value Recipe}
	  
	  我们创建一个简单的service,他提供一个字符串用来最为远程API的授权ID。
	  \begin{JavaScript}
	var myApp = angular.module('myApp', []);	
	myApp.value('clientId', 'a12345654321x');  
	  \end{JavaScript}
	  接着我们通过angular的数据绑定来显示他。
	  \begin{JavaScript}
	myApp.controller('DemoController', ['clientId', function DemoController(clientId) {
  		this.clientId = clientId;
	}]);
	  \end{JavaScript}
	  
	  
	  \begin{HTML}
	<html ng-app="myApp">
  		<body ng-controller="DemoController as demo">
    		Client ID: {{demo.clientId}}
 		</body>
	</html>	  
	  \end{HTML}
	  
	  这个例子使用了value recipe来定义一个简单的service,这个service就是一个字符串。
	  
	  
	  \paragraph{Factory Recipe}
	  
	  value recipe使用简单,但是缺乏一些我们平时需要的很重要的功能,让我们来看看Factory recipe,他就拥有这些重要的功能:
	  
	  \begin{itemize}
	  	\item 可以使用其他service;
	  	\item service的初始化;
	  	\item 延时加载。
	  \end{itemize}
	  
	  
	  Factory recipe通过带有零个或者多个参数的函数来构造新的service,这个函数的返回值就是这个recipe创建的service实例。有一点需要特别注意,angular创建的service都是单例,这就表示injector最多对每个recipe使用一次来穿件service实例,接着,injector将缓存这个实例以供未来使用。
	  
	  我们使用Factory recipe来重写上面的例子。我们可以返回一个字符串字面量。
	  
	  \begin{JavaScript}
myApp.factory('clientId', function clientIdFactory() {
  return 'a12345654321x';
});	  
	  \end{JavaScript}
	  
	  我们也可创建一个service来计算token的值,这个service依赖clientId。
	  \begin{JavaScript}
myApp.factory('apiToken', ['clientId', function apiTokenFactory(clientId) {
  var encrypt = function(data1, data2) {
    // NSA-proof encryption algorithm:
    return (data1 + ':' + data2).toUpperCase();
  };
 
  var secret = window.localStorage.getItem('myApp.secret');
  var apiToken = encrypt(clientId, secret);
 
  return apiToken;
}]);	  
	  \end{JavaScript}
	  
	  
	  \paragraph{Service Recipe}
	  
	  下面是我们定制的Service的构造函数
	  
	  \begin{JavaScript}
function UnicornLauncher(apiToken) {
 
  this.launchedCount = 0;
  this.launch = function() {
    // make a request to the remote api and include the apiToken
    ...
    this.launchedCount++;
  }
}	  
	  \end{JavaScript}

	他依赖于apiToken,我们可以通过Factory Recipe来注册他,通过new操作来生成Service实例。

	\begin{JavaScript}
myApp.factory('unicornLauncher', ["apiToken", function(apiToken) {
  return new UnicornLauncher(apiToken);
}]);	
	\end{JavaScript}	
	
	但是,这个情况下更加适合Service Recipe,他类似于Value和Factory recipe,但是是通过new来调用,他可以带参数,参数也是通过依赖来解决。
	
	\begin{JavaScript}
myApp.service('unicornLauncher', ["apiToken", UnicornLauncher]);	
	\end{JavaScript}  
	
	
	
	\paragraph{Provider Recipe}
	
	
	正如我们前面说的,其他所有的recipe都只是在Provider recipe之上包裹了一层语法糖。Factory Recipe就是注册了一个空的provider recipe,将provider的\$get设置为Factory recipe传入的函数。
	
	大多数情况下,其他recipe已经可以满足我们的需求了,我们使用provider recipe的原因是需要在angular app load起来之前,通过config方法来对service做一些设置。看下面的例子
	
	\begin{lstlisting}
myApp.provider('unicornLauncher', function UnicornLauncherProvider() {
  var useTinfoilShielding = false;
 
  this.useTinfoilShielding = function(value) {
    useTinfoilShielding = !!value;
  };
 
  this.$get = ["apiToken", function unicornLauncherFactory(apiToken) {
 
    // let's assume that the UnicornLauncher constructor was also changed to
    // accept and use the useTinfoilShielding argument
    return new UnicornLauncher(apiToken, useTinfoilShielding);
  }];
});	
	\end{lstlisting}  
	
	\begin{JavaScript}
myApp.config(["unicornLauncherProvider", function(unicornLauncherProvider) {
  unicornLauncherProvider.useTinfoilShielding(true);
}]);	
	\end{JavaScript}
	
	
	可以看到,我们天通过module的API来调用config方法,然后设置了unicornLauncherProvider的参数。在这里,unicorn provider使用过provider injector来注入的,这个injector不同于实例injector。在应用启动完成之前,provider injector初始化所有provider的实例,我们叫他应用生命周期的configuraton阶段。在这个阶段,service是不能够被使用的,因为他们还没有被初始化。
	
	一旦启动阶段结束,和provider的交互将不再被允许,同时开始创建service,我们称之为应用生命周期的运行阶段。
	
	
	\paragraph{Constant Recipe}
	
	因为在应用声明的configuration阶段,所有的service都是不可用的,即使是value recipe,因为他们还没有实例化。但是有些东西,如url的前缀部分,他们没有依赖或者设置,需要在整个生命周期的configuration和run阶段都可以访问,这就是Constant Recipe要做的。
	
	
	\begin{JavaScript}
myApp.constant('planetName', 'Greasy Giant');	
	\end{JavaScript}
	
	
	\begin{JavaScript}
myApp.config(['unicornLauncherProvider', 'planetName', function(unicornLauncherProvider, planetName) {
  unicornLauncherProvider.useTinfoilShielding(true);
  unicornLauncherProvider.stampText(planetName);
}]);	
	\end{JavaScript}
	
	例如,在这个例子里面,unicorn发射的行星名字是在整个应用固定的,所以我们使用constant recipe,并且在configuration阶段提供给provider使用。而由于constant recipe是在整个生命周期都可以访问,所以我们在运行时也可以访问他的值。
	
	
	\begin{JavaScript}
myApp.controller('DemoController', ["clientId", "planetName", function DemoController(clientId, planetName) {
  this.clientId = clientId;
  this.planetName = planetName;
}]);	
	\end{JavaScript}
	  
	\begin{HTML}
<html ng-app="myApp">
  <body ng-controller="DemoController as demo">
   Client ID: {{demo.clientId}}
   <br>
   Planet Name: {{demo.planetName}}
  </body>
</html>	
	\end{HTML}  
	
	
	\paragraph{Special Purpose Objects}
	
	
	\paragraph{结论}
	
	
  
