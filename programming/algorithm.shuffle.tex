\chapter{洗牌算法}


\section{Fisher–Yates shuffle}

\begin{enumerate}
\item 按顺序写下1到N;
\item 取得一个随机数K,范围是1到还没有提出的数字的个数;
\item 从小往大,提取出第K个值,删除他,然后在其他位置写下这个数;
\item 重复第2步,直到所有的数字都提取完毕;
\item 在第3步按顺序写下来的数就是原始数据的一个随机排序。
\end{enumerate}

\begin{center}
\begin{tabular}{|l|l|l|l|}
	\hline
	范围 & 随机数 & 草稿 & 结果\\
	\hline
	 & &1,2,3,4,5,6,7,8& \\
	\hline
	1-8 & 3 & 1,2,\sout{3},4,5,6,7,8 & 3\\
	\hline
	
\end{tabular}
\end{center}
