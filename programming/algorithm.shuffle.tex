\chapter{洗牌算法}

\section{Fisher–Yates shuffle}
\subsection{original method}
\begin{enumerate}
\item 按顺序写下1到N;
\item 取得一个随机数K,范围是1到还没有提出的数字的个数;
\item 从小往大,提取出第K个值,删除他,然后在其他位置写下这个数;
\item 重复第2步,直到所有的数字都提取完毕;
\item 在第3步按顺序写下来的数就是原始数据的一个随机排序。
\end{enumerate}

\begin{tabular}{|c|c|l|l|}
	\hline
	范围 & 随机数 & 草稿 & 结果\\
	\hline
	 & &1,2,3,4,5,6,7,8& \\
	\hline
	1-8 & 3 & 1,2,\sout{3},4,5,6,7,8 & 3\\
	\hline
	1-7 & 4 & 1,2,\sout{3},4,\sout{5},6,7,8 & 3,5\\
	\hline
	1-6 & 5 & 1,2,\sout{3},4,\sout{5},6,\sout{7},8 & 3,5,7\\
	\hline
	1-5 & 3 & 1,2,\sout{3},\sout{4},\sout{5},6,\sout{7},8 & 3,5,7,4\\
	\hline
	1-4 & 4 & 1,2,\sout{3},\sout{4},\sout{5},6,\sout{7},\sout{8} & 3,5,7,4,8\\
	\hline
	1-3 & 1 & \sout{1},2,\sout{3},\sout{4},\sout{5},6,\sout{7},\sout{8} & 3,5,7,4,8,1\\
	\hline
	1-2 & 2 & \sout{1},2,\sout{3},\sout{4},\sout{5},\sout{6},\sout{7},\sout{8} & 3,5,7,4,8,1,6\\
	\hline
	1-1 & 1 & \sout{1},\sout{2},\sout{3},\sout{4},\sout{5},\sout{6},\sout{7},\sout{8} & 3,5,7,4,8,1,6,2\\
	\hline
\end{tabular}
\paragraph{具体实现} C\#实现
\begin{CSharp}[Origin Method]
        public static void OriginMethod(int[] origin, int seed)
        {
            Random rnd = new Random(seed);
            for (int i = origin.Length, j; i > 0; i--)
            {
                int roll = rnd.Next(i);
                int tmp = origin[roll];
                for (j = roll; j < i; j++)
                {
                    origin[j] = origin[j + 1];
                }
                origin[j] = tmp;
            }
        }
\end{CSharp}

\subsection{The modern algorithm}

\begin{CSharp}[Modern Method]
        public static void ModernMethodR(int[] origin, int seed)
        {
            Random rnd = new Random(seed);
            for(int i = origin.Length - 1; i > 0; i--)
            {
                int roll = rnd.Next(i + 1);
                Swap(ref origin[i], ref origin[roll]);
            }
        }

        public static void ModernMethodL(int[] origin, int seed)
        {
            Random rnd = new Random(seed);
            for(int i = 0; i < origin.Length - 1; i++)
            {
                int roll = rnd.Next(i, origin.Length);
                Swap(ref origin[i], ref origin[roll]);
            }
        }


        private static void Swap(ref int x, ref int y)
        {
            int tmp = x;
            x = y;
            y = x;
        }
\end{CSharp}

\subsection{The "Inside-out" algorithm}

\begin{CSharp}[Inside-out]
        public static int[] InsideOut(int[] origin, int seed)
        {
            Random rnd = new Random(seed);

            int[] result = new int[origin.Length];
            for(int i = 0; i < origin.Length; i++)
            {
                int roll = rnd.Next(i + 1);
                if (roll != i)
                {
                    result[i] = result[roll];
                }
                result[roll] = origin[i];
            }

            return result;
        }
\end{CSharp}

\section{一种解决方案}

n个有序数,随机选出m个数。

对每个数进行随机测试,判断是否要选择,通过有序访问整数,可以保证输出结果是有序的。

考虑m=2,n=5的情况。选择第一个整数0的概率为2/5,可以通过语句\lstinline!if(bigrand % 5) < 2!来测试。对于第二个整数1我们不能用相同的概率来选择,决策上要做调整:如果已经选择了0的情况下,1/4的概率选择1;在没有选择零的情况下,2/4的概率选择1。

\begin{CSharp}
        public static int[] Shuffle(int n, int m, int seed)
        {
            Random rnd = new Random(seed);
            int[] ret = new int[m];
            for(int i = 0; i < n; i++)
            {
                if (rnd.Next(n - i) < m)
                {
                    ret[ret.Length - m] = i;
                    m--;
                }
            }
            return ret;
        }
\end{CSharp}        


\section{又一种解决方案}

\begin{CSharp}
        public static void Shuffle(int[] origin, int seed)
        {
            Random rnd = new Random(seed);

            for(int i = 0; i < origin.Length; i++)
            {
                Utils.Swap(ref origin[i], ref origin[rnd.Next(i, origin.Length)]);
            }
        }
\end{CSharp}
