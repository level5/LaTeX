\chapter{HTTP}

\section{概述}


\subsection{资源}


\subsubsection{URI}

\paragraph{URL}
URL 是通过描述资源的位置来标识资源的;


\begin{Bash}[格式]
<scheme>://<user>:<password>@<host>:<port>/<path>;<params>?<query>#<frag>
\end{Bash}

\begin{itemize}
\item schema(大小写无关),无默认值,使用什么协议
\item 服务器,主机和端口(每个schema有特定的默认值,HTTP默认为80端口)。
\item 资源路径
\end{itemize}



\paragraph{URN}
URN则是通过名字来识别资源的‘
\subsection{HTTP方法}

\subsubsection{DELETE, GET, PUT, POST ...}


\subsection{状态码}

\subsubsection{2XX}

\subsubsection{3XX}

\subsubsection{4XX}

\subsubsection{5XX}


\subsection{报文格式}


\subsubsection{起始行}

\subsubsection{首部字段}

\subsubsection{正文}


\subsubsection{使用telnet测试}

原来telnet可以干这个,其实telnet就是打开一个TCP连接,然后自己可以往上面发送内容。


