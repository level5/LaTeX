\chapter{边学Linux边学C语言}


\section{C语言编程规范}

先按下划线命名吧。

\section{语法部分}


\subsubsection{可变参数列表}


\begin{itemize}
\item 是通过宏来实现的;
\item 导入\lstinline$stdarg.h$这个头文件,标准库的一部分使用到一个类型\lstinline$va_list$和三个宏:\lstinline$va_start$, \lstinline$va_arg$, \lstinline$va_end$;
\item 步骤:
\begin{enumerate}
\item 函数申明一个\lstinline$va_list$;
\item 通过调用\lstinline$va_start$来初始化,第一个参数为申明的\lstinline$va_list$,第二个参数为省略号前面最后一个有名字的变量;
\item 使用\lstinline$va_arg$来访问参数,接收两个参数,第一个是\lstinline$va_list$变量,第二个是参数的类型,这也是为什么\lstinline$printf$可以使用不同的参数,因为根据前面格式化字符串来确定后面对应参数类型;
\item 最后当访问完最后一个参数之后,需要调用\lstinline$va_end$ 
\end{enumerate}

\end{itemize}

\begin{C}
/*
 * arguments_list.c
 *
 *  Created on: Jun 12, 2015
 *      Author: root
 */


#include <stdarg.h>

float average(int n_values, ...)
{
	va_list var_arg;
	int count;
	float sum = 0;

	/*
	 * start to visit var list
	 */
	va_start(var_arg, n_values);

	/*
	 *
	 */
	for( count = 0; count < n_values; count +=1 )
	{
		sum += va_arg(var_arg, int);
	}

	/*
	 *
	 */
	va_end(var_arg);

	return sum / n_values;
}

\end{C}

\subsubsection{指针}

\begin{C}[最好使用的写法是这样]
int *a;
\end{C}

\begin{C}[因为这样可能误导以为三个变量都是指针,实际上只有第一个是]
int* b, c, d;
\end{C}

\subsubsection{链接属性 external, internal, none}

\begin{itemize}
\item external,不论声明多少次,位于几个源文件都表示同一个实体;
\item internal,在同一个源文件内的所有声明都是同一个实体;
\item none,多个声明被当做不同的实体;
\end{itemize}

extern和static关键字修改标识符的链接属性,static关键字只对原来是exteranl的属性生效,使其变为internal;extern关键字用于源文件中一个标识符的第一次申明时指定该标识具有external的属性,但是如果它用于该标识的第二次或以后的申明,他并不会改变由第一次申明指定的链接属性。


extern只有在声明的时候才是必需的。因为子文件作用域定义的变量就是external的。

\subsubsection{数组变量和指针变量的区别}

\subsubsection{定义字符串常量}
[http://stackoverflow.com/questions/1431576/how-do-you-declare-string-constants-in-c]

\subsubsection{const在C中和Java中的不同}
[http://stackoverflow.com/questions/27588918/const-and-pointers-in-c]

\begin{C}[为什么这样会有编译错误呢?]
const char *str = "abc";
str[0] = 'd';
\end{C}

这里是C和指针中的解释:

\begin{C}[这两个是相同的,表示a的值不能修改]
int const a;
const int a;
\end{C}


\begin{C}[这个的话,表示指针可以修改,但是指向的值不能修改]
int const *pci;
\end{C}

\begin{C}[这个的话,指针是常量,值可以修改]
int * const cpi;
\end{C}

\begin{C}[这个指针和值都不能修改]
int const * const cpci;
\end{C}

这个按照一篇文章说的来解读变量的定义很方便的,后面找到那个再补充。


这样上面的例子就很清楚了,\lstinline$const char *str$的指向的变量的值是不能修改的,而通过下标取值就是求值的意思,所以编译报错了(看这篇问题里面说,通过这个指针来得到的值,编译器认为就是不可变的,也不知道这种说法对不对,感觉有一定道理)。

\subsection{函数指针,怎么感觉函数名和函数的值是一样的啊}

\begin{C}[这是我不能理解的]
printf("function != &function:%s\n", 
	output_file == &output_file? "true":"false");
\end{C}

根据C和指针上面的解释:初始化表达式中的\&操作符是可选的,因为函数名被使用时,总是由编译器把他转换为函数指针的。\&操作符只是显式的说明了编译器将隐式执行的任务。

\subsection{define宏\#是什么?}

这个是创建运算符(\#),以资源名产生字符串的值。

\begin{C}
#define TEST_NAME 1
#define p(name) print_name(#name, name)

static void print_name(char *name, int value)
{
	printf("value of \"%s\" is %d.\n", name, value);
}

void test_pound_sign()
{
	p(TEST_NAME);
}
\end{C}

\begin{Command-Line}[运行结果]
value of "TEST_NAME" is 1.
\end{Command-Line}