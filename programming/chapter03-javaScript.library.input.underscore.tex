\section{underscore}

返回自己本身的函数,空函数

\begin{JavaScript}
  _.identity = function(value) {
    return value;
  };
  
  _.noop = function(){};
\end{JavaScript}

方法返回传入参数名字为key的property:

\begin{JavaScript}
  _.property = function(key) {
    return function(obj) {
      return obj[key];
    };
  };
\end{JavaScript}

判断对象是否包含指定的key,value对。第8行,此处或后面是用来判断当传入的pair的value是undefined时,obj是否包含key这个property。

\begin{JavaScript}
  _.matches = function(attrs) {
    var pairs = _.pairs(attrs), length = pairs.length;
    return function(obj) {
      if (obj == null) return !length;
      obj = new Object(obj);
      for (var i = 0; i < length; i++) {
        var pair = pairs[i], key = pair[0];
        if (pair[1] !== obj[key] || !(key in obj)) return false;
      }
      return true;
    };
  };
  
  _.has = function(obj, key) {
    return obj != null && hasOwnProperty.call(obj, key);
  }; 
\end{JavaScript}

生成键值对的数组,数组的元素是一个数组,这个数组第一个元素是key,第二个元素是value

\begin{JavaScript}
  _.pairs = function(obj) {
    var keys = _.keys(obj);
    var length = keys.length;
    var pairs = Array(length);
    for (var i = 0; i < length; i++) {
      pairs[i] = [keys[i], obj[keys[i]]];
    }
    return pairs;
  };
\end{JavaScript}
\begin{JavaScript}
  var nativeKeys = Object.keys;
  
  _.keys = function(obj) {
    if (!_.isObject(obj)) return [];
    if (nativeKeys) return nativeKeys(obj);
    var keys = [];
    for (var key in obj) if (_.has(obj, key)) keys.push(key);
    return keys;
  };  
\end{JavaScript}

\begin{JavaScript}
  var createCallback = function(func, context, argCount) {
    if (context === void 0) return func;
    switch (argCount == null ? 3 : argCount) {
      case 1: return function(value) {
        return func.call(context, value);
      };
      case 2: return function(value, other) {
        return func.call(context, value, other);
      };
      case 3: return function(value, index, collection) {
        return func.call(context, value, index, collection);
      };
      case 4: return function(accumulator, value, index, collection) {
        return func.call(context, accumulator, value, index, collection);
      };
    }
    return function() {
      return func.apply(context, arguments);
    };
  };
\end{JavaScript}

用于将context和function绑定。1,2,3个参数的情况下使用对于当前引擎的最佳性能实现。

void expression 表示undefined。void关键字带一个参数,返回的结果永远是undefined
\begin{JavaScript}
  void 0
  void (0)
  void "hello"
  void (new Date())
//all will return undefined
\end{JavaScript}
使用void 0的好处是:保证结果永远是undefined,因为undefined在JavaScript中不是一个关键字。


\begin{JavaScript}
  _.iteratee = function(value, context, argCount) {
    if (value == null) return _.identity;
    if (_.isFunction(value)) return createCallback(value, context, argCount);
    if (_.isObject(value)) return _.matches(value);
    return _.property(value);
  };
\end{JavaScript}

内部使用,对于不同的value返回不同的函数(待补充)

\subsection{Collections Function}
\paragraph{each, forEach}对每个值执行函数。
\begin{JavaScript}
  _.each = _.forEach = function(obj, iteratee, context) {
    if (obj == null) return obj;
    iteratee = createCallback(iteratee, context);
    var i, length = obj.length;
    if (length === +length) {
      for (i = 0; i < length; i++) {
        iteratee(obj[i], i, obj);
      }
    } else {
      var keys = _.keys(obj);
      for (i = 0, length = keys.length; i < length; i++) {
        iteratee(obj[keys[i]], keys[i], obj);
      }
    }
    return obj;
  };
\end{JavaScript}
	
length === +length 判断length是否是数字。

数组就迭代数组,否则迭代所有属性执行callback function。

\paragraph{map, collect}返回一个数组,结果是使用每个值来执行函数的返回值的集合。
\begin{JavaScript}
  _.map = _.collect = function(obj, iteratee, context) {
    if (obj == null) return [];
    iteratee = _.iteratee(iteratee, context);
    var keys = obj.length !== +obj.length && _.keys(obj),
        length = (keys || obj).length,
        results = Array(length),
        currentKey;
    for (var index = 0; index < length; index++) {
      currentKey = keys ? keys[index] : index;
      results[index] = iteratee(obj[currentKey], currentKey, obj);
    }
    return results;
  };
\end{JavaScript}	

傻了!!! 上面的代码可以拆分为:

\begin{JavaScript}
  var keys = obj.length !== +obj.length && _.keys(obj); 
  var length = (keys || obj).length;
  var results = Array(length);
  var currentKey;
\end{JavaScript}

\paragraph{reduce} 从集合通过function来生成一个单一的值,如果不提供memo对象,则取集合中的第一个对象作为memo

\begin{JavaScript}
  var reduceError = 'Reduce of empty array with no initial value';
  
  _.reduce = _.foldl = _.inject = function(obj, iteratee, memo, context) {
    if (obj == null) obj = [];
    iteratee = createCallback(iteratee, context, 4);
    var keys = obj.length !== +obj.length && _.keys(obj),
        length = (keys || obj).length,
        index = 0, currentKey;
    if (arguments.length < 3) {
      if (!length) throw new TypeError(reduceError);
      memo = obj[keys ? keys[index++] : index++];
    }
    for (; index < length; index++) {
      currentKey = keys ? keys[index] : index;
      memo = iteratee(memo, obj[currentKey], currentKey, obj);
    }
    return memo;
  };
\end{JavaScript}

第10行,如果memo不存在,而且是个空的集合,报错。

第11行,如果memo不存在,则取集合中的第一个元素作为memo。

回调函数接受4个参数,memo, value, index(key),集合对象。

\paragraph{reduceRight} 前面的reduce是从左往右,这个方法是从右往左。

\begin{JavaScript}
  var reduceError = 'Reduce of empty array with no initial value';	

  _.reduceRight = _.foldr = function(obj, iteratee, memo, context) {
    if (obj == null) obj = [];
    iteratee = createCallback(iteratee, context, 4);
    var keys = obj.length !== + obj.length && _.keys(obj),
        index = (keys || obj).length,
        currentKey;
    if (arguments.length < 3) {
      if (!index) throw new TypeError(reduceError);
      memo = obj[keys ? keys[--index] : --index];
    }
    while (index--) {
      currentKey = keys ? keys[index] : index;
      memo = iteratee(memo, obj[currentKey], currentKey, obj);
    }
    return memo;
  };
\end{JavaScript}

和前一个方法不同就是从后往前迭代。

这里可以看到0转换为boolean型是false。

\paragraph{some} 判断是否有一个测试返回true。

\begin{JavaScript}
  _.some = _.any = function(obj, predicate, context) {
    if (obj == null) return false;
    predicate = _.iteratee(predicate, context);
    var keys = obj.length !== +obj.length && _.keys(obj),
        length = (keys || obj).length,
        index, currentKey;
    for (index = 0; index < length; index++) {
      currentKey = keys ? keys[index] : index;
      if (predicate(obj[currentKey], currentKey, obj)) return true;
    }
    return false;
  };
\end{JavaScript}

\paragraph{find} 返回第一个满足测试的条件的值

\begin{JavaScript}

  _.find = _.detect = function(obj, predicate, context) {
    var result;
    predicate = _.iteratee(predicate, context);
    _.some(obj, function(value, index, list) {
      if (predicate(value, index, list)) {
        result = value;
        return true;
      }
    });
    return result;
  };
\end{JavaScript}


\paragraph{filter} 返回所有符合测试的值的数组

\begin{JavaScript}
  _.filter = _.select = function(obj, predicate, context) {
    var results = [];
    if (obj == null) return results;
    predicate = _.iteratee(predicate, context);
    _.each(obj, function(value, index, list) {
      if (predicate(value, index, list)) results.push(value);
    });
    return results;
  };
\end{JavaScript}

\paragraph{reject} 所有测试不通过的值得数组,通过调用filter方法来实现。

\begin{JavaScript}
  _.negate = function(predicate) {
    return function() {
      return !predicate.apply(this, arguments);
    };
  };
  
  _.reject = function(obj, predicate, context) {
    return _.filter(obj, _.negate(_.iteratee(predicate)), context);
  };
\end{JavaScript}

\paragraph{every} 所有测试都通过的时候返回true, 没有什么特别不好理解的。

\begin{JavaScript}
  _.every = _.all = function(obj, predicate, context) {
    if (obj == null) return true;
    predicate = _.iteratee(predicate, context);
    var keys = obj.length !== +obj.length && _.keys(obj),
        length = (keys || obj).length,
        index, currentKey;
    for (index = 0; index < length; index++) {
      currentKey = keys ? keys[index] : index;
      if (!predicate(obj[currentKey], currentKey, obj)) return false;
    }
    return true;
  };
\end{JavaScript}

\paragraph{contains} 也没有啥不好懂的。
 
\begin{JavaScript}
  _.contains = _.include = function(obj, target) {
    if (obj == null) return false;
    if (obj.length !== +obj.length) obj = _.values(obj);
    return _.indexOf(obj, target) >= 0;
  };
\end{JavaScript}

\paragraph{invoke} 调用集合中的每个元素的一个方法(使用参数)

\begin{JavaScript}
  _.invoke = function(obj, method) {
    var args = slice.call(arguments, 2);
    var isFunc = _.isFunction(method);
    return _.map(obj, function(value) {
      return (isFunc ? method : value[method]).apply(value, args);
    });
  };
\end{JavaScript}

使用\_.map来调用方法,收集结果。

第2行,将传入的第三个参数起作为参数传给每个方法。

如果method是function,就将每个元素作为上下文来调用这个方法,如果method是方法名,则调用每个元素上的这个方法。

\paragraph{pluck} 用来取得所有元素的某个property的数组。

\begin{JavaScript}
  _.pluck = function(obj, key) {
    return _.map(obj, _.property(key));
  };
\end{JavaScript}

\_.property方法返回的是一个方法,这个方法会返回传入的对象的名字为key的property的值。


\paragraph{where, findWhere} where是找到所有包含attrs上键值对的item。使用\.filter来过滤,过滤条件是\lstinline!_.matches!; findWhere则是找到第一个满足条件的item

\begin{JavaScript}
  _.where = function(obj, attrs) {
    return _.filter(obj, _.matches(attrs));
  };
  
  _.findWhere = function(obj, attrs) {
    return _.find(obj, _.matches(attrs));
  };
\end{JavaScript}

\paragraph{max, min}

\begin{JavaScript}
  _.max = function(obj, iteratee, context) {
    var result = -Infinity, lastComputed = -Infinity,
        value, computed;
    if (iteratee == null && obj != null) {
      obj = obj.length === +obj.length ? obj : _.values(obj);
      for (var i = 0, length = obj.length; i < length; i++) {
        value = obj[i];
        if (value > result) {
          result = value;
        }
      }
    } else {
      iteratee = _.iteratee(iteratee, context);
      _.each(obj, function(value, index, list) {
        computed = iteratee(value, index, list);
        if (computed > lastComputed || computed === -Infinity && result === -Infinity) {
          result = value;
          lastComputed = computed;
        }
      });
    }
    return result;
  };
\end{JavaScript}

分两种情况处理,如果不存在回调方法,直接查找最大的值就可以了。

如果存在回调方法。则使用回调方法来计算比较的值。

第16行,||后部分是用来处理回调计算结果就是\lstinline!-Infinity!,而且result可能没有被赋值(也可能已经赋值为-Infinity,但这个时候不影响结果)的情况下,给result赋值。


\begin{JavaScript}
  _.min = function(obj, iteratee, context) {
    var result = Infinity, lastComputed = Infinity,
        value, computed;
    if (iteratee == null && obj != null) {
      obj = obj.length === +obj.length ? obj : _.values(obj);
      for (var i = 0, length = obj.length; i < length; i++) {
        value = obj[i];
        if (value < result) {
          result = value;
        }
      }
    } else {
      iteratee = _.iteratee(iteratee, context);
      _.each(obj, function(value, index, list) {
        computed = iteratee(value, index, list);
        if (computed < lastComputed || computed === Infinity && result === Infinity) {
          result = value;
          lastComputed = computed;
        }
      });
    }
    return result;
  };
\end{JavaScript}

和max的情况刚好相反。没有什么需要理解的地方了。

\paragraph{shuffle}洗牌算法,这里是用的是Fisher-Yates shuffle, inside-out方法。

算法大致思路是,有一个原始数组,一个拷贝。连续增大的i,每次去随机数在0~i之间,如果随机数等于i,就将随机数位置的值拷贝到i位置,然后从源数组将值拷贝到随机数的位置。

\begin{JavaScript}
  _.shuffle = function(obj) {
    var set = obj && obj.length === +obj.length ? obj : _.values(obj);
    var length = set.length;
    var shuffled = Array(length);
    for (var index = 0, rand; index < length; index++) {
      rand = _.random(0, index);
      if (rand !== index) shuffled[index] = shuffled[rand];
      shuffled[rand] = set[index];
    }
    return shuffled;
  };
\end{JavaScript}


\paragraph{sample}


\paragraph{omit} 生成新的对象,并忽略参数部分
\begin{JavaScript}
  _.omit = function(obj, iteratee, context) {
    if (_.isFunction(iteratee)) {
      iteratee = _.negate(iteratee);
    } else {
      var keys = _.map(concat.apply([], slice.call(arguments, 1)), String);
      iteratee = function(value, key) {
        return !_.contains(keys, key);
      };
    }
    return _.pick(obj, iteratee, context);
  };
\end{JavaScript}

参数有两种类型,一种是第二个是回调函数,第三个是上下文。第二种是第二个起是要排除的key的名字。
\begin{itemize}
\item 当第二个参数是函数的时候,表示返回值是true的时候需要omit,这个时候需要取反一下再来执行\lstinline!_.pick!函数。
\item 当后面是key的名字时,从第5行开始,先选取了从第2个参数起的所有参数,然后转换成String。

这里有一点是\lstinline!String()!和\lstinline!new String()!都可以得到一个String。
new操作符到底做了一些什么呢?
假设有一个函数foo,那么\lstinline!new foo()!发生了什么?
\begin{enumerate}
\item 一个新的对象创建了,继承了,新对象的constructor是foo(这点可以用来判断是不是在通过new来调用foo)
\item 这个新创建的对象被当做this,使用指定的参数来调用构造方法foo。这里new foo和new foo()是等效的
\item 构造方法return的值被当做整个new表达式的返回值。如果对象没有显示的返回值得时候,第一步创建的对象会被当做整个表达式的值。我发现如果返回的是基本类型的话,第一步创建的对象也会被当做表达式的值返回
\end{enumerate}
\begin{JavaScript}[可以理解new foo是这样一个过程,但是好像又有点点不同,因为在chrome的显示上是不一样的]
function createFoo()
{
	var obj = {};
	obj.__proto__ = foo.prototype;
	var result = foo.apply(obj, arguments);
	// 此处判断不一定
	if (result == undefined || typeof result == 'string' || typeof result == 'number' || typeof result == 'boolean') 
	{
		return obj;
	} else 
	{
		return result;
	}
}
\end{JavaScript}

如何做到是否使用new都返回新的实例?
\begin{JavaScript}
function foo()
{
	if (!(this instanceof foo)) {
		return new foo();
	}
	// 此处初始化对象
	this.bar = 100;
	// ...
}

console.log(foo());
console.log(new foo);
console.log(new foo());
\end{JavaScript}

这里为什么要转换为String的原因后面补充,感觉和\lstinline!_.contains!方法有关系。

然后再定义了一个函数,如果key包含在传入的参数中,则返回false,表示不挑选。最后执行\lstinline!~.pick!来选取需要的属性。这里可以看到context传入的其实是我们要omit的key的名字或者undefined,但是因为我们的iteratee没有使用\lstinline!this!关键字,所以无所谓输入值是什么。
\end{itemize}


\paragraph{pick}挑选给定的属性来生成新的对象,和上面的omit刚好是反过来的,所以omit可以直接反过来调用此方法。
\begin{JavaScript}
  _.pick = function(obj, iteratee, context) {
    var result = {}, key;
    if (obj == null) return result;
    if (_.isFunction(iteratee)) {
      iteratee = createCallback(iteratee, context);
      for (key in obj) {
        var value = obj[key];
        if (iteratee(value, key, obj)) result[key] = value;
      }
    } else {
      var keys = concat.apply([], slice.call(arguments, 1));
      obj = new Object(obj);
      for (var i = 0, length = keys.length; i < length; i++) {
        key = keys[i];
        if (key in obj) result[key] = obj[key];
      }
    }
    return result;
  };
\end{JavaScript}
\begin{itemize}
\item 传入的是函数的情况下,使用for in语法,直接将所有函数返回true的属性拷贝到result中。
\item 传入的是key的时候,生成keys的数组,然后循环历遍数组的判断是否存在于obj对象中。第12行感觉是用来将基本类型转换成对象。因为如果我们使用\lstinline!"a" in 1!的话,是会跑出TypeError的异常的。
\end{itemize}

