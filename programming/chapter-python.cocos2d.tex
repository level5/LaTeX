\chapter{cocos2d}

\section{Concept}

\subsection{Scenes}
游戏中独立的一个场景,比如说菜单,关卡1,关卡2。

\subsection{Director}
控制各个scene之间切换的组件。

\subsection{Layers}
用来在Z轴上组织scene的的各层。比如:
\begin{itemize}
\item 背景:背景风景
\item 远景:装饰的树和鸟
\item 中景:平台
\item 近景:玩家,敌人,金币
\item HUD:Heads Up Desiplay。游戏状态,比如血量,经验,敌人状态
\end{itemize}

Layer是定义行为和外观的地方,所以大部分时间编程都是在和Layer打交道。

Layer是定义事件Handler的地方,Event从最前面的层像最后面的层发出,知道有一层接受了处理了这个事件。

cocos定义好的一些Layers:
\begin{itemize}
\item MultiplexLayer, 一组Layers,但是同一时间只有一个Layer可见。
\item ScrollingManager, ScrollableLayer,用来控制可见区域。
\item RectMapLayer, HexMapLayer,展现一组title
\item Menu
\item ColorLayer,实心的有颜色的矩形
\item InterpreterLayer, Director控制弹出的可交互控制台,可以在我们的scene中弹出。
\end{itemize}

Layer是CocosNode的子类,所以他们能够手动或者使用Action来transform。

\subsection{Sprites}
sprite是一个2D的图片,可以移动,旋转,缩放,动画等。

Sprites 是一组sprite,对sprites做transform,所有的孩子都会被transform。

sprite是CocosNode的子类。

\subsection{Events}
cocos2d使用  The pyglet Event Framework来处理事件。

\begin{itemize}
\item emitters, 发射器。pyglet.event.EventDispatcher
\item 每一个emitter可以注册自己想要的事件。通过一个事件名称字符串来标识
\item Example emitter:
\begin{Python}
class Bunker(pyglet.event.EventDispatcher):
    ...
    def building_update(self, dt):
        ...
    if self.elapsed_time>self.building_time:
        self.dispatch_event('on_building_complete', self)

def take_damage(self, damage):
    self.dispatch_event('on_building_under_attack', self)
    self.life -= damage
    if self.life<0:
        self.dispatch_event('on_building_destroyed', self)

# following lines register the events that Bunker instances can emit
Bunker.register_event_type('on_building_complete')
Bunker.register_event_type('on_building_under_attack')
Bunker.register_event_type('on_building_destroyed')
\end{Python}
\item Example Listener:
\begin{Python}
class Commander(object):
    def __init__(self, ...):
        self.buildings = []
        ...

    def invest_resources(self):
        ...
        bunker = Bunker(...)
        # register to receive all events from object bunker
        bunker.push_handlers(self)
        self.buildings.append(bunker)
        ...

    # handlers for the events

    def on_building_complete(self, building):
        ...

    def on_building_under_attack(self, building):
        ...

    def on_building_destroyed(self, building):
        ...
\end{Python}
\item 当想要移除listener的时候,可以使用:
\begin{Python}
emitter.remove_handlers(...) # params as in push_handlers
\end{Python}
\end{itemize}
