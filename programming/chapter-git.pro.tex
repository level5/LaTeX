\chapter{Pro Git}

\section{Install}
这一步没有什么好说的.

\section{Config}
安装好Git之后,第一步就是要进行配置工作.通过一个叫做Git config的工具.通过命令 git -config来执行.

专门用来配置或读取相应的工作环境变量。而正是由这些环境变量,决定了 Git 在各个环节的具体工作方式和行为。这些变量可以存放在以下三个不同的地方:
\begin{itemize}
\item /etc/gitconfig文件:系统中对所有用户都普遍适用的配置。若使用 git config 时用 -{}-system 选项,读写的就是这个文件。

\item \textasciitilde/.gitconfig文件:用户目录下的配置文件只适用于该用户。若使用 git config 时用 -{}-global 选项,读写的就是这个文件。

\item 当前项目的 git 目录中的配置文件(也就是工作目录中的 .git/config 文件):这里的配置仅仅针对当前项目有效。每一个级别的配置都会覆盖上层的相同配置,所以.git/config 里的配置会覆盖 /etc/gitconfig 中的同名变量。

\end{itemize}

\subsection{用户信息}
第一个要配置的是你个人的用户名称和电子邮件地址。这两条配置很重要,每次 Git 提交时都会引用这两条信息,说明是谁提交了更新,所以会随更新内容一起被永久纳入历史记录:
\begin{CMD}
$ git config --global user.name "John Doe"
$ git config --global user.email johndoe@example.com
\end{CMD}

如果用了 --global 选项,那么更改的配置文件就是位于你用户主目录下的那个,如果要在某个特定的项目中使用其他名字或者电邮,只要去掉 -{}-global 选项重新配置即可.

\subsection{文本编辑器}
Git 需要你输入一些额外消息的时候,会自动调用一个外部文本编辑器给你用。默认会使用操作系统指定的默认编辑器,一般可能会是Vi 或者 Vim。
\begin{CMD}
$ git config --global core.editor emacs
\end{CMD}

\subsection{对比工具}

\begin{CMD}
$ git config --global merge.tool vimdiff
\end{CMD}

\subsection{查看配置}
\begin{CMD}
$ git config --list
user.name=Scott Chacon
user.email=schacon@gmail.com
color.status=auto
color.branch=auto
color.interactive=auto
color.diff=auto
$git config user.name
Scott Chacon
\end{CMD}

\section{Git 基本操作}

\subsection{取得Git仓库}
\begin{itemize}
\item 通过导入所有文件来创建新的 Git 仓库
\item 从已有的 Git 仓库克隆出一个新的镜像仓库来。
\end{itemize}

\paragraph{当前目录初始化}
\begin{CMD}
git init
\end{CMD}